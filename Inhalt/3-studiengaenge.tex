
\section{Studiengänge}

In diesem Kapitel wird euch ein kurzer Überblick über euren Studiengang gegeben. Aktuelle Informationen zu den je Studiengang belegbaren Modulen und Regelstudienpläne findet ihr jeweils in den Modulhandbüchern auf den Internetseiten der Fakultäten.
Bei Fragen oder wenn ihr genauere Informationen sucht, solltet ihr euch zunächst an den Fachschaftsrat oder euren Studiengangsberater wenden.
In einigen Fällen wird euch der Fachschaftsrat nicht mehr weiter helfen können.
Dann wird der Studiengangsberater oft der einzige sein, der euch bei Problemen einen Lösungsweg oder eine Lösung anbieten kann.
Der Studiengangsberater ist euer Ansprechpartner, wenn ihr Fragen oder Probleme
mit der Studienordnung, mit der Wahl von Modulen oder anderen \enquote{technischen Problemen} des Studiums habt.
Er kennt die Studienordnung genau und hat daher einen tiefen Einblick in die Details des Studiums.

\textbf{Studienberater Mathematik und Wirtschaftsmathematik}\\
Dr.\ Hans-Georg Rackwitz\\
Theodor-Lieser-Str.\ 5, Raum 127\\
Tel: (0345) 55 24608\\
\email{hans-georg.rackwitz@mathematik.uni-halle.de}\\

\textbf{Studienberater Informatik und Bioinformatik}\\
Apl.\ Prof.\ Dr.\ Klaus Reinhardt\\
Von-Seckendorff-Platz 1, Raum 2.10\\
Tel: (0345) 55 24770\\
\email{klaus.reinhardt@informatik.uni-halle.de}


%%%%%%%%%%%%%%%%%%%%%%%%% INFORMATIK %%%%%%%%%%%%%%%%%%%%%%%%%%%%%%

\subsection{Informatik}
\label{studiengang_informatik}

Liebe Studienanfängerinnen, liebe Studienanfänger,\par

es freut uns sehr, dass ihr euch für ein Studium der Informatik an der MLU entschieden habt. Die folgenden Seiten enthalten einige Informationen über den prinzipiellen Ablauf des Studiums, welche euch den Einstieg erleichtern sollen.\par

In den ersten Fachsemestern werden alle Grundkenntnisse der Informatik und Mathematik erworben, auf welchen später das gesamte Studium aufbaut.
In den Vorlesungen wird euch neue Stoff vorgetragen und durch gezielte Fragen eurerseits können hier bereits erste Unklarheiten beseitigt werden.
Zur Festigung des vermittelten Wissens erfolgt die Bearbeitung von Aufgaben in den zur jeweiligen Vorlesung gehörigen Übungen in kleinen Gruppen mit je einem Übungsleiter.
In der Regel müsst ihr 50\%~bis~60\% dieser Aufgaben korrekt gelöst haben, um das Modul erfolgreich abschließen zu können, weswegen die Bildung von Lerngruppen sehr empfehlenswert ist.
Habt ihr die meisten Aufgaben durchgearbeitet und den Lösungsweg verstanden, sollte es kein Problem mehr sein, die Prüfungen zu bestehen.

In der Regel werdet ihr im 3.~Fachsemester eine Vorlesung in dem von euch frei wählbaren Anwendungsfach hören.
Hier werden euch Grundkenntnisse des jeweiligen Gebietes vermittelt.
Geeignet sind beispielsweise Chemie, Physik, Biologie, Mathematik, Geowissenschaften, Psychologie, BWL oder VWL.

Ab dem fünften Semester könnt ihr Spezialisierungsmodule wählen, welche insgesamt 15 LP einbringen müssen.
Welche Module das sind, wird euch im Gegensatz zum übrigen Regelstudienplan allerdings nicht fest vorgeschrieben.
Ihr habt die Freiheit, aus einer ganzen Reihe von Modulen diejenigen zu wählen, die euch am meisten Spaß machen.
Diese Module werden typischerweise im Bereich der Informatik liegen, allerdings könnt ihr auch ausgewählte Vorlesungen aus eurem jeweiligen Anwendungsfach oder aus den Bereichen Mathematik, Wirtschafts- oder Bioinformatik hören und als Spezialisierungsmodul anrechnen lassen.

Neben diesen Pflichtveranstaltungen gibt es insbesondere in den ersten Semestern oftmals die Möglichkeit, Tutorien zusätzlich zu den regulären Übungen zu besuchen.
Im Gegensatz zu den Übungen werden diese in den meisten Fällen von Studenten geleitet, wodurch die Barriere zwischen Studierenden und Dozenten abgebaut wird.
In den Tutorien könnt ihr ganz ungezwungen Fragen stellen, über den Vorlesungstoff oder Übungsaufgaben sprechen oder euch Vorlesungsinhalte anhand von Beispielen näherbringen lassen. Einige Professoren finden es auch gut, wenn viele Fragen während der Vorlesung gestellt werden und es so zu Diskussionen kommt.

Wenn ihr noch Fragen habt, zögert nicht den Fachschaftsrat oder euren Studienberater Dr.~Bauer anzusprechen.

\textbf{Ansprechpartner im Fachschaftsrat:}\\
Peter Böttcher, \email{peter.boettcher@fsr-matheinfo.de}\\
%%%%%%%%%%%%%%%%%%%%%%%%%%%%%%%%%%%% BIOINFORMATIK %%%%%%%%%%%%%%%%%%%%%%%%%%%

\newpage

\subsection{Bioinformatik}
\label{studiengang_bioinformatik}


Liebe Studienanfängerinnen und Studienanfänger,\par
für die meisten von euch beginnt ein neuer Lebensabschnitt:
Ihr habt euch entschieden, Bioinformatik zu studieren und damit für ein nicht ganz einfaches Studium.
Im Bereich der Informatik wird von euch hohes Abstraktionsvermögen und mathematische Logik verlangt.
In den biowissenschaftlichen Fächern müsst ihr beweisen, dass ihr auch mal viel pauken könnt.
Die Anforderungen an diese interdisziplinäre Fachrichtung sind umfangreich – aber keine Sorge:
Mit ein bisschen Fleiß und genügend Kraft werdet ihr das Studium meistern.

Danach stehen euch viele Optionen zur Verfügung.
Bioinformatiker haben sehr gute Vermittlungschancen, was den Beruf angeht.
Sowohl viele Firmen, als auch wissenschaftliche Institute suchen gut ausgebildete Bioinformatiker.
Der Abschluss ist kein klassischer Abschluss, er ist modern und deswegen kann man sagen, dass ihr Sachen lernt, die ihr später auch so anwenden könnt.
Für viele ist auch ein anschließender Master-Studiengang eine Option.
Mit dem Master in der Tasche erhöhen sich auch die Chancen, später den „Traumberuf“ ausführen zu können.

Ihr werdet in der Regel Informatik- und Mathematikveranstaltungen mit Informatikern besuchen und einige Biologievorlesungen gemeinsam mit Biologen und Biochemikern hören.
Die Chemie- und einige Biologieveranstaltungen sind für euch „zugeschnitten“.
Im späteren Studienverlauf kommen auch interdisziplinäre Veranstaltungen (wie z.B. \textit{Algorithmen auf Sequenzen}) dazu.
Grundsätzlich wird jedes Modul mit einer Klausur oder einer mündlichen Prüfung abgeschlossen und geht gewichtet in die Abschlussbenotung ein.

Die Informatik- und Mathematikveranstaltungen sind vor allem von den wöchent\-lichen „Hausaufgaben“, den Übungen, geprägt.
Diese solltest du auf keinen Fall vernachlässigen, da du sonst weder das nötige Wissen erlangst, um die Klausuren zu bestehen, noch die nötigen Punkte, um das Modul abschließen zu können.
Du musst meistens 50\% (bei manchen Professoren auch 60\%) der Übungsaufgaben richtig gelöst haben, um das Modul zu absolvieren.
Sicherlich wirst du am Anfang Probleme haben die Übungen zu lösen, aber lass dich davon nicht entmutigen!
Es wäre geradezu unmenschlich, wenn du auf Anhieb alles richtig lösen könntest.

Bildet Lerngruppen, um euch den Aufgaben gemeinsam zu stellen und um euch gegenseitig die Lösungen zu erklären.
Nur so könnt ihr auf Dauer die erforderlichen Punkte erreichen.
Wenn ihr euch allerdings nicht mit dem Stoff befasst, sondern nur abschreibt, verursacht das am Ende des Semesters deutlich mehr Arbeit, weil ihr alles nacharbeiten müsst.
Unter Umständen merkt ihr dann auch sehr schnell, dass ihr so viel Rückstand habt, dass ihr den Stoff bis zum Klausurtermin nicht mehr aufholen könnt. Habt ihr jedoch fleißig mitgearbeitet, so müsst ihr für die Prüfung nur nochmal alles wiederholen.
Grundsätzlich hat sich gezeigt:
Wer im Semester die Übungsaufgaben selbst bearbeitet und ohne „Mogelei“ die erforderlichen Punkte erreicht, wird die Prüfung bestehen und das auch nicht zu knapp.

Die Bio- und Chemievorlesungen laufen ganz anders ab.
Hier müsst ihr im Laufe des Semester keine einzige Leistung erbringen.
Erst am Ende kommt die Prüfung, für die ihr dann auch richtig viel lernen müsst.
Es lohnt sich natürlich auch hier schon während des Semesters auf zu passen und nachzuarbeiten.

Es ist immer ratsam, sich für die Vorlesungen das Skript (meist wird es offiziell von den Professoren zur Verfügung gestellt) und ältere Klausuren zu Rate zu ziehen.
Wenn ihr hier noch Fragen habt, dann sprecht auch mal ältere Studenten oder den Fachschaftsrat an.
Die Übungen sind auch eine sehr gute Möglichkeit, die Fragen los zu werden.
Oft werden Übungsserien nur wiederholt und vorgerechnet, aber die Übung ist die beste Möglichkeit schnell zu richtigen Antworten zu kommen.

Wenn ihr noch Fragen oder Probleme habt, dann stehen euch der Studienberater Dr.~Bauer und der Fachschaftsrat jederzeit zur Verfügung.
Es gilt wie immer: Fragen kostet nichts!

\textbf{Ansprechpartner im Fachschaftsrat:}\\
Tugce Kuru, \email{rene.geiss@fsr-matheinfo.de}


%%%%%%%%%%%%%%%%%%%%%%%%%%% LEHRAMT INFO %%%%%%%%%%%%%%%%%%%%%%%%%

\newpage

\subsection{Informatik Lehramt}
\label{studiengang_infolehramt}


Liebe Studienanfängerinnen und Studienanfänger,\\
willkommen an der Martin-Luther-Universität Halle-Wittenberg.
Schön, dass ihr euch gerade für diesen Studiengang entschieden habt, denn von euch gibt es hier noch nicht so viele.

Am Besten lest ihr euch die Texte für Informatik und für die Lehrämter der Mathematik durch.

Ansonsten kann euch jetzt noch Eines mit auf den Weg gegeben werden:
Dieses Studium ist kein Leichtes aber ihr werdet euch sicher nach ein paar Wochen hinein gefunden haben.
Sprecht mit euren Kommilitoninnen und Kommilitonen, bildet Gruppen und dann werdet ihr sicher auch Erfolg haben.

Die Pflichtmodule sind dieselben, bis auf dass LAG-Studenten das Modul \textit{Automaten und Berechenbarkeit} noch im Bereich der Informatikpflichtmodule haben.

Falls ihr ein Problem haben solltet, wendet euch einfach an den Fachschaftsrat, wir sind für euch da.


\textbf{Ansprechpartner im Fachschaftsrat:}\\
Peter Böttcher, \email{peter.boettcher@fsr-matheinfo.de}\\

\newpage

%%%%%%%%%%%%%%%%%%%%%%%%% MATHEMATIK %%%%%%%%%%%%%%%%%%%%%%%%%%%%%%%%
\subsection{Mathematik}
\label{studiengang_mathematik}

Liebe Studienanfängerinnen und Studienanfänger,\par

herzlich willkommen an der Martin-Luther-Universität und herzlichen Glückwunsch zu eurer Entscheidung für ein Bachelor-Studium der Mathematik.
Euch erwartet ein vielseitiges und spannendes, aber auch sehr anspruchsvolles Studium, in dem ihr unter anderem ein sehr viel umfassenderes Bild der Mathematik gewinnen werdet, als es der Schulunterricht vermitteln kann und will.
Ihr werdet Mathematik als eine Wissenschaft kennenlernen, die sich mit präziser Denk- und Arbeitsweise gleichermaßen vielen abstrakten Fragestellungen wie auch angewandten Problemen aus den Naturwissenschaften, der Medizin, den Wirtschaftswissenschaften und anderen Bereichen widmet.

Da sich die Mathematik praktisch in sehr vielen Bereichen wiederfindet, ist die berufliche Situation für Absolventen sehr gut.
Die wahrscheinlich bekanntesten Berufsfelder liegen im Bank- und Versicherungswesen.
Eine ebenfalls sehr attraktive Beschäftigungsmöglichkeit ist sicherlich die des wissenschaftlichen Mitarbeiters an der Universität.
Aber auch in anderen Bereichen gibt es viele Möglichkeiten der Beschäftigung mit einem abgeschlossenen Mathematikstudium.

Dieses Studium besteht aus einer soliden mathematischen Ausbildung, die von Studienbeginn an zu selbstständigem Arbeiten anhält. Das in den Vorlesungen vermittelte Wissen wird in den zugehörigen Übungen durch die Bearbeitung von Übungsauf\-gaben gefestigt.
An dieser Stelle legen wir euch ans Herz, die Übungsaufgaben sorgfältig zu bearbeiten. Diese sind nämlich meist Voraussetzung für die Teilnahme an den Prüfungen und dienen allem voran der Festigung des Stoffes, was das Lernen für Klausuren oder für mündliche Prüfungen erleichtert und vor allem verkürzt. 
In den ersten beiden Fachsemestern werden in den Grundlagenvorlesungen \textit{Analysis}, \textit{Lineare Algebra} und \textit{Numerik} unverzichtbare Grundkenntnisse und Methoden der Mathematik erworben, welche eine solide Grundlage für das gesamte Mathematikstudium bilden. Zusätzlich werden Grundkenntnisse in Informatik im Rahmen der Module \textit{Objektorientierte Programmierung} und \textit{Datenstrukturen und Effiziente Algorithmen I}, welche ihr durch die Wahl des Anwendungsfaches Informatik noch erweitern könnt. Weitere wählbare
Anwendungsfächer sind Biologie, Chemie, Physik, und Wirtschaftswissenschaft.

Zu Beginn des Studiums gibt es neben Vorlesungen und Übungen auch noch Tutorien und Workshops.
Studenten aus höheren Semestern wiederholen hier den wichtigsten Stoff der Vorlesung und beantworten eure Fragen dazu. Diese Veranstaltungen sind zwar freiwillig, aber dennoch ein wichtiger Bestandteil der Wissensvermittlung. Sie ersetzen jedoch nicht das Selbststudium daheim. 

Aber auch die Verbindung zur Praxis soll nicht zu kurz kommen. Dazu dient das Praktikum im vierten Semester.
Es sollte in einem wirtschaftlichen Betrieb absolviert werden, kann aber auch in einem anderen Institut der Martin-Luther-Universität stattfinden.
In dem Praktikum sollen typische Studieninhalte des Studienganges zur Anwendung kommen.
Dabei muss es von einer Hochschullehrerin bzw. einem Hochschullehrer des Instituts für Mathematik betreut werden.

\textbf{Ansprechpartner im Fachschaftsrat:}\\
Alexander Klemps, \email{alexander.klemps@fsr-matheinfo.de}\\
Lukas Rettler, \email{lukas.rettler@fsr-matheinfo.de}\\

\newpage

%%%%%%%%%%%%%%%%%%%%%%%%%% WIRTSCHAFTSMATHE %%%%%%%%%%%%%%%%%%%%%%%%%%%%
\subsection{Wirtschaftsmathematik}
\label{studiengang_wima}

Liebe Studienanfängerinnen, liebe Studienanfänger,\par

herzlich willkommen an der Martin-Luther-Universität und herzlichen Glückwunsch zu eurer Entscheidung für ein Bachelor-Studium der Wirtschaftsmathematik.

Mit diesem Studium soll der Spagat zwischen Mathematik und Wirtschaft gelingen.
Das heißt neben der Beschäftigung mit der Mathematik (vgl. Informationen für Bachelor Mathematik auf Seite~\pageref{studiengang_mathematik}) stehen noch Betriebs- und Volkswirtschaftlehre auf dem Stundenplan.
Hier sind während des gesamten Studiums fünf Fächer zu belegen.
In den ersten beiden Semestern bieten sich z.B. \textit{Grundlagen der BWL} oder \textit{Grundlagen der VWL} und \textit{Mikroökonomik~I} an.
Weitere Möglichkeiten findet ihr in der Prüfungsordnung.

Wirtschaft wirkt am Anfang meist sehr einfach, dadurch nimmt man es gern auf die leichte Schulter.
Die Klausuren, für die man sich rechtzeitig im Löwenportal anmelden muss, sind aber nicht immer ein Kinderspiel, deswegen versucht kontinuierlich den Stoff zu lernen und auch mit anderen (auch VWL- und BWL-Studenten) zusammen zu arbeiten.

Diese Gruppenarbeit ist in der Mathematik das wichtigste, weil der Stoff am Anfang gar nicht einfach, sondern oft unverständlich wirkt.
Hier sind die Übungen das Wichtigste zum Verstehen der mathematischen Inhalte.
In den ersten beiden Semestern hört ihr Analysis I und II, Lineare Algebra I und II, sowie Optimierung.
Es ist in den Mathematik-Modulen immer sehr wichtig, die Übungsaufgaben zu bearbeiten, die euch beim Verstehen des Stoffes helfen und außerdem oft Voraussetzung sind, um an den Prüfungen teilzunehmen.
Neben den beiden Bestandteilen Wirtschaft und Mathematik müsst ihr euch auch noch mit Informatik „herumärgern“.

Wir hoffen, ihr beißt euch durch die ersten Semester, denn das sind die Schwierigsten.
Wir mussten uns auch erst an die neue Denk- und Arbeitsweise gewöhnen.
Also gebt nicht auf!
Wenn ihr Hilfe braucht, könnt ihr euch an uns oder euren Studienberater, Dr.~Rackwitz, wenden.

\textbf{Ansprechpartner im Fachschaftsrat:}\\
Alexander Klemps, \email{alexander.klemps@fsr-matheinfo.de}\\

\newpage

%%%%%%%%%%%%%%%%%%%%%%%%%%%%% MATHE LAG %%%%%%%%%%%%%%%%%%%%%%%%%%%%%

\subsection{Mathematik Lehramt}

Ihr wollt also Lehrer werden?!\\
Gute Wahl! Denn Mathe-Lehrer sind mehr gesucht als eh und je! 
Hier vorab ein paar Infos, worauf ihr euch einlasst: Auch wenn ihr in der Schule nicht zu den Einserkandidaten in Mathematik gehört habt, habt ihr durchaus reale Chancen dieses komplexe Studium erfolgreich zu bestreiten.
Es kommt nur auf Eines an: Durchbeißen.
Und das heißt in den ersten zwei Semestern am Ball bleiben, wenn ihr mit den Bachelor-Mathematikern zusammen \textit{Analysis} und \textit{Lineare Algebra} hört.
Und auch wenn es schwer fällt:
Aufstehen, zur Vorlesung gehen, fleißig Übungsaufgaben lösen und auf jeden Fall mit anderen Leuten zusammentun, damit die Verzweiflung mit anderen geteilt werden kann.

Und nein: Ihr seid nicht blöd.
Man kann als normalsterbliches Wesen (fast) nie alle Aufgaben fehlerfrei lösen, aber dafür gibt es auch die 50\% (bei manchen Profs auch 60\%) Hürde ($\rightarrow$  genaueres steht in den Modulbeschreibungen bzw. wird zu Beginn des Semesters bekannt gegeben).
Wenn ihr euch regelmäßig mit den Aufgaben beschäftigt und sie auch wirklich verstanden habt, sollte es auch gut machbar sein, die abschließenden Klausuren zu bestehen.

Lasst euch aber nicht allzu schnell entmutigen.
Rauft euch zusammen.
Löst die Aufgaben zusammen.
Erklärt’s euch gegenseitig.
Schließlich werdet ihr ja Lehrer.

Leider ist das noch nicht alles, denn neben der Mathematik studiert ihr ja noch ein anderes Fach (manche sogar noch zwei) und das erziehungswissenschaftliche Begleitstudium mit den Fächern Pädagogik und Psychologie muss auch noch belegt werden.
Da ist euer Organisationstalent gefragt.
Informiert euch so früh wie möglich über die Veranstaltungszeiten (\link{http://studip.uni-halle.de} oder Aushänge) und sprecht bei Überschneidungen die entsprechenden Dozenten an.
Manchmal kann da noch geschoben werden, machmal aber auch nicht.
Dann versucht vielleicht schon einmal eine andere Veranstaltung zu besuchen, wenn ihr die Voraussetzungen dafür schon habt.

Für Psychologie und Pädagogik ist zusätzlich zu beachten, dass die Einschreibungen momentan online im Stud.IP erfolgen und nur eine begrenzte Anzahl von Plätzen verfügbar ist (Einschreibungen enden dafür schon sehr zeitig.).
Versucht dann lieber Vorlesungen, für die es oft keine Begrenzungen gibt, im 1.~Semester zu besuchen, wenn ihr dafür noch Zeit habt.

Eine Zwischenprüfung im eigentlichen Sinne gibt es nicht mehr.
Ersetzt wird diese durch die jeweiligen Modulleistungen am Ende des Semesters.
Für all diejenigen von euch, die BAföG erhalten ist es also wichtig euer Studium so zu gestalten, dass ihr am Ende des 4.~Semesters auf 105~Leistungspunkte kommt.
Davon dürfen maximal 30 in Form von Prüfungen angemeldet sein, der Rest muss schon bestanden sein.
Dabei ist es zumindest derzeit noch egal, woher diese stammen, sei es euer zweites Fach oder aus Pädagogik und Psychologie.

Bei Problemen und Fragen stehen euch neben den Vertretern des Fachschaftsrates auch sehr gerne der gelegentlich zusammenkommende Lehrerstammtisch und Herr Kortenkamp zur Verfügung.
Er ist für die Mathematik-Didaktik zuständig, womit ihr ab dem dritten Semester zu tun habt.

Einer der größten Unterschiede zwischen dem LAG- und dem LAS-Studium besteht darin, dass die angehenden Sekundarschullehrer erst im dritten Semester die \textit{Analysis I} besuchen und nicht wie die angehenden Gymnasiallehrer schon im ersten Semester.
Stattdessen besuchen die LAS-Studenten die Veranstaltung \textit{Elemente der Mathematik}.

\textbf{Ansprechpartner im Fachschaftsrat:}\\
Florian Johnke, \email{florian.johnke@fsr-matheinfo.de},\\
Sarah Henkel, \email{sarah.henkel@fsr-matheinfo.de}\\
Larissa Zuralski, \email{larissa.zuralski@fsr-matheinfo.de}




\label{studiengang_lehramt}

\label{studiengang_lag}

\label{studiengang_las}



