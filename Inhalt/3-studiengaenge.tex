
\section{Studiengänge}

In diesem Kapitel wird euch ein kurzer Überblick über euren Studiengang gegeben.
Das wird in den meisten Fällen alles sein, was du brauchst, aber falls du noch Fragen hast oder genauere Informationen suchst, dann solltest du dich zunächst an den Fachschaftsrat oder deinen Studiengangsberater wenden.
In einigen Fällen wird dir der Fachschaftsrat nicht mehr weiter helfen können.
Dann wird der Studiengangsberater oft der einzige sein, der euch bei Problemen einen Lösungsweg oder eine Lösung anbieten kann.
Der Studiengangsberater ist euer Ansprechpartner, wenn ihr Fragen oder Probleme mit der Studienordnung, mit der Wahl von Modulen oder anderen „technischen Problemen“ des Studiums habt.
Er kennt die Studienordnung genau und hat daher einen tiefen Einblick in die Details des Studiums.

\textbf{Studienberater Mathematik}\\
Dr. Hans-Georg Rackwitz\\
Herrenstraße 20, Raum 127\\
Tel: (0345)55 24608	\\
\email{hans-georg.rackwitz@mathematik.uni-halle.de}\\

\textbf{Studienberater Informatik und Bioinformatik}\\
Dr. Christoph Bauer\\
Von-Seckendorff-Platz 1, Raum 2.21\\
Tel: (0345) 55 24718\\
\email{christoph.bauer@informatik.uni-halle.de}


%%%%%%%%%%%%%%%%%%%%%%%%% INFORMATIK %%%%%%%%%%%%%%%%%%%%%%%%%%%%%%

\subsection{Informatik}
\label{studiengang_informatik}
    
\begin{table}[btp]
    \begin{small}
    \begin{tabularx}{\textwidth}{|X||c|c|c|c|c|c||r|}
        \hline
        \textbf{Modul}&\multicolumn{6}{l||}{\textbf{Leistungspunkte im Sem.}}&\textbf{LP}\\\hline
        &1&2&3&4&5&6&\\\hline\hline
        \multicolumn{8}{|c|}{\textbf{Informatik-Grundlagen}}\\\hline
        Objektorientierte Programmierung&5&&&&&&5\\\hline
        Einführung in die Rechnerarchitektur&5&&&&&&5\\\hline
        Mathematische Grundlagen der Informatik und Konzepte der Modellierung &7&8&&&&&15\\\hline
        Einführung in Betriebssysteme&&5&&&&&5\\\hline
        Konzepte der Programmierung&&&5&&&&5\\\hline
        Automaten und Berechenbarkeit&&&&10&&&10\\\hline
        Einführung in die technische Informatik&&5&&&&&5\\\hline
        Datenstrukturen und effiziente Algorithmen~I&&5&&&&&5\\\hline\hline
%        \multicolumn{1}{c||}{Summe}&17&23&5&10&&&55\\\hline\hline
        \multicolumn{8}{|c|}{\textbf{Mathematik}}\\\hline
        Diskrete Strukturen, lineare Algebra und Analysis (Mathematik B)&8&7&&&&&15\\\hline
        Stochastik für Informatiker&&&&5&&&5\\\hline\hline
%        \multicolumn{1}{c||}{Summe}&8&7&5&&&&20\\\hline\hline
        \multicolumn{8}{|c|}{\textbf{Informatik-Vertiefung}}\\\hline
        Datenbanken~I&&&10&&&&10\\\hline
        Datenstrukturen und effiziente Algorithmen~II&&&5&&&&5\\\hline
        Einführung in die Rechnernetze und verteilte Systeme&&&&&5&&5\\\hline
        Softwaretechnik&&&5&&&&5\\\hline
        Einführung in die Computergrafik oder Einführung in die Bildverarbeitung&&&&5&&&5\\\hline
        Gestaltung und Durchführung von Fachvorträgen in der Informatik&&&&&5&&5\\\hline
        Projektpraktikum&&&&5&10&&15\\\hline\hline
%        \multicolumn{1}{c||}{Summe}&&&20&10&20&&50
        \multicolumn{8}{|c|}{\textbf{Nebenfach}}\\\hline
        Anwendungsfach&&&5&5&5&&15\\\hline\hline
        \multicolumn{8}{|c|}{\textbf{Pflichtbereich ASQ}}\\\hline
        Allgemeine Schlüsselqualifikationen&5&&&&&5&10\\\hline\hline
%        \multicolumn{1}{|c||}{Summe}&5&&&&&5&10\\\hline\hline
        \textbf{Bachelor-Arbeit}&&&&&&15&15\\\hline
    \end{tabularx}
    \end{small}
    \caption{Regelstudienplan für den Bachelor Informatik (180~LP)
    \label{plan-info}. Hinzu kommen noch Veranstaltungen zur „Spezialisierung“, siehe Tabelle~\ref{plan-info2}}
\end{table}

\begin{table}[!th]
    \begin{small}
    \begin{tabularx}{\textwidth}{|X||c|c|c|c|c|c||r|}
        \hline
        \textbf{Modul}&\multicolumn{6}{l||}{\textbf{Leistungspunkte im Sem.}}&\textbf{LP}\\\hline
        &1&2&3&4&5&6&\\\hline\hline
        \multicolumn{8}{|X|}{\textbf{Spezialisierung}}\\\hline
        Informatik&&&&&0/5&0/5/10&0/5/10/15\\
        Wirtschaftsinformatik&&&&&0/5&0/5/10&0/5/10/15\\
        Bioinformatik&&&&&0/5&0/5/10&0/5/10/15\\
        Mathematik&&&&&&0/5&0/5\\
        Anwendungsfach&&&&&&0/5&0/5\\\hline
    \end{tabularx}
    \end{small}
    \caption{Spezialiserung im Studiengang Bachelor Informatik (180~LP) im 5. und 6.~Semester. \label{plan-info2}}
\end{table}

Liebe Informatikerin, lieber Informatiker,\\
Du findest auf der Institutsseite und anschließend in diesem Heft einen anschaulichen Regelstudienplan.
Dieser empfiehlt dir für die kommenden drei Jahre die von dir zu belegenden Veranstaltungen, welche meist im Institut für Informatik stattfinden.

In den ersten Fachsemestern werden alle Grundkenntnisse der Informatik und Mathematik erworben, worauf später das gesamte Studium aufbaut.
In den Vorlesungen wird dir und deinen Kommilitonen der neue Stoff vorgetragen und durch deine gezielten Fragen erste Unklarheiten beseitigt.
Die dazugehörigen Übungen finden in kleinen Gruppen mit einem Übungsleiter statt.
Dabei wird durch die Bearbeitung von Aufgaben der Stoff weiter vertieft.
In der Regel musst du 50\%~-~60\% dieser Aufgaben korrekt gelöst haben, um das Modul erfolgreich abschließen zu können.
Aus diesem Grund ist das Bilden von kleinen Lerngruppen sehr empfehlenswert.
Hast du alle Aufgaben durchgearbeitet und verstanden, sollte es kein Problem mehr sein, die Prüfungen zu bestehen.

Es dürfte im 3.~Fachsemester ein Anwendungsfach hinzu kommen, welches du frei wählen darfst.
Hier werden dir Grundkenntnisse des jeweiligen Gebietes vermittelt.
Geeignet sind beispielsweise Chemie, Physik, Biologie, Mathematik, Geowissenschaften, Psychologie, BWL oder VWL.

Ab dem fünften Semester kannst du dir deine Spezialisierungsmodule aussuchen, für die du 15 LP aufbringen musst.
Welche Module das sind, wird dir im Gegensatz zum übrigen Regelstudienplan allerdings nicht fest vorgeschrieben.
Du hast die Freiheit, aus einer ganzen Reihe von Modulen diejenigen zu wählen, die dir am meisten Spaß machen.
Diese Module werden typischerweise im Bereich der Informatik liegen, allerdings kannst du dir auch ausgewählte Vorlesungen aus deinem Anwendungsfach oder aus den Bereichen Mathematik, Wirtschafts- oder Bioinformatik anhören und als Spezialisierungsmodul anrechnen lassen.

Neben diesen Pflichtveranstaltungen gibt es vor allem in den ersten Semestern oftmals die Möglichkeit, sogenannte Tutorien zu besuchen.
Diese sind freiwillige Lehrveranstaltungen, die als Ergänzung zu manchen Fächern angeboten werden.
Im Gegensatz zu den Übungen werden die Tutorien größtenteils von Studenten geleitet, wodurch die Barriere zwischen Studierenden und Dozenten abgebaut wird.
In den Tutorien kannst du ganz unverbindlich Fragen stellen, über den Vorlesungstoff oder Übungsaufgaben sprechen oder dir Vorlesungsinhalte an Hand von Beispielen verinnerlichen.

Wenn du noch Fragen hast, zögere nicht den Fachschaftsrat oder deinen Studienberater Dr.~Bauer anzusprechen.

\textbf{Ansprechpartner im Fachschaftsrat:}\\
Alexander Weiß, Felix Knispel, Martin Schneider, Delia Elmrich \\
\email{alexander.weiss@fsr-matheinfo.de},\\
\email{felix.knispel@fsr-matheinfo.de},\\
\email{martin.schneider@fsr-matheinfo.de}\\
\email{delia.elmrich@fsr-matheinfo.de}
%%%%%%%%%%%%%%%%%%%%%%%%%%%%%%%%%%%% BIOINFORMATIK %%%%%%%%%%%%%%%%%%%%%%%%%%%

\subsection{Bioinformatik}
\label{studiengang_bioinformatik}

\begin{table}[tbp]
\begin{small}
\begin{tabularx}{\textwidth}{|X||c|c|c|c|c|c||r|}
    \hline
    \textbf{Modul}&\multicolumn{6}{l||}{\textbf{Leistungspunkte im Sem.}}&\textbf{LP}\\\hline
    &1&2&3&4&5&6&\\\hline\hline
    \multicolumn{8}{|c|}{\textbf{Pflichtbereich Informatik}}\\\hline
    Objektorientierte Programmierung&5&&&&&&5\\\hline
    Mathematische Grundlagen der Informatik und Konzepte der Modellierung&7&8&&&&&15\\\hline
    Softwaretechnik&&&5&&&&5\\\hline
    Datenstrukturen und effiziente Algorithmen~I&&5&&&&&5\\\hline
    Datenbanken I&&&&&10&&10\\\hline
    Algorithmen auf Sequenzen~I&&&5&&&&5\\\hline
    Statistische Datenanalyse in der Bioinformatik~I&&&&&5&&5\\\hline
    Spezielle Probleme der Bioinformatik &&&&10&&&10\\\hline\hline
%    Summe&12&13&10&10&15&60\\\hline\hline
    \multicolumn{8}{|c|}{\textbf{Pflichtbereich Mathematik}}\\\hline
    Diskrete Strukturen, lineare Algebra und Analysis (Mathematik B)&8&7&&&&&15\\\hline
    Stochastik&&&&5&&&5\\\hline\hline
%    Summe&8&7&&5&&20\\\hline\hline
    \multicolumn{8}{|c|}{\textbf{Pflichtbereich Biologie}}\\\hline
    Zellbiologie&5&&&&&&5\\\hline
    Botanik für Bioinformatiker&&&5&&&&5\\\hline
    Zoologie für Bioinformatiker&&&5&&&&5\\\hline
    Ökologie für Bioinformatiker&&&&5&&&5\\\hline
    Genetik für Bioinformatiker&&&5&&&&5\\\hline
    Mikrobiologie für Bioinformatiker&&&&5&&&5\\\hline\hline
%    Summe&5&&15&10&&30\\\hline\hline
    \multicolumn{8}{|c|}{\textbf{Pflichtbereich Biochemie}}\\\hline
    Allgemeine Biochemie&&&&&10&&10\\\hline\hline
%    Summe&&&10&&&10\\\hline\hline
    \multicolumn{8}{|c|}{\textbf{Pflichtbereich Chemie}}\\\hline
    Allgemeine und Grundlagen der physikalischen Chemie&5&&&&&&5\\\hline
    Organische und Bioorganische Chemie&&5&5&&&&10\\\hline\hline
%    Summe&5&5&5&&&15\\\hline\hline
    \multicolumn{8}{|c|}{\textbf{Pflichtbereich ASQ}}\\\hline
    Allgemeine Schlüsselqualifikationen&&5&&5&&&10\\\hline\hline
    \textbf{Bachelor-Arbeit}&&&&&&15&15\\\hline
%    Summe&&5&5&&&10\\\hline
    \end{tabularx}
    \end{small}
    \caption{Regelstudienplan für den Bachelor Bioinformatik (180~LP). Hinzu kommen Veranstaltungen aus dem Wahlbereich, s. Tabelle \ref{plan-bioinfo2} \label{plan-bioinfo}}
\end{table}

\begin{table}[!th]
    \begin{small}
    \begin{tabularx}{\textwidth}{|X||r|}
        \hline
        \textbf{Modul}&\textbf{LP}\\\hline\hline
        \multicolumn{2}{|c|}{\textbf{Wahlbereich Informatik}}\\\hline
        Automaten und Berechenbarkeit&10\\\hline
        Datenstrukturen und effiziente Algorithmen~II&5\\\hline
        Einführung in Rechnernetze und verteilte Systeme&5\\\hline
        Einführung in Rechnerarchitektur&5\\\hline
        Konzepte der Programmierung&5\\\hline
        Einführung in die Computergrafik&5\\\hline
        Theorie der Datensicherheit&5\\\hline
        Einführung in die Bildverarbeitung&5\\\hline
        Komponenten- und Servicebasierte Software&5\\\hline
        Einführung in die künstliche Intelligenz&5\\\hline
        Grundlagen des WWW&5\\\hline\hline
%        Summe&10&5&5&20&15&55\\\hline\hline
        \multicolumn{2}{|c|}{\textbf{Wahlbereich biowissenschaftlich orientierte Fächer}}\\\hline
        Orientierungsmodul Biologie&5\\\hline
        Pflanzenphysiologie Bioinformatiker&5\\\hline
        Spezielle Mikrobiologie für Bioinformatiker&5\\\hline
        Tierphysiologie für Bioinformatiker&5\\\hline
        Ökologiepraktikum&5\\\hline
        Populationsgenetik für Bioinformatiker&5\\\hline
        Biogeographie&5\\\hline
        Molekulare Genetik für Bioinformatiker&5\\\hline
        Biochemie und Biotechnologie für Bioinformatiker (Fortgeschrittene)&10\\\hline
        Grundlagen der Genetik&5\\\hline
        Molekularbiologie in der Tierzucht&5\\\hline
        Molekulargenetik der Nutzpflanzen~I&5\\\hline
    \end{tabularx}
    \end{small}
    \caption{Veranstaltungen in den wahlbereichen für das Studienfach Bachelor Bioinformatik (180~LP). Du musst in jedem Wahlbereich auf 10~Leistungspunkte kommen.\label{plan-bioinfo2}}
\end{table}

Liebe Studienanfängerinnen und Studienanfänger,\\
für die meisten von euch beginnt ein neuer Lebensabschnitt:
Ihr habt euch entschieden, Bioinformatik zu studieren und damit für ein nicht ganz einfaches Studium.
Im Bereich der Informatik wird von euch hohes Abstraktionsvermögen und mathematische Logik verlangt.
In den biowissenschaftlichen Fächern müsst ihr beweisen, dass ihr auch mal viel pauken könnt.
Die Anforderungen an diese interdisziplinäre Fachrichtung sind umfangreich – aber keine Sorge:
Mit ein bisschen Fleiß und genügend Kraft werdet ihr das Studium meistern.

Danach stehen euch viele Optionen zur Verfügung.
Bioinformatiker haben sehr gute Vermittlungschancen, was den Beruf angeht.
Sowohl viele Firmen, als auch wissenschaftliche Institute suchen gut ausgebildete Bioinformatiker.
Der Abschluss ist kein klassischer Abschluss, er ist modern und deswegen kann man sagen, dass ihr Sachen lernt, die ihr später auch so anwenden könnt.
Für viele ist auch ein anschließender Master-Studiengang eine Option.
Mit dem Master in der Tasche erhöhen sich auch die Chancen, später den „Traumberuf“ ausführen zu können.

Ihr werdet Informatik- und Mathematikveranstaltungen mit Informatikern haben und Biovorlesungen mit Biologen und Biochemikern hören.
Die Chemie- und einige Biologieveranstaltungen sind für euch „zugeschnitten“.
Im späteren Studienverlauf kommen auch interdisziplinäre Veranstaltungen (wie z.B. „Algorithmen auf Sequenzen“) dazu.
Grundsätzlich wird jedes Modul mit einer Klausur oder einer mündlichen Prüfung abgeschlossen und geht gewichtet in die Abschlussbenotung ein.

Die Informatik- und Mathematikveranstaltungen sind vor allem von den wöchent\-lichen „Hausaufgaben“, den Übungen, geprägt.
Diese solltest du auf keinen Fall vernachlässigen, da du sonst weder das nötige Wissen erlangst, um die Klausuren zu bestehen, noch die nötigen Punkte, um das Modul abschließen zu können.
Du musst meistens 50\% (bei manchen Professoren auch 60\%) der Übungsaufgaben richtig gelöst haben, um das Modul zu absolvieren.
Sicherlich wirst du am Anfang Probleme haben die Übungen zu lösen, aber lass dich davon nicht entmutigen!
Es wäre geradezu unmenschlich, wenn du auf Anhieb alles richtig lösen könntest.

Bildet Lerngruppen, um euch den Aufgaben gemeinsam zu stellen und um euch gegenseitig die Lösungen zu erklären.
Nur so könnt ihr auf Dauer die erforderlichen Punkte erreichen.
Wenn ihr euch allerdings nicht mit dem Stoff befasst, sondern nur abschreibt, verursacht das am Ende des Semesters deutlich mehr Arbeit, weil ihr alles nacharbeiten müsst.
Unter Umständen merkt ihr dann auch sehr schnell, dass ihr so viel Rückstand habt, dass ihr den Stoff bis zum Klausurtermin nicht mehr aufholen könnt. Habt ihr jedoch fleißig mitgearbeitet, so müsst ihr für die Prüfung nur nochmal alles wiederholen.
Grundsätzlich hat sich gezeigt:
Wer im Semester die Übungsaufgaben selbst bearbeitet und ohne „Mogelei“ die erforderlichen Punkte erreicht, wird die Prüfung bestehen und das auch nicht zu knapp.

Die Bio- und Chemievorlesungen laufen ganz anders ab.
Hier müsst ihr im Laufe des Semester keine einzige Leistung erbringen.
Erst am Ende kommt die Prüfung, für die ihr dann auch richtig viel lernen müsst.
Es lohnt sich natürlich auch hier schon während des Semesters auf zu passen und nachzuarbeiten.

Es ist immer ratsam, sich für die Vorlesungen das Skript (meist wird es offiziell von den Professoren zur Verfügung gestellt) und ältere Klausuren zu Rate zu ziehen.
Wenn ihr hier noch Fragen habt, dann sprecht auch mal ältere Studenten oder den Fachschaftsrat an.
Die Übungen sind auch eine sehr gute Möglichkeit, die Fragen los zu werden.
Oft werden Übungsserien nur wiederholt und vorgerechnet, aber die Übung ist die beste Möglichkeit schnell zu richtigen Antworten zu kommen.

Die Tabelle~\ref{plan-bioinfo} auf Seite~\pageref{plan-bioinfo} zeigt eure Pflichtveranstaltungen.
Die ersten Wahlveranstaltungen solltet ihr frühestens im 5.~Semester wählen, siehe Tabelle~\ref{plan-bioinfo2} auf Seite~\pageref{plan-bioinfo2}.
Um dazu mehr Informationen zu finden, solltet ihr euch eure Prüfungs- und Studienordnungen durchlesen.

Wenn ihr noch Fragen oder Probleme habt, dann stehen euch der Studienberater, Dr.~Bauer, und der Fachschaftsrat jederzeit zur Verfügung.
Es gilt wie immer: Fragen kostet nichts!

\textbf{Ansprechpartner im Fachschaftsrat:}\\
Iris Eckert\\
\email{iris.eckert@fsr-matheinfo.de}


%%%%%%%%%%%%%%%%%%%%%%%%%%% LEHRAMT INFO %%%%%%%%%%%%%%%%%%%%%%%%%

\subsection{Informatik Lehramt}
\label{studiengang_infolehramt}

\begin{table}[tbp]
    \begin{footnotesize}
    \begin{tabularx}{\textwidth}{|b{.4\textwidth}|X|X|X|}
        \hline
        Modultitel                                                            & LP & Note in Abschlusszeugnis & Studien\-semester \\
        \hline
        \multicolumn{4}{|c|}{\textbf{Pflichtmodule Informatik 70}}\\\hline
        Objektorientierte Programmierung                                      &  5 & Ja   & 1. \\
        Einführung in Rechnerarchitektur                                      &  5 & Ja   & 1. \\
        Mathematische Grundlagen der Informatik und Konzepte der Modellierung & 15 & Nein & 1. und 2. \\
        Datenstrukturen und effiziente Algorithmen I                          &  5 & Ja   & 2. oder 4. \\
        Technische Informatik, Betriebssysteme und Rechnernetze (Lehramt)     &  5 & Nein & 3. \\
        Konzepte der Programmierung                                           &  5 & Nein & 3. \\
        Datenbanken I                                                         & 10 & Ja   & 3. , 5. oder 7. \\
        Automaten und Berechenbarkeit*                                        & 10 & Ja   & 4. oder 6. \\
        Softwaretechnik (Lehramt)                                             &  5 & Ja   & 5. \\
        Informatik und Gesellschaft                                           &  5 & Nein & 5. oder später \\
        \hline
        \multicolumn{4}{|c|}{\textbf{Wahlmodule Informatik 10(5)}}\\\hline
        Algorithmen auf Sequenzen I                      & 5 & Nein & 5. oder später \\
        Datenstrukturen und effiziente Algorithmen II    & 5 & Nein & 5. oder später \\
        Einführung in die Bildverarbeitung               & 5 & Nein & 5. oder später \\
        Einführung in die Computergrafik                & 5 & Nein & 5. oder später \\
        Einführung in die KI                             & 5 & Nein & 5. oder später \\
        Einführung in Rechnernetze und verteilte Systeme & 5 & Nein & 5. oder später \\
        Grundlagen des WWW                               & 5 & Nein & 5. oder später \\
        Komponenten- und serviceorientierte Software     & 5 & Nein & 5. oder später \\
        Theorie der Datensicherheit                      & 5 & Nein & 5. oder später \\
        \hline
        \multicolumn{4}{|c|}{\textbf{Fachdidaktik Informatik 15}}\\\hline
        Didaktik der Informatik AB  & 5 & Ja   & 3. oder 4. \\
        Didaktik der Informatik CDE & 5 & nein & ab 4. \\
        Didaktik der Informatik FG  & 5 & Ja   & ab 5. \\
        \hline
    \end{tabularx}
    \end{footnotesize}
    \caption{\label{plan-laI}Regelstudienplan für den Studiengang Informatik LAG bzw. LAS. Das mit * gekennzeichnete Modul sowie die Wahlmodule sind nur von den LAGs zu belegen. Von den Wahlmodulen sind von den LAGs ein bzw. zwei Fächer, falls Informatik das erste Fach ist, zu belegen.}
\end{table}

Liebe Studienanfängerinnen und Studienanfänger,\\
willkommen an der Martin-Luther-Universität Halle-Wittenberg.
Schön, dass ihr euch gerade für diesen Studiengang entschieden habt, denn von euch gibt es hier noch nicht so viele.

Am Besten lest ihr euch die Texte für die Informatiker für den informatischen Teil und für den Lehramtsteil den Text für die Lehrämter der Mathematik durch.

Ansonsten kann euch jetzt noch eines mit auf den Weg gegeben werden:
Dieses Studium ist kein Leichtes aber ihr werdt euch sicher nach ein paar Wochen hinein gefunden haben.
Sprecht mit euren Kommilitoninnen und Kommilitonen, bildet Gruppen und dann werdet ihr sicher auch Erfolg haben.

Ihr findet in Tabelle \ref{plan-laI} auf Seite \pageref{plan-laI} einen Regelstudienplan, der euch eine Empfehlung gibt, wann ihr welche Module besuchen solltet.
Der Unterschied  -- wie ihr dort sehen werdet -- zwischen Lehramt an Gymnasien und an Sekundarschulen ist nicht sehr groß:
Die Pflichtmodule sind dieselben, bis auf dass LAG-Studenten das Modul „Automaten und Berechenbarkeit“ noch im Bereich der Informatikpflichtmodule haben.

Falls ihr ein Problem haben solltet, wendet euch einfach an den Fachschaftsrat, wir sind für euch da.

\textbf{Ansprechpartner im Fachschaftsrat:}\\
Mara Jakob\\
\email{mara.jakob@fsr-matheinfo.de}


%%%%%%%%%%%%%%%%%%%%%%%%% MATHEMATIK %%%%%%%%%%%%%%%%%%%%%%%%%%%%%%%%
\subsection{Mathematik}
\label{studiengang_mathematik}

Liebe Studienanfängerinnen und Studienanfänger,\\
herzlich willkommen an der Martin-Luther-Universität und herzlichen Glückwunsch zu eurer Entscheidung für ein Bachelor-Studium der Mathematik mit Anwendungsfach.
Euch erwartet ein vielseitiges und spannendes, aber auch anspruchsvolles Studium, in dem ihr unter anderem ein sehr viel umfassenderes Bild der Mathematik gewinnen werdet, als es der Schulunterricht vermitteln kann und will.
Ihr werdet Mathematik als eine Wissenschaft kennenlernen, die sich mit präziser Denk- und Arbeitsweise gleichermaßen vielen abstrakten Fragestellungen wie auch angewandten Problemen aus den Naturwissenschaften, der Medizin, den Wirtschaftswissenschaften und anderen Bereichen widmet.

Da sich die Mathematik praktisch in sehr vielen Gebieten wiederfindet, ist die berufliche Situation für Absolventen sehr gut.
Die wahrscheinlich bekanntesten Berufsfelder liegen im Bank- und Versicherungswesen.
Eine ebenfalls sehr attraktive Beschäftigungsmöglichkeit ist sicherlich die des wissenschaftlichen Mitarbeiters an der Universität.
Aber auch in anderen Bereichen gibt es viele Möglichkeiten der Beschäftigung mit einem abgeschlossenen Mathematikstudium.

Dieses Studium besteht aus einer soliden mathematischen Ausbildung, die von Studienbeginn an zu selbstständigem arbeiten anhält.
Die Vermittlung erfolgt in Vorlesungen, an die Übungen geknüpft sind.
Euch wird ans Herz gelegt, die Übungsaufgaben sorgfältig zu bearbeiten. Sie sind meist Voraussetzung für die Teilnahme an den Prüfungen und dienen außerdem der Festigung des Stoffes, was das Lernen für die Klausuren oder mündlichen Prüfungen erleichtert und vor allem verkürzt. 
In den ersten beiden Fachsemestern werden in den Grundmodulen Analysis, Lineare Algebra und Numerik unverzichtbare Grundkenntnisse und Methoden der Mathematik erworben und damit eine solide Grundlage für das gesamte Mathematikstudium gelegt.

Am Anfang des Studiums gibt es neben Vorlesungen und Übungen auch noch Tutorien:
Studenten aus höheren Semestern wiederholen hier den wichtigsten Stoff der Vorlesung und beantworten Fragen dazu.
Diese Veranstaltungen sind zwar freiwillig, aber dennoch ein wichtiger Bestandteil der Wissensvermittlung.
Zusätzlich werden Grundkenntnisse in Informatik (Objektorientierte Programmierung, Datenstrukturen und Effiziente Algorithmen I) und im jeweiligen Anwendungsgebiet erworben. 
Als Anwendungsfächer wählbar sind z.B. Biologie, Chemie, Physik, Informatik, Betriebswirtschaftslehre und Volkswirtschaftslehre.

Aber auch die Verbindung zur Praxis soll nicht zu kurz kommen.
Dazu dient das Praktikum im vierten Semester.
Es sollte in einem wirtschaftlichen Betrieb absolviert werden, kann aber auch in einem anderen Institut der Martin-Luther-Universität stattfinden.
In dem Praktikum sollen typische Studieninhalte des Studienganges zur Anwendung kommen.
Dabei muss es von einer Hochschullehrerin bzw. einem Hochschullehrer des Instituts für Mathematik betreut werden.

Den Regelstudienplan für das Studienprogramm Mathematik (180~Leistungspunkte) findet ihr in Tabelle~\ref{plan-mathe} auf Seite~\pageref{plan-mathe}.

\textbf{Ansprechpartner im Fachschaftsrat:}\\
Marie-Luise Hein\\
\email{marie-luise.hein@fsr-matheinfo.de}


\begin{table}[tbp]
    \begin{small}
    \begin{tabularx}{\textwidth}{|X||c|c|c|c|c|c||r|}
        \hline
        \textbf{Modul}&\multicolumn{6}{l||}{\textbf{Leistungspunkte im Sem.}}&\textbf{LP}\\\hline
        &1&2&3&4&5&6&\\\hline\hline
        Analysis~I+II&9&9&&&&&18\\\hline
        Lineare Algebra~I+II&9&9&&&&&18\\\hline
        ASQ~I+II&5&&&5&&&10\\\hline
        Datenstrukturen und \newline effiziente Algorithmen~I&&5&&&&&5\\\hline
        Objektorientierte Programmierung&5&&&&&&5\\\hline
        Numerik~I+II&&9&9&&&&18\\\hline
        Analysis~III (mit Proseminar)$^{*}$&&&9(+3)$^{*}$&&&&9(+3)$^{*}$\\\hline
        Algebra (mit Proseminar)$^{*}$&&&9(+3)$^{*}$&&&&9(+3)$ ^{*}$\\\hline
        Wahrscheinlichkeitstheorie und Statistik&&&&8&&&8\\\hline
        Maßtheorie&&&&8&&&8\\\hline
        Praktikum (Mathematik)&&&&6&&&6\\\hline
        Fachseminar&&&&&5&&5\\\hline
        Funktionalanalysis&&&&&8&&8\\\hline
        Vertiefungsmodul~I od. II&&&&&15&&15 \\\hline
        Anwendungsfach&&&5&5&5&5&20\\\hline\hline
        \textbf{Bachelor-Arbeit}&&&&&&15&15\\\hline\hline
    \end{tabularx}
    \end{small}
    \caption{\label{plan-mathe}Regelstudienplan für den Bachelor Mathematik (180~LP).\newline
     $^{*}$ Es muss nur eines der beiden Module durch das Proseminar ergänzt werden.   
     Dieser Regelstudienplan ist nur eine Empfehlung.}
\end{table}


\newpage
%%%%%%%%%%%%%%%%%%%%%%%%%% WIRTSCHAFTSMATHE %%%%%%%%%%%%%%%%%%%%%%%%%%%%
\subsection{Wirtschaftsmathematik}
\label{studiengang_wima}

Liebe Studienanfängerinnen und Studienanfänger,\\
herzlich willkommen an der Martin-Luther-Universität und herzlichen Glückwunsch zu eurer Entscheidung für ein Bachelor-Studium der Wirtschaftsmathematik.

Mit diesem Studium soll der Spagat zwischen Mathematik und Wirtschaft gelingen.
Das heißt neben der Beschäftigung mit der Mathematik (vgl. Informationen für Bachelor Mathematik auf Seite~\pageref{studiengang_mathematik}) stehen noch Betriebs- und Volkswirtschaftlehre auf dem Stundenplan.
Hier sind während des gesamten Studiums fünf Fächer zu belegen.
In den ersten beiden Semestern bieten sich z.B. Grundlagen der BWL oder Grundlagen der VWL und Mikroökonomik~I oder Wertschöpfungsmanagement an.
Weitere Möglichkeiten findet ihr in der Prüfungsordnung.

Wirtschaft wirkt am Anfang meist sehr einfach, dadurch nimmt man es gern auf die leichte Schulter.
Die Klausuren, für die man sich rechtzeitig im Löwenportal anmelden muss, sind aber nicht immer ein Kinderspiel, deswegen versucht kontinuierlich den Stoff zu lernen und auch mit anderen (auch VWL- und BWL-Studenten) zusammen zu arbeiten.

Diese Gruppenarbeit ist in der Mathematik das wichtigste, weil der Stoff am Anfang gar nicht einfach, sondern oft unverständlich wirkt.
Hier sind die Übungen das Wichtigste zum Verstehen der mathematischen Inhalte.
In den ersten beiden Semestern hört ihr Analysis I und II, Lineare Algebra I und II, sowie Optimierung.
Es ist in den Mathematik-Modulen immer sehr wichtig, die Übungsaufgaben zu bearbeiten, die euch beim Verstehen des Stoffes helfen und außerdem oft Voraussetzung sind, um an den Prüfungen teilzunehmen.
Neben den beiden Bestandteilen Wirtschaft und Mathematik müsst ihr euch auch noch mit Informatik „herumärgern“.

Den Regelstudienplan für das Studienprogramm Wirtschaftsmathematik (180~Leistungspunkte) findet ihr in Tabelle~\ref{plan-wima} ab Seite~\pageref{plan-wima}.

Wir hoffen, ihr beißt euch durch die ersten Semester, denn das sind die Schwierigsten.
Wir mussten uns auch erst an die neue Denk- und Arbeitsweise gewöhnen.
Also gebt nicht auf!
Wenn ihr Hilfe braucht, könnt ihr euch an uns oder euren Studienberater, Dr.~Rackwitz, wenden.

\textbf{Ansprechpartner im Fachschaftsrat:}\\
Marie-Luise Hein\\
\email{marie-luise.hein@fsr-matheinfo.de}


\begin{table}[tbp]
    \begin{small}
    \begin{tabularx}{\textwidth}{|X||c|c|c|c|c|c||r|}\hline
        \textbf{Modul}&\multicolumn{6}{l||}{\textbf{Leistungspunkte im Sem.}}&\textbf{LP}\\\hline
        &1&2&3&4&5&6&\\\hline\hline
        Analysis~I+II&9&9&&&&&18\\\hline
        Lineare Algebra~I+II&9&9&&&&&18\\\hline
        Objektorientierte Programmierung&5&&&&&&5\\\hline
        Datenstrukturen und \newline
        effiziente Algorithmen&&5&&&&&5\\\hline
        ASQ~I+II&5&&&&&5&10\\\hline
        Optimierung&&10&10&&&&20\\\hline
        Maßtheorie&&&&8&&&8\\\hline
        Analysis III&&&9&&&&9\\\hline
        Wahrscheinlichkeitstheorie und Statistik&&&&8&&&8\\\hline
        Numerische Mathematik \newline für Wirtschaftswissenschaftler&&&&8&&&8\\\hline
        Wahlfachbereich Informatik&&&&&5&&5\\\hline
        Fachseminar&&&&&5&&5\\\hline
        Versicherungsmathematik&&&&&5&&5\\\hline
        Vertiefungsmodul&&&&&5&&5\\\hline
        Wirtschaftswissenschaftsmodul&&&&10&5&10&25\\\hline
        \textbf{Bachelor-Arbeit}&&&&&&15&15\\\hline\hline
    \end{tabularx}
    \end{small}
    \caption{\label{plan-wima}Regelstudienplan für den Bachelor Wirtschaftsmathematik (180~LP). Das Pflichtmodul „Wirtschaftswissenschaftsmodul“ ist ein Platzhalter für eins der im unteren Teil der Tabelle aufgeführten Wirtschaftswissenschaftsmodule. Dieser Regelstudienplan ist nur eine Empfehlung.}
\end{table}




%%%%%%%%%%%%%%%%%%%%%%%%%%%%% MATHE LAG %%%%%%%%%%%%%%%%%%%%%%%%%%%%%

\subsection{Mathematik Lehramt}

Ihr wollt also Lehrer werden?!\\
Gute Wahl! Denn Mathe-Lehrer sind mehr gesucht als eh und je! 
Hier vorab ein paar Infos, worauf ihr euch einlasst: Auch wenn ihr in der Schule nicht zu den Einserkandidaten in Mathematik gehört habt, habt ihr durchaus reale Chancen dieses komplexe Studium erfolgreich zu bestreiten.
Es kommt nur auf eins ein: Durchbeißen.
Und das heißt in den ersten zwei Semestern am Ball bleiben, wenn ihr mit den Bachelor-Mathematikern zusammen „Analysis“ und „Lineare Algebra“ hört.
Und auch wenn es schwer fällt:
Aufstehen, zur Vorlesung gehen, fleißig Übungsaufgaben lösen und auf jeden Fall mit anderen Leuten zusammentun, damit die Verzweiflung mit anderen geteilt werden kann.

Und nein: Ihr seid nicht blöd.
Man kann als normalsterbliches Wesen (fast) nie alle Aufgaben fehlerfrei lösen, aber dafür gibt es auch die 50\% (bei manchen Profs auch 60\%) Hürde ($\rightarrow$  genaueres steht in den Modulbeschreibungen bzw. wird zu Beginn des Semesters bekannt gegeben).
Wenn ihr euch regelmäßig mit den Aufgaben beschäftigt und sie auch wirklich verstanden habt, sollte es auch gut machbar sein, die abschließenden Klausuren zu bestehen.

Lasst euch aber nicht allzu schnell entmutigen.
Rauft euch zusammen.
Löst die Aufgaben zusammen.
Erklärt’s euch gegenseitig.
Schließlich werdet ihr ja Lehrer.

Leider ist das noch nicht alles, denn neben der Mathematik studiert ihr ja noch ein anderes Fach (manche sogar noch zwei) und das erziehungswissenschaftliche Begleitstudium mit den Fächern Pädagogik und Psychologie muss auch noch belegt werden.
Da ist euer Organisationstalent gefragt.
Informiert euch so früh wie möglich über die Veranstaltungszeiten (\link{http://studip.uni-halle.de} oder Aushänge) und sprecht bei Überschneidungen die entsprechenden Dozenten an.
Manchmal kann da noch geschoben werden, machmal aber auch nicht.
Dann versucht vielleicht schon einmal eine andere Veranstaltung zu besuchen, wenn ihr die Voraussetzungen dafür schon habt.

Für Psychologie und Pädagogik ist zusätzlich zu beachten, dass die Einschreibungen momentan online im Stud.IP erfolgen und nur eine begrenzte Anzahl von Plätzen verfügbar ist (Einschreibungen enden dafür schon sehr zeitig.).
Versucht dann lieber Vorlesungen, für die es oft keine Begrenzungen gibt, im 1.~Semester zu besuchen, wenn ihr dafür noch Zeit habt.

Eine Zwischenprüfung im eigentlichen Sinne gibt es nicht mehr.
Ersetzt wird diese durch die jeweiligen Modulleistungen am Ende des Semesters.
Für all diejenigen von euch, die BAföG erhalten ist es also wichtig euer Studium so zu gestalten, dass ihr am Ende des 4.~Semesters auf 105~Leistungspunkte kommt.
Davon dürfen maximal 30 in Form von Prüfungen angemeldet sein, der Rest muss schon bestanden sein.
Dabei ist es zumindest derzeit noch egal, woher diese stammen, sei es euer zweites Fach oder aus Pädagogik und Psychologie.

Bei Problemen und Fragen stehen euch neben den Vertretern des Fachschaftsrates auch sehr gerne der gelegentlich zusammenkommende Lehrerstammtisch und Herr Kortenkamp zur Verfügung.
Er ist für die Mathematik-Didaktik zuständig, womit ihr ab dem dritten Semester zu tun habt.

Einer der größten Unterschiede zwischen dem LAG- und dem LAS-Studium besteht darin, dass die angehenden Sekundarschullehrer erst im dritten Semester die "' Analysis I "' besuchen und nicht wie die angehenden Gymnasiallehrer schon im ersten Semester.
Stattdessen besuchen die LAS-Studenten die Veranstaltung "' Elemente der Mathematik "'.

Eine Übersicht über die Matheveranstaltungen, die während des Studiums absolviert werden, findet ihr in den Tabellen \ref{plan-las} und \ref{plan-lag} auf den folgenden Seiten.

\textbf{Ansprechpartner im Fachschaftsrat:}\\
Mara Jakob\\
\email{mara.jakob@fsr-matheinfo.de}



\label{studiengang_lehramt}

\label{studiengang_lag}

\begin{table}[tbp]
    \begin{footnotesize}
    \begin{tabularx}{\textwidth}{|@{~}X@{~}|@{~}c@{~}|@{~}X@{~}|@{~}X@{~}|@{~}c@{~}|@{~}c@{~}|@{~}c@{~}|@{~}c@{~}|}
        \hline
        Modul & Semes\-ter & Voraus\-setzungen & Modul\-leistung & SWS & \begin{sideways}Be\-notung\end{sideways} & \begin{sideways}Anteil an Abschluss\-note \end{sideways}& \begin{sideways}Leistungspunkte\end{sideways}\\\hline\hline
        Analysis~I  &   1.  &                &   mdl. Prf.   &  6   &   nein    &   -   &   10\\\hline
        Analysis~II &   2.  &  (Analysis~I)   &   mdl. Prf.   &  3   &   ja      &   ja  &   5\\\hline
        Lineare Algebra &  1.-2.   &         &   mdl. Prf.   &  2x6 &   nein    &   -   &   15\\\hline
        Wahr\-schein\-lich\-keits\-theorie und Sta\-tis\-tik&4.&(Analysis~I+II)&mdl. Prf.&6&ja&ja&6\\\hline
        Pro\-seminar&4.&(Analy\-sis~I+II), (Lineare Algebra)&Vortrags\-aus\-arbeitung&2&nein&-&5\\\hline
        Grundl. der numerischen Mathe\-matik&ab 3.&(Analy\-sis~I), (Lineare Algebra)&Klausur&4&ja&ja&5\\\hline
        Algebra&ab 3.&(Analy\-sis~I+II), (Lineare Algebra)&Klausur&6&ja&ja&7\\\hline
        Fach\-seminar&5.&&Vortrags\-aus\-arbeitung&2&nein&-&5\\\hline\hline
        \multicolumn{8}{|c|}{Geometrie*}\\\hline
    %            Geometrie/ Differential\-geometrie&5.od. 7.&Analysis~I+II, Lineare Algebra&Klausur oder mdl. Prf.&6&ja&ja&7\\\hline\hline
        Geometrie&5.od. 7.&(Analysis~I+II), Lineare Algebra&Klausur oder mdl. Prf.&6&ja&ja&7\\\hline
        Differential\-geometrie&5.od. 7.&Analysis~I+II, Lineare Algebra&Klausur oder mdl. Prf.&6&ja&ja&7\\\hline\hline
        \multicolumn{8}{|c|}{Grundlagen*}\\\hline
    %            Geschichte der Mathe\-matik/Grund\-lagen der Mathe\-matik&ab 4.&Analysis~I+II, Lineare Algebra&Beleg\-arbeit&3&ja&ja&5\\\hline\hline
        Geschichte der Mathe\-matik&ab 4.&(Analysis~I), (Lineare Algebra)&Beleg\-arbeit&3&ja&ja&5\\\hline
        Grund\-lagen der Mathe\-matik&ab 4.&(Analysis~I+II), (Lineare Algebra)&Beleg\-arbeit oder Klausur&3&ja&ja&5\\\hline\hline
        \multicolumn{8}{|c|}{Analysis/Numerik*}\\\hline
    %            Funktionen\-theorie/gewöhnl. Differentialgl./Theorie u. Num. gewöhnl. Dgl.&ab 5.&&Klausur&3/3/4&ja&ja&5\\\hline\hline
        Funktionen\-theorie&ab 5.&(Analysis~I+II), (Lineare Algebra)&Klausur od. mdl. Prf.&3&ja&ja&5\\\hline
        Gewöhnl. Differentialgl.&ab 5.&(Analysis~I+II), (Lineare Algebra)&Klausur od. mdl. Prf.&3&ja&ja&5\\\hline
        Theorie u. Num. gewöhnl. Dgl.&ab 5.&&Klausur&4&ja&ja&5\\\hline\hline
        Ver\-tiefungs\-modul (nur wenn Erst\-fach)&ab 3.&(Analysis~I+II), (Lineare Algebra)&Klausur od. mdl. Prf.&3/4&nein&-&5\\\hline
        Mathe\-matik\-didaktik~I&3.-4.&&Beleg\-arbeit oder Klausur&2x2&ja&ja&5\\\hline
        Mathe\-matik\-didaktik~II&4.-5.&(Mathe\-matikdid.~I)&Beleg\-arbeit& 2x2&nein&-&5\\\hline
        Mathe\-matik\-didaktik~III&6.-8.&(Mathe\-matik\-did.~I+II)&mdl. Prf.&2x2&ja&ja&5\\\hline
    \end{tabularx}
    \end{footnotesize}
        \caption{\label{plan-lag}Regelstudienplan für den Studiengang
    Mathematik -- Lehramt an Gymnasien. Bei den mit * gekennzeichneten
    Teilgebieten muss jeweils nur eine Veranstaltung besucht werden. Dies
    ist nur eine Empfehlung.}
\end{table}


\label{studiengang_las}

\begin{table}[tbp]
    \begin{footnotesize}
    \begin{tabularx}{\textwidth}{|@{~}X@{~}|@{~}c@{~}|@{~}X@{~}|@{~}X@{~}|@{~}c@{~}|@{~}c@{~}|@{~}c@{~}|@{~}c@{~}|}
        \hline
        Modul & Semes\-ter & Voraus\-setzungen & Modul\-leistung & SWS & \begin{sideways}Be\-notung\end{sideways} & \begin{sideways}Anteil an Abschluss\-note \end{sideways}& \begin{sideways}Leistungspunkte\end{sideways}\\\hline\hline

        Lineare Algebra &1.-2.&&mdl. Prf.&2x6&nein&-&15\\\hline
        Elemente der Mathematik &1.-2.&&Klausur&4&nein&-&15\\\hline
        Analysis~I &3.&&mdl. Prf.&6&ja&ja&10\\\hline
        Elemente der Kombinatorik und Stochastik &3.&Elemente der Mathematik&Klausur&4&ja&ja&5\\\hline
        Elemente der Geometrie &3. od. 5.&(Elemente der Mathematik)&mdl. Prf.&4&ja&ja&5\\\hline
        Proseminar & ab 3. & & (Analysis~I),(Lineare Algebra) &2&nein&-&5\\\hline
        Algebra & 3. od. 5. &(Analysis~I), )Lineare Algebra) & Klausur & 4 & ja & ja & 5\\\hline\hline

        \multicolumn{8}{|c|}{Mathematik*}\\\hline
        Analysis~II & ab 4. & (Analysis~I) & mdl. Prf. & 4 & ja & ja & 5\\\hline
        Geschichte der Mathematik & ab 4. & (Analysis~I), (Lineare Algebra) & Belegarbeit & 3 & ja & ja & 5\\\hline
        Grundlagen der numerischen Mathematik & ab 5. & (Analysis~I), (Lineare Algebra) & Klausur & 4 & ja & ja & 5\\\hline
        Mathematische Biologie & ab 4. &Analysis~I, Lineare Algebra & Klausur od. mdl. Prf. & 3 & ja & ja & 5\\\hline
        Funktionentheorie & ab 5. & (Analysis~I+II),(Lineare Algebra)& mdl. Prf. & 3 & ja & ja & 5\\\hline
        Geometrie & ab 5. &( Analysis~I), Lineare Algebra & Klausur od. mdl. Prf. & 4 & ja & ja & 5\\\hline
        Diskrete Mathematik & ab 5. & (Analysis~I), Lineare Algebra & Klausur od. mdl. Prf. & 4 & ja & ja & 5\\\hline\hline

        Fachseminar & 5./6. & & Vortrags\-ausarbeitung & 2 & nein & - & 5\\\hline
        Ver\-tiefungs\-modul (nur wenn Erst\-fach)&ab 4.&&&3/4&nein&-&5\\\hline
        Mathe\-matik\-didaktik~I &3.-4.&&Beleg\-arbeit oder Klausur&2x2&ja&ja&5\\\hline
        Mathe\-matik\-didaktik~II &4.-5.&(Mathe\-matikdid.~I)&Beleg\-arbeit&2x2&nein&-&5\\\hline
        Mathe\-matik\-didaktik~III &6.-8.&(Mathe\-matikdid.~I+II)&mdl. Prf.&2x2&ja&ja&5\\\hline
    \end{tabularx}
    \end{footnotesize}
    \caption{\label{plan-las}Regelstudienplan für den Studiengang Mathematik -- Lehramt an Sekundarschulen. Bei den mit * gekennzeichneten Teilgebiet muss jeweils nur eine Veranstaltung besucht werden. Dies ist nur eine Empfehlung.}
\end{table}


