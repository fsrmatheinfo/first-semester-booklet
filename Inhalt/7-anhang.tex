
\section{Anhang}

\subsection{Uni-ABC}

\begin{description}
\item[AAA] Akademisches Auslandsamt, veraltet für International Office.
\item[akademisches Viertel] siehe~c.t.
\item[BAFöG] Kürzel für „Bundesausbildungsförderungsgesetz“.
             Regelt die finanzielle Unterstützung von Studierenden durch den Staat in Form von Stipendium und Darlehen.
             Für immer weniger Studierende Hauptfinanzierungsquelle des Studiums.
\item[c.t.] Abkürzung für „cum tempore“.
            Der zugehörigen Uhrzeit wird das sogenannte „akademische Viertel“ zugerechnet.
            Eine mit 14.00~c.t. ausgezeichnete Veranstaltung beginnt erst 14.15~Uhr.
\item[CP] Abkürzung für „Credit Points“ (Leistungs-Punkte). Wichtige Einheit in den Studienordnungen der Bachelor/Master-Stdiengänge.
          s.~Kapitel~\ref{creditpoints}: Was sind Leistungs-Punkte (LP) bzw. Credit Points (CP)?
\item[Dekan] lat. „Führer von zehn“. Das Oberhaupt einer Fakultät, Sprecher des Fakultätsrates.
\item[dies academicus] lat. „akademischer Tag“. Stellt einen Feiertag an der Uni dar.
                       Während des „dies“, wie er kurz heißt, finden keine Vorlesungen oder Seminare statt.
                       Der dies findet ein- oder zweimal pro Semester zu solchen Veranstaltungen wie Sportfest,
                       Universitätsfest bzw. StudentInnenfestival oder feierlicher Immatrikulation statt.
\item[Fakultät] Gruppierung von verschiedene Instituten. Die Uni besteht aus 10~Fakultäten.
                Das Institut Mathematik befindet sich in der Naturwissenschaftlichen Fakultät~II (NatFak~II)
                und das Institut der Informatik in der Naturwissenschaftlichen Fakultät~III (NatFak~III).
\item[FSR] Abkürzung für „Fachschaftsrat“. Studentische Interessenvertretung am Institut.
           (s.~Kapitel~\ref{vertretung}: Wer vertritt wen?)
\item[HiWi] Kurz für „Hilfswissenschaftler“, also eine wissenschaftliche Hilfskraft.
            HiWis sind in der Regel selbst Studenten, meist höheren Semesters.
\item[(IT)$^2$] Der „Industrietag Informationstechnik“ ist eine Veranstaltung des Universitätszentrum für Informatik (UZI)
                mit der Industrie- und Handelskammer (IHK) und findet einmal pro Semester statt.
\item[ITZ] IT-Servicezentrum.\\
\web{http://itz.uni-halle.de/}
\item[Kanzler] Verwaltungschef der Uni.
\item[Kommilitone] lat. „Waffenbruder“, Mitstudent, Studiengenosse, existiert auch in der weiblichen Form Kommilitonin.
\item[LHG] Abkürzung für „Landeshochschulgesetz“.
           Regelt die Strukturen an den Hochschulen in Sachsen-Anhalt, definiert Inhalt des Studiums, legt fest, wer studieren darf etc.
\item[Mensa] Kurz für „mensa academica“, einem Speisehaus für StudentInnen mit verbilligtem Essen.
\item[MLU] Kürzel für die „Martin-Luther-Universität Halle-Wittenberg“
\item[Rektor] Akademisches Oberhaupt und erster Repräsentant der Universität
\item[Senat] Beschlussfassendes Organ der akademischen Selbstverwaltung.
             Zuständig für die meisten Entscheidungen in Hinblick auf Forschung und Studium.
\item[SoSe] Häufig genutzte alternative Abkürzung für das Sommersemester.
\item[SS] Ist das vom 1.4.--30.09. dauernde Sommersemester.
\item[s.t.] Abkürzung für „sine tempore“, d.h. die zugehörige Uhrzeit ist ohne akademisches Viertel zu verstehen.
\item[StuRa] Abkürzung für „Studierendenrat“. Oberstes Organ der studentischen Selbstverwaltung an ostdeutschen Hochschulen.
             In anderen Universitäten heißt er meist AStA (Allgemeiner Studentenausschuss).\\
             \web{http://www.stura.uni-halle.de/}
\item[SWS] Abkürzung für „Semesterwochenstunden“. Hier werden die Stunden je Woche in einem Semester zusammengezählt.
           Habt ihr eine Vorlesung mit 4+2~SWS, dann heißt das, dass ihr in einer Woche 4~Stunden Vorlesung und 2~Stunden Übung habt.
\item[ULB] Universitäts- und Landesbibliothek\\
           \web{http://bibliothek.uni-halle.de/}
\item[URZ] Universitätsrechenzentrum: Veraltet für ITZ.\
\item[USZ] Universitätssportzentrum\\
           \web{http://www.usz.uni-halle.de/}
\item[UZI] Universitätszentrum für Informatik\\
           \web{http://www.uzi.uni-halle.de/}
\item[WiSe] Häufig genutzte alternative Abkürzung für das Wintersemester.
\item[WS] Ist das vom 1.10.--31.03. dauernde Wintersemester.
\end{description}

\subsection{Nützliche Links}
\begin{description}
\item[\web{http://www.fsr-matheinfo.de/}]
    Das ist die Seite des Fachschaftsrats für Mathematik und Informatik.
    Hier findet ihr viele nützliche Informationen zu eurem Studium und auch immer Hilfe bei Problemen.
\item[\web{http://www.uni-halle.de/}]
    Die Seite der Uni. Von hier aus findet man alle Informationen, die irgendwie mit der Universität zu tun haben.
\item[\web{http://www.informatik.uni-halle.de/}]
    Das ist die Seite des Instituts für Informatik.
\item[\web{http://www.mathematik.uni-halle.de/}]
    Das ist die Seite des Instituts für Mathematik.
\item[\web{http://www.stura.uni-halle.de/}]
    Die Seite des Studierendenrats. Hier findet ihr alle Informationen zu dem wichtigsten Studentengremium und mehr.
\item[\link{https://www.studip.uni-halle.de}]
    Das Vorlesungsverzeichnis der Uni.
    Hier bekommt ihr Informationen zu Veranstaltungen, schreibt euch in diese ein und erhaltet Termine, Dateien sowie Hilfe dazu.
\item[\link{https://studmail.uni-halle.de}]
    Das ist die Seite eurer Uni-Mailadressen.
    Über diesen Account (deren Zugangsdaten euch per Brief zugesandt wurden) erhaltet ihr alle wichtigen Mails.
    Ihr solltet daher hier öfter mal reinschauen oder ihr leitet euch diese Mails auf euer normales E-Mail-Konto weiter.
\item[\link{https://loewenportal.uni-halle.de}]
    Die Selbstbedienungsfunktion der Uni, neuerdings auch „Löwenportal“ genannt.
    Hier könnt ihr euch Immatrikulationsbeschneigungen und andere Belege ausdrucken und für Module anmelden.
%\item[\web{http://www.informatik.uni-halle.de/studenten-infos/pruefungsamt/}]
%    Das Prü\-fungs\-amt unserer Institute. Hier findet ihr Prüfungstermine und Ergebnisse.
\item[\web{http://www.bibliothek.uni-halle.de/}]
    Das ist der Internetauftritt unserer Bibliothek. Hier kann man Literatur suchen, vorbestellen und Ausleihfristen verlängern.
\item[\web{http://www.usz.uni-halle.de/}]
    Hier könnt ihr euch über das Sportangebot an der Uni informieren und euch für kostenpflichtige Sportkurse anmelden.
\item[\web{http://www.studentenwerk-halle.de/}]
    Das Studentenwerk Halle setzt sich für die soziale, wirtschaftliche, kulturelle und gesundheitliche Förderung der Studierenden ein.
    Hier findet ihr Speisepläne, Hilfe zum BAföG und vieles mehr.
\end{description}

\subsection{Wichtige Kontakte}

\subsubsection{Fachschaftsrat Mathematik/Informatik}
Von-Seckendorff-Platz 1, Raum 0.31\\
Sprechzeiten: Montag 16:00--17:00~Uhr\\ und nach Vereinbarung
Tel: (0345) 55 24605\\
\email{fachschaft@mathinf.uni-halle.de}

\subsubsection{Prüfungsamt der NatFak III (inkl. Informatik)}
Erika Brandt\\
Von-Seckendorff-Platz 4\\
Raum: 4.25\\
Sprechzeiten: Dienstag und Donnerstag, 9.00~Uhr bis 11.00~Uhr und
13.00~Uhr bis 15.00~Uhr\\
Tel: (0345) 55 24711\\
\email{pruefungsamt@informatik.uni-halle.de}\\
\web{http://www.natfak3.uni-halle.de/pruefungsamt\_natfak\_3/}

\subsubsection{Prüfungsamt der NatFak II (inkl. Mathematik)}
Kerstin Baranowski\\
Von-Danckelmann-Platz 3
Sprechzeiten: Dienstag und Donnerstag 10.30~Uhr bis 11.30~Uhr, sowie
Dienstag, Mittwoch und Donnerstag 13.30~Uhr bis 16.00 Uhr\\
Tel: (0345) 55 25600\\
\email{kerstin.baranowski@natfak2.uni-halle.de}\\
\web{http://www.natfak2.uni-halle.de/studium/}

\subsubsection{Studienberater Mathematik und Wirtschaftsmathematik}
Dr.\ Hans-Georg Rackwitz\\
Theodor-Lieser-Str.\ 5, Raum 127\\
Tel: (0345) 55 24608\\
\email{hans-georg.rackwitz@mathematik.uni-halle.de}\\

\subsection{Studienberater Informatik und Bioinformatik}
Apl.\ Prof.\ Dr.\ Klaus Reinhardt\\
Von-Seckendorff-Platz 1, Raum 2.10\\
Tel: (0345) 55 24770\\
\email{klaus.reinhardt@informatik.uni-halle.de}

\subsubsection{Studierendenrat}
Universitätsplatz 7\\
Tel: (0345) 55 21411\\
\email{stura@uni-halle.de} \\
\web{http://www.stura.uni-halle.de/studierendenrat/kontakt/}

\subsubsection{BAföG-Amt}
Wolfgang-Langenbeck-Str. 3\\
Tel: (0345) 68 47113\\
\email{bafoeg@studentenwerk-halle.de}\\
\web{http://www.studentenwerk-halle.de/bafoeg/bafoeg/}

\subsubsection{Studierenden-Service-Center}
Universitätsplatz 11 (im Löwengebäude)\\
Hier findet ihr das Immatrikulationsamt und die allgemeine
Studienberatung.\\
Öffnungszeiten: Montag bis Donnerstag 10.00~Uhr bis 16.00~Uhr, sowie \\
\web{http://www.uni-halle.de/ssc/}

\subsubsection{International Office}
Universitätsring 19/20 \\
First Floor \\
06108 Halle (Saale) \\
Tel: +49 345 55 21589\\
\email{info@international.uni-halle.de}\\
\web{http://www.international.uni-halle.de/international\_office/}

\subsubsection{IAESTE LC Halle}
Studierendenaustauschorganisation im DAAD mit vielen IT-Plätzen \\
Friedemann-Bach-Platz 6 \\
06108 Halle (Saale) \\
\web{http://www.iaeste-halle.de} \\
Kurzfristige Angebote: \web{https://www.iaeste.de/lateoffers/}

