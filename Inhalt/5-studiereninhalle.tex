
\section{Studieren in Halle}

\subsection{Was kann mein Studentenausweis?}
Die Uni Service Card (USC) besitzt einen kontaktlosen, äußerlich nicht sichtbaren Chip und einen wiederbeschreibbaren Streifen für visuell lesbare Aufdrucke.
Damit kann sie mehrere Funktionen erfüllen. Sie dient...

\begin{itemize}
 \item als Studierendenausweis.
       Dazu muss sie den Gültigkeitsaufdruck für das jeweilige Semester tragen.
 \item als Mensakarte.
       Alle Hinweise dazu findet ihr auf dem Faltblatt des Studentenwerks, das ihr zusammen mit der USC erhalten haben solltet
       oder unter \link{http://www.studentenwerk-halle.de/hochschulgastronomie/}.
 \item als Kopierkarte.
 \item als Bibliothekskarte.
       Dazu muss man sich jedoch einmalig in der Hauptbibliothek anmelden.
       Dann kann sie anstelle der bisherigen Bibliothekskarte benutzt werden.
       Wenn ihr eine alte Bibliothekskarte habt, werdet ihr sie bei eurem nächsten Bibliotheksbesuch abgeben müssen.
 \item als MDV Vollticket.
       Im Studierendenausweis ist seit dem Wintersemester 2014/2015 eine Fahrkarte für Bus und Bahn im Raum Halle enthalten. Nähere Informationen findet ihr unter: \link{http://semesterticket-halle.de/}.
 \item als Mitgliedskarte für die Studierendenschaft.
       Das ist für Wahlen von Bedeutung.
       Auch dafür erhält die USC einen Aufdruck.
 \item als Zugangsmedium für einige Universitätsgebäude.
\end{itemize}

Für alle Bezahlfunktionen ist in der USC ein Chip integriert, für das ein Guthaben geführt wird.
Dieses könnt ihr an den Aufwertern des Studentenwerkes aufladen.

\clearpage %schick machen

\paragraph{Was es bei der USC zu beachten gibt}

Zu Beginn jedes Semesters muss die USC validiert werden. Das könnt ihr nach der Rückmeldung fürs neue Semester jederzeit tun.
Dabei wird der Chip für das neue Semester freigeschaltet und die USC erhält die bereits erwähnten Aufdrucke.
Diese Validierung müsst ihr an einer der Validierungsstationen selbst vornehmen.
In unserer Universität gibt es vier Validierungsstationen, die an folgenden Orten stehen:

\begin{itemize}
    \item im Löwengebäude, Raum~1 (Inforaum)
    \item im Juridicum, Eingangshalle
    \item im Universitätsrechenzentrum, Kurt-Mothes-Str.~1, 1.~Etage
    \item in der Heidemensa, Theodor-Lieser-Str.~5, im Erdgeschoss gegenüber den Aufladestationen
\end{itemize}

Wichtig ist es, dass ihr eure USC pfleglich behandelt, um sie funktionsfähig zu erhalten.
Ihr dürft sie nicht knicken, nicht extremen Temperaturen oder starkem Druck aussetzen und nicht in stärkere Magnetfelder bringen.

Wenn ihr Probleme mit eurer USC habt, dann wendet euch an das Immatrikulationsamt.
Die erste USC erhaltet ihr gratis.
Wenn ihr sie funktionsunfähig macht oder sie euch abhanden kommt, erhaltet ihr gegen eine Gebühr von \EUR{10,30} eine neue USC.

\paragraph{Studentenkarte mit erweiterten Funktionen}

Seit Oktober 2010 führt das Institut für Informatik einen Testbetrieb mit neuen Studentenkarten durch.
Diese neuen Karten verfügen über einen weiteren Chip mit dem man unter anderem Authentifizierungsdienste nutzen kann.
Das heißt für den Login, in den Computer-Pools der Informatik benötigt man nur noch die Karte mit PIN und kein schwierig zu merkendes Passwort mehr.
Auch für das Stud.IP existiert diese Art der Anmeldung schon.
Im Löwenportal soll man künftig mit der Karte auch Prüfungen anmelden können ohne weitere Eingabe von TANs.
Weiterhin kann man mit den Karten kryptographische Funktionen nutzen.
So kann man z.B. verschlüsselte Emails empfangen oder Emails signieren -- also unterschreiben.

Wer eine solche Karte haben möchte (dies gilt für alle Studenten), kann diese bei Dr. Sandro Wefel im Raum 2.16 beantragen.
Weitere Informationen zum neuen Studentenausweis stellt das Universitätsrechenzentrum (URZ) bereit:\\[1.0em]
    \link{http://www.urz.uni-halle.de/zertifizierungsbetrieb/}


\subsection{Was macht das Studentenwerk?}

Als sozialer Service für Studierende ist das Studentenwerk Halle für folgende Hochschulen verantwortlich:

\begin{itemize}
    \item Martin-Luther-Universität Halle-Wittenberg
    \item Burg Giebichenstein Hochschule für Kunst und Design
    \item Hochschule Anhalt~(FH) mit Standorten in Köthen, Bernburg und Dessau
    \item Hochschule Merseburg~(FH)
\end{itemize}

Der Verwaltungsrat des Studentenwerkes beeinflusst und kontrolliert die wirtschaftliche Entwicklung und das Handeln der Geschäftsführung.
Das Gremium setzt sich aus studentischen und nichtstudentischen Vertretern zusammen, die durch die Senate der Hochschulen alle zwei Jahre legitimiert werden.

\paragraph{BAföG}

Das Studentenwerk Halle ist als Amt für Ausbildungsförderung mit der Durchführung des Bundesausbildungsförderungsgesetzes (BAföG) beauftragt.
Um innerhalb der gesetzlichen Vorgaben umfassender und persönlicher informieren zu können, stehen dir die Mitarbeiter des Amtes für Ausbildungsförderung zu individuellen Beratungsgesprächen zur Verfügung.
Ein kurzer Gang zum BAFöG-Amt spart euch \;u.U. sehr viel Zeit beim Ausfüllen der Formulare.

\paragraph{Verpflegung}
Ein vollwertiges Essen: schmackhaft, appetitlich, preiswert -- das Studentenwerk bietet in den Verpflegungseinrichtungen eine Auswahl unterschiedlicher Speisekomponenten.
Du kannst dir dein Komplettmenü nach deinen Wünschen selbst gestalten, das du mit deiner USC oder (nicht mehr in allen Mensen) mit Bargeld bezahlen kannst.

\paragraph{Wohnen}
Dem bundesweiten Trend studentischer Wohnformen folgend, kann dir das Studentenwerk vom Einzelzimmer bis zum Appartement Wohnraum anbieten und
versucht, deine veränderten Wohnbedürfnisse zu realisieren und je nach Sanierungsgrad die Mieten sozialverträglich zu gestalten.

\paragraph{Beratung}
Das Studentenwerk berät dich.
Es versucht, mit dir eine Lösung aus einer vermeintlichen Auswegslosigkeit zu finden und hilft dir bei der Vermittlung zu den speziellen Fachkräften.

\paragraph{Kultur}
Durch die Semesterbeiträge der Studierenden ist das Studentenwerk in der Lage, für kulturelle Projekte oder Initiativen Fördermittel auszugeben.
Damit wird die Hochschullandschaft um ein vielfältiges buntes Angebot an Kleinkunst reicher und kann sich als Ergänzung zu den Angeboten der Städte durchaus messen.

Das Studentenwerk organisiert im Sommersemester eine kulturelle Veranstaltung -- das Unisportfest, welches neben sportlichem Können auch Gaudi -- es gibt ein Trabi-Wettschieben -- zum Gegenstand des Spektakels hat und in Zusammenarbeit mit dem Universitätssportzentrum den Abschluss des Universitätssportfestes bildet.
Der anschließende Sportlerball erfreut sich großer Beliebtheit.

Der studentische Foto-Club „Conspectus“ steht für Interessierte jederzeit zur Ver\-fügung, wenn es um fachliche Anleitung beim Fotografieren geht.
Die Mensa Harz bietet ideale Möglichkeiten, Ausstellungen zu präsentieren und vielfältige Veranstaltungen durchzuführen.
Verschiedene interessante Projekte konnten bisher mit Fördergeldern bedacht werden:

\begin{itemize}
    \item Kino-Clubs an den verschiedenen Standorten
    \item Faschingsveranstaltungen
    \item Filmprojekte
    \item Musikveranstaltungen in den Studentenclubs
    \item Vorträge
    \item Ausstellungen
    \item Foto-Clubs
    \item Projekte ausländischer Studierender
\end{itemize}

\paragraph{Kinderbetreuung}
Zur Betreuung der Kinder studentischer Eltern unterhält das Studentenwerk an den Standorten Halle und Köthen jeweils eine Kindertageseinrichtung, in denen Studentenkinder bevorzugt aufgenommen werden und die speziell auf die Bedürfnisse studentischer Eltern eingestellt sind.
Am Standort Halle berät das Studentenwerk insbesondere auch die Beschäftigten der Martin-Luther-Universität Halle-Wittenberg bei der Antragstellung auf einen Platz in der Kindertageseinrichtung „Weinberg“ und informiert über die jeweilige Unterbringungssituation.
Der Elternbeitrag in beiden Kindertageseinrichtungen wird nach Festbeträgen ermittelt.
Es werden Kinder im Alter von 0 bis zu 6~Jahren betreut und diese können auf der Basis des pädagogischen Konzeptes der Einrichtung verschiedene Angebote wahrnehmen.

Weitere Informationen kann man unter
\web{http://www.studentenwerk-halle.de/} nachlesen.

\subsection{Stipendien}

Für viele Studenten sind Stipendien attraktiv.
Sie dienen als Fi\-nan\-zier\-ungs\-mö\-glich\-keit des Studiums und helfen vielen Studenten über die Runden, bieten aber mehr als „nur“ Geld.
Die Studienstiftungen, die Stipendien anbieten, veranstalten Seminare und Workshops zur besseren Qualifikation und begleiten Stipendiaten im Wust des bürokratischen Alltags.

Das Bundesministerium für Bildung und Forschung (BMBF) fördert eine Vielzahl solcher Stiftungen, die i.A. unabhängig von anderen Organisationen handeln, oft aber Kirchen, Gewerkschaften oder Parteien nahe stehen.
Somit zahlt sich dann auch ehrenamtliches Engagement aus.
Oft werden engagierte Leute bevorzugt gefördert.
Natürlich ist es auch hilfreich erfolgreiche Leistungen während des Studiums vorweisen zu können.

Falls dich Stipendien interessieren, kannst du gern beim Fachschaftsrat anfragen.
Im Fachschaftssraum liegen in der Regel viele Flyer von Stiftungen aus, in denen genau beschrieben ist, was man für ein Stipendium machen muss, also welche Pflichten und Rechte mit einem Stipendium einhergehen.
Es lohnt sich ein Blick auf die Website des BMBF.

\link{http://www.bmbf.de/de/11869.php}\\
\link{http://www.bmbf.de/de/294.php}\\
\link{http://www.deutschland-stipendium.de/}\\
\link{http://www.uni-halle.de/deutschland-stipendium/}

\subsection{Studentenalltag}

Zuerst soll hier gesagt sein, dass Uni nichts mit der Schule gemeinsam hat.
Dies wird nur oberflächlich durch die Tatsache negiert, dass es an unserem Fachbereich „Hausaufgaben“ gibt.
In der Schule konnte man diese ignorieren, ohne dass es zu großen Problemen kam.
Im Studium ist das anders, hier sind Hausaufgaben meist Grundlage für die folgenden Prüfungen und gleichzeitig werden sie (meist jede Woche) korrigiert und bewertet.
Solltest du meinen, dass du ohne Hausaufgaben zu machen eine Prüfung bestehen wirst, so wirst du bald merken, dass du nicht einmal zur Prüfung zugelassen wirst.
Dafür musst du in der Regel mindestens 50\% der Punkte in den Übungen haben.
Die genauen Bedingungen legt aber jeder Professor für seine Veranstaltung fest, welche in der ersten Vorlesung bekanntgegeben werden müssen -- also nicht verschlafen!

Weiter kann man nur sagen: Sei fleißig, bilde Übungsgruppen (nicht zum Abschreiben!) und wende dich bei Fragen an uns, denn nichts ist schlimmer als mitzuspielen, ohne die Spielregeln zu kennen!

\subsection{Was mache ich in meiner Freizeit?}

Wenn jemand den ganzen Tag fleißig studiert hat, darf er sich hin und wieder auch anderen Dingen widmen.
Viele Infos zu Halle und zu kulturellen Einrichtungen findet ihr meist gut sortiert unter:

\web{http://www.halle.de/}\\
\web{http://www.kulturfalter.de/}

\paragraph{Fachschaftsratsveranstaltungen}
Wir -- der Fachschaftsrat -- organisieren einige Feiern bzw. Feste für unsere Studenten und wir freuen uns immer sehr auf euch.
Einige Termine, die ihr nicht verpassen solltet:

\begin{itemize}
    \item Das Erstsemestergrillen ist speziell für unsere „Erstis“; zum Kennenlernen und um uns auszufragen.
          So werdet ihr gut informiert und mit viel Unterstützung von euren Kommilitoninnen und Kommilitonen euer Studium möglichst gut meistern!
    \item Unsere Weihnachtsfeier findet immer kurz vor den Weihnachtsferien -- meist in unserer Caféteria statt.
          Ein Kommen lohnt sich nicht nur, um einen schönen Abend mit euren Kommilitoninnen und Kommilitonen zu verbringen und neue kennenzulernen,
          sondern auch, weil wir für euch eine tolle Karaoke-Anlage aufbauen.
          Außerdem sind Glühwein und Getränke günstig, Plätzchen und Naschkram sogar umsonst.
    \item Die NatFusion ist eine gemeinsam mit den Physikern im Frühling ausgerichtete Party,
          bei der der Zusammenhalt zwischen unseren beiden Fakultäten verstärkt werden soll.
          Es gibt Musik, Ratespiele und andere Überraschungen.
    \item Unser Sommerfest, welches meist Mitte Juni stattfindet, bezeichnen wir auch oft als „Sportfest“,
          denn wir richten kleine Turniere in Volley- und Fußball aus.
          Bei diesen dürfen nur Studierende aus den Instituten für Informatik und Mathematik teilnehmen, wir sind also „unter uns“
          und ohne blöde Übersportler, die einem den Spielspass versauen könnten.\\
          Wo wir grade bei Spielspass sind: Traditionell gibt es noch den Festplattenweitwurf,
          bei dem ihr -- wie ihr schon ahnt -- eine Festplatte so weit wie möglich werfen sollt.
          Das macht -- wir sprechen aus eigener Erfahrung -- unglaublich viel Spass!
\end{itemize}

Des weiteren organisieren wir dann und wann ein paar Spielabende und Grillfeiern, bei denen es uns einfach um ein gemütliches Zusammensein geht oder wir auf etwas hinweisen möchten, wie zum Beispiel mit unserem traditionellen Wahlgrillen.
Wir bemühen uns immer auch fleischfreies Grillgut anzubieten, damit auch die Vegetarier und Feinschmecker unter euch auf ihre Kosten kommen.

Wir hängen in jedem Fall immer Plakate aus; leider haben wir oft das Problem, dass diese trotz Allem oft übersehen werden und deswegen bitten wir euch:
Lauft mit offenen Augen durch die Gebäude, oft gibt es interessante kleine Aushänge oder eben Plakate von uns.

\paragraph{Theater}
Als größte Stadt Sachsen-Anhalts hat Halle natürlich einiges zu bieten.
Da wäre z.B. das Neue Theater (Große Ulrichstraße), das Thalia-Theater (Puschkinstraße bzw. Thaliapassage), das Opernhaus (Opernplatz), das Puppentheater und jede Menge kleinerer und größerer Theatergruppen (\;u.a. die Kiebitzensteiner als bekannte Kabarettgruppe).
Es lohnt sich auf jeden Fall mal bei ihnen reinzuschauen.
Manchmal organisieren wir auch gemeinsame Theater- bzw. Opernbesuche und verlosen Freikarten!

\paragraph{Kino}
Wer sich lieber den bewegten Bildern widmen möchte, kann dies entweder in großen Multiplex-Kinos (CinemaxX im Charlottencenter) und dem neuen 3D-Kino (thelightcinema im Neustadt Centrum) oder in vielen kleinen Programm-Kinos tun.

Multiplex-Kino: \web{http://www.cinemaxx.de}\\
3D-Kino: \web{http://www.lightcinemas.de/}\\
Programm-Kino Zazie: \web{http://www.kino-zazie.de}\\
Programm-Kino Lux: \web{http://www.luxkino.de}\\
Unikino: \web{http://www.unikino.uni-halle.de}


\paragraph{Kneipen und Diskos}
Wer Zerstreuung in geistigen Getränken sucht, kann dies ausgiebig auf der halleschen Kneipenmeile tun.
Sie erstreckt sich von der Sternstraße über den Marktplatz hinweg bis zur Kleinen Ulrichstraße.
Aber auch abseits der Kneipenmeile ist noch viel los.
Vor allem am Universitätsring entwickelte sich in den letzten Jahren eine rege Kneipenszene.
Wer gerne Cocktails trinkt, sollte auf jeden Fall mal im Enchilada vorbei schauen.
Wer allerdings gern tanzen geht, findet im Turm, im Flower-Power und in der Tanzbar „Palette“ gute Anlaufstellen.

Wer zwar gern Musik hört, aber nicht so gern in Diskos, sondern lieber auf Konzerte geht, dem seien außerdem noch die Rockstation, das (neue) Hühnermanhattan, das Objekt 5, das Reil 78 und das VL ans Herz gelegt.
In letzteren beiden genannten findet man auch andere Beschäftigungs- und Engagementmöglichkeiten, es lohnt sich also die Webseiten mal abzuchecken.

Partytermine: \web{http://partytermine-halle.de/} \\
Enchilada: \web{http://www.enchilada.de/} \\
Turm Halle: \web{http://www.turm-halle.de/} \\
Flower-Power: \web{http://www.urania70.com/} \\
Tanzbar „Palette“: \web{http://www.tanzbar-palette.de/} \\
Rockstation: \web{http://www.rockstation-halle.com/} \\
Halle-Nightlife: \web{http://halle-nightlife.de}\\
Hühnermanhattan: Schaut in die StudiVZ-Gruppe „hühnermanhattenlover“ \\
Objekt 5: \web{http://www.objekt5.de/} \\
Reil 78: \web{http://www.reil78.de/} \\
VL: \web{http://www.ludwigstrasse37.de/}\\
Charles Bronson: \web{http://www.wearecharlesbronson.de/}\\
Chaise: \web{http://www.chaise.de/}

\paragraph{Bauernclub}
In Halle gibt einen Studentenclub namens Bauernclub.
Dieser wird von Studenten für Studenten geführt und zeichnet sich deshalb durch eine entsprechende lockere Atmosphäre und sehr niedrige Preise aus.
Auch bietet der Bauernclub die Möglichkeit, seine Räumlichkeiten für eigene Feiern zu mieten.
Am besten man besucht ihn und macht sich selbst ein Bild.

\web{http://www.bauernclub.de}
        
\subsection{hastuzeit}
Die hastuzeit ist die hallesche Studierendenzeitschrift und wird von der Studierendenschaft der MLU herausgegeben.

In ihr könnt ihr von aktuellen Ereignissen der Hochschulpolitik und amüsanten Geschichten aus dem Studentenalltag lesen.
Auch Erfahrungsberichte von Auslandssemestern oder die Kulturszene Halles sind beliebte Themen.

Den Druck der Hefte finanziert der StuRa, sodass sie für euch kostenlos sind.
In der Regel erscheint die hastuzeit viermal im Semester und wird uniweit, zum Teil auch in den Studentenwohnheimen verteilt.
Im Institut für Informatik findet ihr sie meist in der 3. Etage oder im Fachschaftsraum (0.31).
Ihr könnt euch aber auch über ihre Homepage und per Twitter auf dem Laufenden halten.

Die fleißige Redaktion freut sich nicht nur über Leserbriefe und Anregungen, sondern ihr könnt auch eigene Beiträge verfassen oder das Team als neues Mitglied verstärken.

\web{http://www.hastuzeit.de/}\\
\web{http://twitter.com/hastuzeit}\\
\email{redaktion@hastuzeit.de}



