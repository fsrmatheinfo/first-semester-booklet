
\section[Was sind Leistungs-Punkte?]{Was sind Leistungs-Punkte~(LP) bzw. Credit Points~(CP)?}
\label{creditpoints}

Das Bachelor-Studium umfasst insgesamt 180~Leistungspunkte~(LP).
Dies entspricht sechs Semestern Regelstudienzeit.
Ein Leistungspunkt entspricht 30~Stunden studentischer Arbeitszeit, die sich aus dem Besuch von Vorlesungen und Übungen, sowie dem Ausarbeiten von Hausaufgaben zusammensetzt.

Das Master-Studium weist insgesamt 120~LP auf, also vier Semester Regelstudienzeit.
Der weiterbildende, berufsbezogene Master kann auch nur 60~LP, also zwei Semester, umfassen.

Ein Bachelor- bzw. ein Master-Studiengang besteht aus einem oder zwei Studienprogrammen.
Ein Studienprogramm regelt das Studium einer wissenschaftlichen Disziplin und entspricht einem Studienfach.
Es kann aber auch interdisziplinär angelegt sein.

Drei Varianten des Bachelor-Studiums sind möglich:
\begin{enumerate}
    \item Studiengänge mit einem Studienprogramm (180~LP)
    \item Studiengänge mit zwei gleichgewichtigen Studienprogrammen\\
          (90~LP pro Fach)
    \item Studiengänge mit einem großen und einem kleinen Studienprogramm\\
          (120~LP und 60~LP)
\end{enumerate}

Die Lehramtsstudiengänge sind 2007 modularisiert worden, das bedeutet, dass in diesem Studiengang ebenfalls Leistungspunkte erworben werden müssen.
Ein Lehramtsstudent für Sekundarschulen muss 80~LP in seinem Erstfach und 75~LP in seinem Zweitfach erwerben, sowie weitere 85~LP unter Anderen mit Bildungswissenschaften und Praktika.
Ein Lehramtsstudent für Gymansien hingegen muss 95~LP in seinem Erstfach und 90~LP in seinem Zweitfach erwerben und ebenfalls die oben genannten 85~LP.
Genaueres kann man auf folgender Webseite einsehen:\\[1.0em]
\web{http://www.zlb.uni-halle.de/studium/}
