\section{Begrüßungen}

\subsection{Vorsitzende des Fachschaftsrats Mathematik/Informatik}

Hallo liebe Erstsemester und Kommilitonen,

im Namen des Fachschaftsrates Mathematik/Informatik begrüße ich euch allerherzlichst an der Martin-Luther-Universität. \\
Nun beginnt für euch ein neuer Lebensabschnitt und sicherlich seid ihr gespannt, was euch in den nächsten Wochen und Monaten an der Universität erwarten wird. Mit dem Beginn der Vorlesungen und den dazugehörigen Übungen umrahmt vom studentischen Leben neben der Uni erwartet euch ein neuer Tagesablauf. 
Seid ab den ersten Wochen fleißig und teilt euch eure Zeit gut ein; auf diesen Weg werdet ihr schnell euren ganz persönlichen Wochenrhythmus finden.
Habt Mut, eure Kommilitonen anzusprechen, gemeinsam lassen sich die Herausforderungen des Studiums leichter bewältigen. Auf diesem Wege werdet ihr viele neue Leute kennenlernen und aus den anfänglichen Lerngruppen werden im Laufe der Zeit Freundschaften. Traut euch auch, Dozenten und Übungsleiter anzusprechen. Neben erhofften Problemlösungen entsteht so schnell ein nettes Gespräch, welches euch über den Rand des Vorlesungsstoffes hinaus blicken lässt.

Darüber hinaus stehen wir euch als Fachschaftsrat jederzeit bei Fragen und Problemen zu Seite. Wir hoffen, euch mit diesem kleinen Heft eine hilfreiche Übersicht zur Bewältigung der anstehenden Herausforderungen bereitzustellen.
Aber auch der Fachschaftsrat braucht eure Hilfe. Wir sind -- wie ihr auch -- Studenten und freuen uns sehr über eure Unterstützung. Falls ihr also dem Fachschaftsrat beitreten wollt, um die Interessen eurer Kommilitonen zu vertreten und den Studenten bei Fragen und Problemen zur Seite zu stehen, dann stellt euch im nächsten Frühjahr doch zur Wahl und werdet selbst ein Mitglied des Fachschaftsrates. Alle Informationen hierzu und weitere Möglichkeiten sich zu engagieren, findet ihr auf unserer Homepage.
Ich wünsche euch viel Spaß und Erfolg im Studium. Auf dass ihr die kleinen und größeren Hürden des Studiums meistert und am Ende einen der begehrten Abschlüsse in den Händen haltet. Habt ihr dieses erreicht, so stehen euch alle Türen für das künftige Berufsleben offen. Bis dahin versuchen wir euch zu unterstützen, wo wir nur können.

Zum Schluss möchte ich euch noch unsere Homepage ans Herz legen, auf der ihr weitere Informationen findet und den Terminkalender mit den nächsten anstehenden Fachschaftsveranstaltungen: \texttt{http://fsr-matheinfo.de/}

Delia Elmrich\\
Vorsitzende des Fachschaftsrates Mathematik/Informatik


\subsection{Dekan und Studiendekan der Naturwissenschaftlichen Fakultät II}

Liebe Studienanfängerinnen und Studienanfänger der Naturwissenschaftlichen
Fakultät II,

herzlich willkommen an der Martin-Luther-Universität und an der Naturwissenschaftlichen Fakultät II mit den Instituten für Chemie, für Physik und für Mathematik. Wir freuen uns, dass Sie sich für einen der mehr als 20 Studiengänge unserer Fakultät entschieden haben, und hoffen, dass Sie sich schnell bei uns einleben werden.

Der moderne Weinberg-Campus begrüßt Sie mit vielen in den vergangenen Jahren komplett sanierten oder neu erbauten Gebäuden und kurzen Wegen zwischen den Universitätsinstituten, Forschungseinrichtungen und Bibliotheken. Nach Abschluss der Sanierungsarbeiten wird uns auch wieder direkt am Von-Danckelmann-/Von-Seckendorff-Platz eine Mensa des Studentenwerks zur Verfügung stehen.

Die im Jahr 2006 aus den Fachbereichen Chemie und Physik hervorgegangene Naturwissenschaftlichen Fakultät II umfasst heute neben diesen beiden klassischen naturwissenschaftlichen Fachrichtungen auch das Institut für Mathematik und deckt damit ein breites fachliches Spektrum der experimentellen und theoretischen Forschung und Lehre ab. Alle zwölf Bachelor- und Masterstudiengänge der Fakultät wurden Mitte diesen Jahres erneut akkreditiert und tragen damit auch künftig dieses Gütesiegel der renommierten Akkreditierungsagentur ACQUIN.
 
Wir wünschen Ihnen viel Spaß an unserer Fakultät, viel Erfolg im Studium und eine gute Zeit an unserer altehrwürdigen Martin-Luther-Universität.
 
Prof. Dr. Wolfgang H. Binder (Dekan)\\
Prof. Dr. Martin Arnold (Studiendekan)\\
Naturwissenschaftliche Fakultät II (Chemie, Physik, Mathematik)


\subsection{Dekan und Studiendekan der Naturwissenschaftlichen Fakultät III}

Liebe Studentinnen, liebe Studenten,

ein herzliches Willkommen an der Naturwissenschaftlichen Fakultät III.
Wir freuen uns auf die Zusammenarbeit mit Ihnen.
Sie haben sich mit der Informatik oder Bioinformatik für ein zukunftsträchtiges Fach entschieden, mit dem Sie nach einem erfolgreichen Abschluss exzellente Berufsaussichten haben.
Nahezu alle Branchen sind von der IT durchdrungen und benötigen deshalb Absolventen von Informatikstudiengängen.

Bevor Sie soweit sind liegt jedoch das Studium vor Ihnen.
Mit dem Studium beginnt für Sie ein neuer Lebensabschnitt.
Sie werden viele neue Eindrücke gewinnen, neue Freunde im Kreise Ihrer Kommilitonen (hoffentlich nicht nur in Ihrem eigenen Fach) werden hinzukommen, und nicht zuletzt werden Sie mit einer für Sie neuen Art des Lernens im universitären Umfeld vertraut.
Gerade in den Informatikfächern ist es wichtig, sich selbst den notwendigen Stoff zu erarbeiten.
Vorlesungen, Übungen, Seminare und Praktika bieten den Rahmen dafür.
Scheuen Sie sich nicht, in den Lehrveranstaltungen Fragen an die Dozentinnen und Dozenten zu stellen. 

Bei aller Arbeit für das Studium, sollte auch das soziale Studentenleben nicht zu kurz kommen. Gehen Sie auf Feste, werden Sie in Sport oder Kultur aktiv, arbeiten Sie in der studentischen Interessenvertretung mit -- die Universität und die Vereine in Halle und Umgebung bieten vielfältige Möglichkeiten.
Die mit solchen Erfahrungen erworbenen sogenannten Soft-Skills sind sowohl für Ihr Studium als auch für das spätere Berufsleben mehr als nützlich.

Wir wünschen Ihnen einen guten Start ins Studium sowie viel Spass und viel Erfolg beim Studieren

Prof. Dr. Olaf Christen (Dekan)\\
Prof. Dr. Wolf Zimmermann (Studiendekan)\\
Naturwissenschaftliche Fakultät III (Agrar- und Ernährungswissenschaften, \newline Geowissenschaften, Informatik)


\subsection{Institutsdirektor des Instituts für Mathematik}

Liebe Studienanfängerinnen, liebe Studienanfänger,

herzlich willkommen an der Martin-Luther-Universität und herzlichen Glückwunsch zu Ihrer Entscheidung für ein mathematisches Studienfach, sei es einer unserer Bachelor-Studiengänge „Mathematik“ oder „Wirtschaftsmathematik“ oder einer der Lehramtstudiengänge im Fach Mathematik.

Sie beginnen ein vielseitiges und spannendes, aber auch anspruchsvolles Studium, in dem Sie u.a. ein sehr viel umfassenderes Bild der Mathematik gewinnen werden, als es der Schulunterricht vermitteln kann.Sie werden die Mathematik als eine Wissenschaft kennen lernen, die sich mit präziser Denk- und Arbeitsweise gleichermaßen vielen abstrakten Fragestellungen wie auch angewandten Problemen aus den Naturwissenschaften, der Medizin, den Wirtschaftswissenschaften und anderen Bereichen widmet.Viele von uns Mathematikerinnen und Mathematikern begeistern sich sowohl für die abstrakte als auch für die angewandte Seite der Mathematik.Sie sind herzlich eingeladen, sich hiervon „anstecken“ zu lassen.

Unter Ihnen sind nicht wenige, die mit dem vor Ihnen liegenden Studium in einer wirtschaftlich schwierigen Zeit den Grundstein für ein späteres erfolgreiches Berufsleben legen wollen.Sie haben eine gute Wahl getroffen, denn den Absolventinnen und Absolventen mathematischer Studiengänge bieten sich viele Möglichkeiten in Wissenschaft und Forschung, in der freien Wirtschaft und bei Banken und Versicherungen.Und auch für künftige Lehrerinnen und Lehrer mit Unterrichtsfach Mathematik besteht erheblicher Bedarf.

Aber das Studium besteht nicht aus Mathematik allein und zur Studentenzeit gehört weit mehr als nur das Studium.Als klassisch geprägte Universität lädt unsere 511 Jahre alte Martin-Luther-Universität zu vielen Blicken über den Tellerrand und zum engen Kontakt zu Studierenden anderer Fachrichtungen ein.In Halle (Saale) können Sie sich auf ein für eine Stadt dieser Größe bemerkenswert reichhaltiges kulturelles Angebot freuen, das darauf wartet, von Ihnen entdeckt zu werden.

Begleiten wird Sie über die gesamte Studienzeit unser Institut für Mathematik.Trotz der überschaubaren Größe mit nur elf Professuren sind bei uns die Mathematik und ihre Didaktik universitär vertreten.Anders als an Massenuniversitäten können wir Ihnen ein enges Verhältnis zwischen Studierenden und Hochschullehrern bieten.Nutzen Sie diesen Standortvorteil.Wir freuen uns auf Sie, auf die Studierenden des Jahrgangs 2013, und wünschen Ihnen eine glückliche Studentenzeit sowie ein interessantes und erfolgreiches Studium an unserer Martin-Luther-Universität.

Prof. Dr. Dr. h.c. Wilfried Grecksch (Geschäftsführender Direktor)\\
Institut für Mathematik
    

\subsection{Institutsdirektor des Instituts für Informatik}

Liebe Studienanfängerinnen und –anfänger,

ich begrüße Sie ganz herzlich an unserem Institut. Willkommen in unserer kleinen Familie. Sie haben sich entschieden, an einem der, von der Anzahl der Professoren her  kleinsten Informatik-Instituten in Deutschland zu studieren, wenn nicht an dem kleinsten. Dementsprechend sind auch die Studierendenzahlen mit jeweils um die 40 Studienanfängern in Informatik und Bioinformatik überschaubar. Man kennt sich, die Studierenden untereinander, aber auch die Dozent(inn)en die Studierenden. Man spricht miteinander vor und nach einer Vorlesung, im Treppenhaus, im Flur, über fachliche aber auch persönliche Themen.
 
Sie haben sich entschieden, ein als sehr schwer eingeschätztes Fach zu studieren. Das Studium der Informatik und Bioinformatik wird Ihnen viel abverlangen, viel Arbeit, große Ausdauer. Dabei wird, insbesondere im ersten Jahr Ihres Studiums, Faktenwissen nicht im Mittelpunkt stehen, vielmehr neue, mit Ab­straktion verbundene Denkweisen, die notwendig sind, um später erfolgreich als Informatikerin oder Informatiker arbeiten zu können. Das Erlernen dieser zum Teil ungewohnten Denkweisen wird vielen von Ihnen schwer fallen und wird Ihnen nur durch viel Üben gelingen. Neue Denkweisen zu erlernen, geht nicht durch Auswändiglernen bzw. „sich kurz vor der Prüfung mal hinsetzen“. Der Vergleich zu einer Spitzensportlerin bzw. einem Spitzensportler ist in diesem Zusammenhang durchaus legitim. Auch sie bzw. er muss jahrelang, zum Teil täglich, trainieren, um die Bewegungsabläufe einzutrainieren und die angestrebte Leistung zu erbringen.

Nach Regelstudienplan werden Sie in jedem Semester ungefähr 20 Stunden Präsenz- veranstaltungen haben. Besuchen Sie die Vorlesungen, die Übungen und die Tutorien! Arbeiten Sie aktiv in Übungen und Tutorien mit! Bearbeiten Sie die Übungs-aufgaben! Fragen Sie in den Veranstaltungen nach! Diskutieren Sie die Themen und Aufgaben mit Ihren Kommilitoninnen und Kommilitonen! All dies wird Ihnen helfen, nicht nur das einzelne Modul zu bestehen, sondern Ihr Studium erfolgreich, gewinnbringend für Sie und Ihren späteren Arbeitgeber zu gestalten. 

Vieles wird Ihnen schnell leichter von der Hand gehen, wenn Sie die Angebote des Instituts und der „alten Hasen“ wahrnehmen. Nutzen Sie insbesondere die Möglichkeiten, die ein kleines Institut bietet! Wir, die Professoren, die Mitarbeiterinnen und Mitarbeiter stehen bei Problemen und für Fragen zur Verfügung! Viele Türen stehen tagsüber offen, durch die Sie gehen können. Verstehen Sie uns nicht als Kontrahenten! Lassen Sie uns gemeinsame Sache machen! 

Es lohnt sich, die Kraft zu investieren: de facto sicherer Arbeitsplatz, freie Wahl des Arbeitsortes, ob in Sachsen-Anhalt, in Deutschland oder irgendwo außerhalb von Deutschland, und, das ist das Wichtigste, Spaß an der Arbeit. Ein Problem mit mathematisch-informatischen Methoden zu „knacken“ oder ein Projekt zu einem erfolgreichen Ende zu führen, ob als Einzelkämpfer oder in der Gruppe, es ist ein gutes Gefühl.
 
Last but not least: Wir sind uns bewusst, dass es neben der Universität Verpflichtungen oder Hobbies geben kann, denen Sie nachkommen müssen oder wollen. Ich denke beispielsweise an Studierende mit Kindern, an Studierende, die einem Leistungssport nachgehen, an Studierende, die für  ihren Lebensunterhalt selbst aufkommen müssen. Wenngleich es nicht immer möglich sein wird, Familie, Sport, oder Nebenjob optimal mit Ihrem Studium zu verbinden, sprechen Sie uns bitte rechtzeitig auf Probleme an. Wir werden unser Bestes geben, Ihnen weiterzuhelfen.

Denken Sie bitte auch daran, dass sich das Leben nicht nur hinter Büchern und Rechnern abspielt. Eine Universität zu besuchen, heißt sich weiterentwickeln, nicht nur fachspezifisch, sondern auch als Mensch. 

In diesem Sinne wünsche ich Ihnen eine spannende, erfolgreiche und schöne Zeit am Institut für Informatik unserer Alma Mater.

Univ.-Prof. Dr. Paul Molitor (Geschäftsführender Direktor)\\
Institut für Informatik
