\section{Begrüßungen}

\subsection{Vorsitzender des Fachschaftsrats Mathematik/Informatik}

Hallo liebe Erstsemester und Kommilitonen,

im Namen des Fachschaftsrates Mathematik/Informatik begrüße ich euch allerherzlichst an der Martin-Luther-Universität. \\
Nun beginnt für euch ein neuer Lebensabschnitt und sicherlich seid ihr gespannt, was euch in den nächsten Wochen und Monaten an der Universität erwarten wird. Mit dem Beginn der Vorlesungen und den dazugehörigen Übungen umrahmt vom studentischen Leben neben der Uni erwartet euch ein neuer Tagesablauf. 
Seid ab den ersten Wochen fleißig und teilt euch eure Zeit gut ein; auf diesen Weg werdet ihr schnell euren ganz persönlichen Wochenrhythmus finden.
Habt Mut, eure Kommilitonen anzusprechen, gemeinsam lassen sich die Herausforderungen des Studiums leichter bewältigen. Auf diesem Wege werdet ihr viele neue Leute kennenlernen und aus den anfänglichen Lerngruppen werden im Laufe der Zeit Freundschaften. Traut euch auch, Dozenten und Übungsleiter anzusprechen. Neben erhofften Problemlösungen entsteht so schnell ein nettes Gespräch, welches euch über den Rand des Vorlesungsstoffes hinaus blicken lässt.

Darüber hinaus stehen wir euch als Fachschaftsrat jederzeit bei Fragen und Problemen zu Seite. Wir hoffen, euch mit diesem kleinen Heft eine hilfreiche Übersicht zur Bewältigung der anstehenden Herausforderungen bereitzustellen.
Aber auch der Fachschaftsrat braucht eure Hilfe. Wir sind -- wie ihr auch -- Studenten und freuen uns sehr über eure Unterstützung. Falls ihr also dem Fachschaftsrat beitreten wollt, um die Interessen eurer Kommilitonen zu vertreten und den Studenten bei Fragen und Problemen zur Seite zu stehen, dann stellt euch im nächsten Frühjahr doch zur Wahl und werdet selbst ein Mitglied des Fachschaftsrates. Alle Informationen hierzu und weitere Möglichkeiten sich zu engagieren, findet ihr auf unserer Homepage.
Ich wünsche euch viel Spaß und Erfolg im Studium. Auf dass ihr die kleinen und größeren Hürden des Studiums meistert und am Ende einen der begehrten Abschlüsse in den Händen haltet. Habt ihr dieses erreicht, so stehen euch alle Türen für das künftige Berufsleben offen. Bis dahin versuchen wir euch zu unterstützen, wo wir nur können.

Zum Schluss möchte ich euch noch unsere Homepage ans Herz legen, auf der ihr weitere Informationen findet und den Terminkalender mit den nächsten anstehenden Fachschaftsveranstaltungen: \texttt{http://fsr-matheinfo.de/}

Sarah Henkel\\
Vorsitzende des Fachschaftsrates Mathematik/Informatik


\subsection{Dekan und Studiendekan der Naturwissenschaftlichen Fakultät II}

Liebe Studienanfängerinnen und Studienanfänger der Naturwissenschaftlichen
Fakultät II,

herzlich willkommen an der Martin-Luther-Universität und an der
Naturwissenschaftlichen Fakultät II mit den Instituten für Chemie, für Physik
und für Mathematik. Wir freuen uns, dass Sie sich für einen der mehr als 20
Studiengänge unserer Fakultät entschieden haben, und hoffen, dass Sie sich
schnell bei uns einleben werden.

Der moderne Weinberg-Campus begrüßt Sie mit vielen in den vergangenen Jahren
komplett sanierten oder neu erbauten Gebäuden und kurzen Wegen zwischen den
Universitätsinstituten, Forschungseinrichtungen und Bibliotheken. Direkt am
Von-Danckelmann-/Von-Seckendorff-Platz steht Ihnen die Heide-Mensa des
Studentenwerks zur Verfügung.

Die im Jahr 2006 aus den Fachbereichen Chemie und Physik hervorgegangene
Naturwissenschaftlichen Fakultät II umfasst heute neben diesen beiden
klassischen naturwissenschaftlichen Fachrichtungen auch das Institut für
Mathematik und deckt damit ein breites fachliches Spektrum der experimentellen
und theoretischen Forschung und Lehre ab. Alle zwölf Bachelor- und
Masterstudiengänge der Fakultät wurden Mitte vergangenen Jahres erneut
akkreditiert und tragen damit auch künftig dieses Gütesiegel der renommierten
Akkreditierungsagentur ACQUIN.

Wir wünschen Ihnen viel Spaß an unserer Fakultät, viel Erfolg im Studium und
eine gute Zeit an unserer altehrwürdigen Martin-Luther-Universität.

Prof.\ Dr.\ Wolfgang H.\ Binder (Dekan),\\
Prof.\ Dr.\ Martin Arnold (Studiendekan)\\
Naturwissenschaftliche Fakultät II 

\newpage

\subsection{Dekan und Studiendekan der Naturwissenschaftlichen Fakultät III}

Liebe Studentinnen, liebe Studenten,

ein herzliches Willkommen an der Naturwissenschaftlichen Fakultät III.
Wir freuen uns auf die Zusammenarbeit mit Ihnen.
Sie haben sich mit der Informatik oder Bioinformatik für ein zukunftsträchtiges Fach entschieden, mit dem Sie nach einem erfolgreichen Abschluss exzellente Berufsaussichten haben.
Nahezu alle Branchen sind von der IT durchdrungen und benötigen deshalb Absolventen von Informatikstudiengängen.

Bevor Sie soweit sind liegt jedoch das Studium vor Ihnen.
Mit dem Studium beginnt für Sie ein neuer Lebensabschnitt.
Sie werden viele neue Eindrücke gewinnen, neue Freunde im Kreise Ihrer Kommilitonen (hoffentlich nicht nur in Ihrem eigenen Fach) werden hinzukommen, und nicht zuletzt werden Sie mit einer für Sie neuen Art des Lernens im universitären Umfeld vertraut.
Gerade in den Informatikfächern ist es wichtig, sich selbst den notwendigen Stoff zu erarbeiten.
Vorlesungen, Übungen, Seminare und Praktika bieten den Rahmen dafür.
Scheuen Sie sich nicht, in den Lehrveranstaltungen Fragen an die Dozentinnen und Dozenten zu stellen. 

Bei aller Arbeit für das Studium, sollte auch das soziale Studentenleben nicht zu kurz kommen. Gehen Sie auf Feste, werden Sie in Sport oder Kultur aktiv, arbeiten Sie in der studentischen Interessenvertretung mit -- die Universität und die Vereine in Halle und Umgebung bieten vielfältige Möglichkeiten.
Die mit solchen Erfahrungen erworbenen sogenannten Soft-Skills sind sowohl für Ihr Studium als auch für das spätere Berufsleben mehr als nützlich.

Wir wünschen Ihnen einen guten Start ins Studium sowie viel Spass und viel Erfolg beim Studieren

Prof. Dr. Olaf Christen\\
Dekan der Naturwissenschaftlichen Fakultät III 

\newpage

\subsection{Institutsdirektorin des Instituts für Mathematik}

Liebe Studienanfängerinnen, liebe Studienanfänger,

herzlich willkommen an der Martin-Luther-Universität und speziell an unserem Institut! Wir freuen uns, dass Sie sich für ein Mathematik-Studium entschieden haben. Da ich selbst die Vorlesung "Lineare Algebra" halten werde, werde ich viele von Ihnen bald persönlich kennenlernen und freue mich darauf, Sie bei den ersten Schritten in der Welt der Universitätsmathematik zu begleiten.\par
Nun beginnt eine spannende Zeit für Sie - ganz gleichgültig, ob Sie das Studium mit konkreten beruflichen Plänen aufgenommen haben oder ob Sie aus purer Neugier im Bereich Mathematik gelandet sind: Ich kann Ihnen einige Herausforderungen und Überraschungen versprechen!\par
Was Sie in den nächsten Jahren lernen werden, wird weit über das hinausgehen, was Ihnen im Schulfach Mathematik begegnet ist und wird die Themen oft in ein völlig neues Licht rücken. Angefangen bei den Grundlagenvorlesungen erarbeiten Sie sich Stück für Stück ein umfassendes Wissen und Verständnis für die Zusammenhänge zwischen den mathematischen Gebieten. Dabei lernen Sie unterschiedliche Bereiche der reinen und der angewandten Mathematik kennen und entscheiden später selbst, wo Sie sich spezialisieren möchten. Hauptbestandteil des Studiums sind die Entwicklung abstrakter Theorien und Methoden, die Vernetzung der verschiedenen Gebiete und die Behandlung konkreter Fragestellungen aus der Praxis (z.B. Naturwissenschaften, Medizin, Wirtschaftswissenschaften).\\
Wir betreuen Sie individuell auf Ihrem Weg zum Beruf, sei es nun in der Schule, in einem der zahlreichen Tätigkeitsfelder in der Wirtschaft oder in der Forschung. Bei Fragen oder Problemen finden Sie stets ein offenes Ohr.\par

Aber vergessen Sie dabei nicht den Blick über den Tellerrand! An der Martin-Luther- Universität gibt es ein breites Spektrum von Veranstaltungen und Möglichkeiten, sich einzubringen, und die Stadt Halle bietet ein bemerkenswert reichhaltiges kulturelles Angebot. Viel Spaß dabei, das alles zu entdecken! \par

Wir freuen uns darauf, Sie während Ihres Studiums zu begleiten und wünschen Ihnen viel Spaß und Erfolg!\par

Rebecca Waldecker \\
Geschäftsführende Direktorin des Instituts für Mathematik

\newpage

\subsection{Institutsdirektor des Instituts für Informatik}

Liebe Studierende,

ich begrüße Sie ganz herzlich als neue Studierende an unserem Institut.
Willkommen in unserer, vergleichsweise kleinen Familie. Ich freue mich, dass Sie
sich nicht nur entschlossen haben, Informatik, Bioinformatik oder Lehramt
Informatik zu studieren, d.\,h.\ ein Fach von  besonderer und weiterhin stark
wachsender Bedeutung für unsere Gesellschaft, sondern auch, dass Sie für Ihr
Studium unser Institut ausgewählt haben.

Warum auch immer, die Grußworte des Geschäftsführenden Direktors des Instituts
für Informatik an die neuen Studierenden kommen in den Heften für Erstsemestler
immer als letzte in der Reihe der Grußworte. Ich könnte jetzt vieles aus den
vorangegangenen Grußworten wiederholen. Ja, Sie beginnen einen neuen
Lebensabschnitt. Haben Sie aber auch schon mal  gemacht, als Sie von der
Grundschule zum Gymnasium gewechselt sind. War damals für viele nicht einfach.
Aber Sie haben es alle geschafft. Ja, nehmen Sie bitte Ihr Studium ernst,
investieren Sie bitte Arbeit in Ihr Studium, das Studium der Informatik und
Bioinformatik ist ein Vollzeitjob. Ist auch nicht wirklich neu! Das Abitur haben
die meisten von Ihnen doch nicht geschenkt bekommen? Ja, nehmen Sie sich Zeit,
gemeinsam mit Kommilitonen regelmäßig etwas zu unternehmen, machen Sie Sport
oder/und machen Sie hin- und wieder Party. Das ist nun wirklich nicht neu! Oder
haben Sie das bisher nicht gemacht?

Wirklich neu für Sie ist, dass wir Sie als erwachsene Persönlichkeiten behandeln
werden. Sie sind verantwortlich für das, was Sie tun. Sie sind verantwortlich,
ob Sie die Vorlesungen und Übungen besuchen. Sie sind verantwortlich, wie gut
vorbereitet Sie in die Kurse, Übungen und Prüfungen gehen. Sie sind
verantwortlich, dass Sie sich neben Ihrem Studium einen Ausgleich schaffen. Sie
sind verantwortlich, ob Sie die Angebote des Instituts wahrnehmen. Sie sind
verantwortlich, ob Sie den Rat der \enquote{alten Hasen} und der Dozenten annehmen. Sie
sind verantwortlich, ob Sie die Möglichkeiten, die ein eher kleines Instituts
wie das unserige bietet, wahrnehmen und bspw.\ die offenen Türen zu den
Arbeitszimmern der Dozenten nutzen.

Wir werden Ihnen zur Seite stehen. Verstehen Sie uns nicht als Kontrahenten,
auch wenn wir bei Prüfungen auf der anderen Seite des Tisches sitzen! Lassen Sie
uns gemeinsame Sache machen, für Ihre Zukunft, die Zukunft des Instituts und die
Zukunft des Landes.

Wir sind uns bewusst, dass es neben der Universität Verpflichtungen oder Hobbies
geben kann, denen Sie nachkommen müssen oder wollen. Auch wir haben solche
Verpflichtungen und Hobbies. Ich denke beispielsweise an Studierende mit
Kindern, an Studierende, die einem Leistungssport nachgehen, an Studierende, die
für  ihren Lebensunterhalt selbst aufkommen müssen. Wenngleich es nicht immer
möglich sein wird, Familie, Sport oder Nebenjob optimal mit Ihrem Studium zu
verbinden, sprechen Sie uns bitte rechtzeitig auf Probleme an. Wir werden unser
Bestes geben, Ihnen weiterzuhelfen.

Zum Schluss meines Grußwortes ein erster und wahrscheinlich für Ihr Studium der
wichtigste Rat: Das Studium der Informatik, der Bioinformatik und Lehramt
Informatik wird Ihnen viel abverlangen. Gemeinsam werden Sie das schaffen. Sie
sollten in kleinen Gruppen von 2--3 Personen arbeiten, sowohl bei der
Bearbeitung der Übungen wie auch bei den Prüfungsvorbereitungen. Sie helfen und
unterstützen sich so gegenseitig. Ohne diese gegenseitige Hilfe und
Unterstützung wird es janz schwer. Gemeinsam heißt aber auch gemeinsam mit den
Dozenten. Fragen Sie die Dozenten, wenn Sie etwas nicht verstanden haben. Sitzen
Sie bitte nicht einfach rum und vergeuden Ihre Zeit.

In diesem Sinne wünsche ich Ihnen eine spannende, erfolgreiche und schöne Zeit
am Institut für Informatik unserer Alma Mater.

Prof.\ Dr.\ Paul Molitor \\
Geschäftsführender Direktor des Instituts für Informatik
