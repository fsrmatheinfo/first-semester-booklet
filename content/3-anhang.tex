\section{Lexikon}

\begin{encyclopedia}
    \definition{akademisches Viertel}{
        siehe cum tempore.
    }
    \definition{BAFöG}{
        Bundesausbildungsförderungsgesetz.
        Regelt die finanzielle Unterstützung von Studierenden
        durch den Staat in Form von Stipendium und Darlehen.
        siehe auch BAFöG-Amt.
    }
    \definition{cum tempore}{
        Das \textquote{akademische Viertel}, eine Viertelstunde,
        wird der angegebenen Zeit hinzugerechnet.
        Z.B.: 14:00~c.t. = 14:15~Uhr.
    }
    \definition{c.t.}{
        siehe cum tempore.
    }
    \definition{sine tempore}{
        Die angegebene Zeit ist wörtlich zu verstehen.
        Z.B.: 14:00~s.t. = 14:00~Uhr.
    }
    \definition{s.t.}{
        siehe sine tempore.
    }
    \definition{Dekan}{
        Oberhaupt einer Fakultät,
        Sprecher des Fakultätsrates.
    }
    \definition{dies academicus}{
        Feiertag an der Uni,
        an dem keine Vorlesungen oder Übungen stattfinden.
        Findet 1-2 mal im Semester zu besonderen Veranstaltungen statt.
    }
    \definition{Fakultät}{
        Gruppierung von Instituten.
        Die MLU besteht aus 10~Fakultäten.
        Das Institut für Mathematik ist in der
        Naturwissenschaftlichen Fakultät~II (NatFak~II),
        das Institut für Informatik in der
        Naturwissenschaftlichen Fakultät~III (NatFak~III).
    }
    \definition{HiWi}{
        Hilfswissenschaftler bzw. Wissenschaftliche Hilfskraft.
        HiWis sind in der Regel selbst Studenten,
        meist höheren Semesters.
    }
    \definition{Hochschulgesetz}{
        Landeshochschulgesetz (HSG~LSA).
        Regelt Strukturen der Hochschulen in Sachsen-Anhalt,
        definiert Inhalt des Studiums,
        legt fest, wer studieren darf etc.
    }
    \definition{(IT)\textsuperscript{2}}{
        Industrietag~Informationstechnik.
        Fachtagung des UZI.
        Findet jedes Semester statt.
    }
    \definition{ITZ}{
        IT-Servicezentrum.
        Früher Universitätsrechenzentrum (URZ).
        \url{https://itz.uni-halle.de/}
    }
    \definition{Kanzler}{
        Verwaltungschef der Uni.
    }
    \definition{Kommilitone}{
        Mitstudierende, Studiengenosse.
    }
    \definition{Mensa}{
        Speisehaus für Studierende mit vergünstigtem Essen.
    }
    \definition{MLU}{
        Kürzel für die Martin-Luther-Universität~Halle-Wittenberg
    }
    \definition{Rektor}{
        Akademisches Oberhaupt und erster Repräsentant der Universität.
    }
    \definition{Senat}{
        Oberstes beschlussfassendes Organ der gesamten Universität.
        Zuständig für die meisten Entscheidungen zu Forschung und Studium.
    }
    \definition{SoSe}{
        siehe SS.
    }
    \definition{SS}{
        Sommersemester.
        Vom 01.04. bis zum 30.09.
    }
    \definition{StuRa}{
        Studierendenrat.
        Oberstes Organ der studentischen Selbstverwaltung.
        \url{https://stura.uni-halle.de/}
    }
    \definition{SWS}{
        Semesterwochenstunden.
        Stunden je Woche im Semester.
        Z.B.: Modul mit 4+2~SWS bedeutet jede Woche
        4~Stunden Vorlesung und 2~Stunden Übung.
    }
    \definition{ULB}{
        Universitäts- und Landesbibliothek.
        Katalog unter \url{https://lhhal.gbv.de/}.
        \url{https://bibliothek.uni-halle.de/}
    }
    \definition{Bib}{
        siehe ULB.
    }
    \definition{Bibliothek}{
        siehe ULB.
    }
    \definition{USZ}{
        Universitätssportzentrum.
        Anmeldung für Kurse online: \url{https://usz.uni-halle.de/}.
        Unbedingt rechtzeitig am Mittwoch der ersten Vorlesungswoche anmelden.
    }
    \definition{UZI}{
        Universitätszentrum für Informatik.
        \url{https://uzi.uni-halle.de/}
    }
    \definition{WiSe}{
        siehe WS.
    }
    \definition{WS}{
        Wintersemester.
        Vom 01.10. bis zum 31.03.
    }
    \definition{VSP}{
        Von-Seckendorff-Platz.
        Informatikseite am Heidecampus.
    }
    \definition{VDP}{
        Von-Danckelmann-Platz.
        Physikerseite am Heidecampus.
    }
    \definition{TLS}{
        Theodor-Lieser-Straße.
        Straße an der Heidi.
    }
    \definition{Cantor-Haus}{
        Heimat des Instituts für Mathematik.
    }
    \definition{AudiMax}{
        Größter Hörsaal, am Hauptcampus.
    }
    \definition{Mel}{
        Melanchtonianum, Hörsaalgebäude am Hauptcampus.
    }
    \definition{Hauptcampus}{
        Zentraler Campus am Universitätsplatz.
    }
    \definition{Kleine Ulli}{
        Kleine Ulrichstraße.
        Kneipenstraße in der Innenstadt.
    }
    \definition{LP}{
        Leistungspunkte.
        Bekommt man für jedes Modul.
        Zum Erreichen des Abschlusses notwendig.
    }
    \definition{CP}{
        Credit Points, siehe LP.
    }
    \definition{ZLB}{
        Zentrum für Lehrerbildung.
        In der Nähe des Hauptcampus.
        Koordination \& Beratung für das Lehramtsstudium.
    }
    \definition{ASQ}{
        Allgemeine Schlüsselqualifikationen.
        Fachfremde Wahlpflichtmodule.
        Soll \textquote{Fachidioten} vorbeugen.
    }
    \definition{FSR}{
        Fachschaftsrat
        \begin{itemize}[noitemsep]
            \item Studierendenvertretung / Interessenvertretung
            \item für Fachbereiche Mathematik und Informatik
            \item 9~Mitglieder
            \item jährlich gewählt im Mai
            \item Zahlreiche Veranstaltungen, u.a. Spieleabende, Weihnachtsfeier, NatFusion, Sommerfest
            \item Freier Eintritt, günstige Verpflegung bei allen Veranstaltungen.
            \item Bindeglied zu Professoren und Mitarbeitern
            \item finanziert Anschaffungen für die Studierenden
            \item Ansprechpartner für Fragen und Probleme
            \item finanziert sich aus einem Teil des Semesterbeitrags (\EUR{7,50}~pro Semester)
            \item öffentliche Sitzung im Semester vsl. donnerstags 18 Uhr
            \item jeder darf mitmachen -- lass dich wählen
        \end{itemize}
        \url{https://fachschaft.mathinf.uni-halle.de}\\
        \email{fachschaft@mathinf.uni-halle.de} \\
        Instagram: fsrmatheinfo \\
        Von-Seckendorff-Platz 1, Raum 0.31
    }
    \definition{Fachschaft}{
        siehe FSR.
    }
    \definition{Spieleabend}{
        Spiel, Spaß \& Bier mit Karaoke.
        Veranstaltung des FSR.
    }
    \definition{Weihnachtsfeier}{
        Kekse, Glühwein \& Schrottwichteln in Weihnachtsatmosphäre.
        Veranstaltung des FSR.
    }
    \definition{NatFusion}{
        Open-Air-Party im Frühjahr mit der Physik, Chemie, Biochemie.
        Veranstaltung des FSR.
    } 
    \definition{Sommerfest}{
        Outdoor-Spieleabend mit Festplatten-Zielwurf.
        Veranstaltung des FSR.
    }
    \definition{Tutorium}{
        Zusätzliche Übungsmöglichkeit zum Vorlesungsinhalt.
        Wird oft von älteren Studierenden geleitet.
    }
    \definition{Übung}{
        Besprechung des Vorlesungsinhalts,
        Bearbeiten von Übungsaufgaben,
        Besprechen der Übungsserien.
        Manchmal Anwesenheitspflicht.
    }
    \definition{Übungsserie}{
        Verpflichtende Vor- oder Teilleistung.
        Meist müssen ca. 50\% der Punkte erreicht werden.
        Häufig kann in Gruppen gearbeitet werden.
    }
    \definition{Hausaufgabe}{
        siehe Übungsserie.
    }
    \definition{Vorlesung}{
        Vortrag des Dozenten über das Modulthema.
    }
    \definition{Club}{
        siehe Nachtleben.
    }
    \definition{Bar}{
        siehe Nachtleben.
    }
    \definition{Restaurant}{
        Viele Restaurants in der Kleinen Ulli und Sternstraße.
    }
    \definition{Sternstraße}{
        Straße mit vielen Restaurants.
    }
    \definition{Nachtleben}{
        u.a. Turm, Palette, Druschba, 
        Schorre, Charles Bronson, Flo-Po,
        Enchi, Bauernclub, Objekt~5.
        Weitere: \url{https://kulturfalter.de/}
    }
    \definition{Palette}{
        Tanzbar.
    }
    \definition{Druschba}{
        Club.
    }
    \definition{Schorre}{
        Club.
    }
    \definition{Objekt~5}{
        Bar.
    }
    \definition{Charles Bronson}{
        Club.
    }
    \definition{Turm}{
        Studentenclub in der Moritzburg.
    }
    \definition{Flo-Po}{
        Flower-Power.
        Musikkneipe mit Karaoke.
    }
    \definition{Enchi}{
        Enchilada.
        Mexikanische Cocktailbar am Hauptcampus.
        Montags Cocktailwürfeln.
    }
    \definition{Bauernclub}{
        Studentenclub.
    }
    \definition{hastuzeit}{
        Studierendenzeitschrift über Studentenalltag,
        Hochschulpolitik und Erfahrungen mit Auslandssemestern.
        Vom StuRa herausgegeben.
        \url{https://hastuzeit.de/}
    }
    \definition{Fakultätsrat}{
        Vertretung der Professoren, 
        Studierenden und Mitarbeiter einer Fakultät.
    }
    \definition{MDV}{
        Mitteldeutscher Verkehrsbund.
        Regionalverkehr in der Region Leipzig/Halle.
        Gültigkeitsbereich des Semestertickets
        (nicht gültig für MDV-Norderweiterung!)
    }
    \definition{Semesterticket}{
        Verkehrsticket im Studierendenausweis,
        gültig im MDV-Gebiet
        (ausgenommen MDV-Norderweiterung!)
    }
    \definition{Studierendenausweis}{
        Universeller Ausweis fürs Studium.
        Immer dabei haben, jedes Semester rechtzeitig validieren!
        \begin{itemize}[noitemsep]
            \item Mensa
            \item Drucken/Kopieren
            \item Bibliothek
            \item Semesterticket
            \item Identitätsnachweis für Prüfungen und Wahlen
            \item Schlüsselkarte
            \item Ermäßigungskarte im öffentlichen Leben
        \end{itemize}
        Validierungsstellen im Löwengebäude,
        der Heidi, der Weini,
        und Haus 31 in den Franckeschen Stiftungen.
    }
    \definition{Studentenwerk}{
        Sozialer Service für Studierende
        \begin{itemize}[noitemsep]
            \item BAFöG
            \item Mensa
            \item Wohnen
            \item Beratung
            \item Kinderbetreuung
            \item Kultur
        \end{itemize}
        \url{https://studentenwerk-halle.de/}
    }
    \definition{Studieren mit Kind}{
        Kitas, Vergünstigungen etc. vom Studentenwerk.
        \url{https://uni-halle.de/familiengerecht}
    }
    \definition{Stipendium}{
        Förderung aus Stiftungen und Wirtschaft 
        zur finanziellen Unterstützung im Studium.
        Übersicht auf \url{https://stipendienlotse.de/}.
    }
    \definition{Mathe-Treff}{
        Treffpunkt für ratlose Studis,
        Unterstützung bei allen Mathe-Fragen.
        Täglich geöffnet: R\,1.30 im Cantor-Haus.
        \url{https://studieninfo.mathematik.uni-halle.de/mathe-treff/}
    }
    \definition{Info-Treff}{
        Treffpunkt für ratlose Studis,
        Unterstützung bei allen Info-Fragen.
        Di. bis Fr. geöffnet: R\,5.08 im VSP.
    }
    \definition{Mentoring}{
        Mentoringprogramm für Bioinformatiker.
        \url{https://studieninfo.informatik.uni-halle.de/unser-institut/mentoring/}
    }
    \definition{LaTeX}{
        Textsatzprogramm.
        Gut geeignet für mathematische \& wissenschaftliche Texte.
        Von Professoren bevorzugt; besser als Word.
        LaTeX-Kurs des FSR oder LaTeX-ASQ besuchen.
        LaTeX-Vorlage unter \url{https://fachschaft.mathinf.uni-halle.de/informationen/latex}.
    }
    \definition{PDF}{
        Abgabeformat für die meisten Übungsserien.
        Kann mit LaTeX erstellt werden.
    }
    \definition{Pool}{
        siehe Computerpool.
    }
    \definition{Computerpool}{
        In der 3. Etage im VSP.
        Kostenlose Druckseiten.
        \url{https://informatik.uni-halle.de/studium/pools/}
    }
    \definition{WLAN}{
        Uniweites kostenloses WLAN über das DFN.
        Einrichtung unter \url{https://wlan.itz.uni-halle.de/}.
        Siehe auch eduroam.
    }
    \definition{eduroam}{
        Weltweites kostenloses WLAN.
        Einrichtung unter \url{https://cat.eduroam.org/}.
    }
    \definition{DFN}{
        Deutsches Forschungsnetzwerk.
        Universitäres (ziemlich schnelles) Breitbandnetz.
        \url{https://dfn.de/}
    }
    \definition{Campus}{
        Ort des universitären Geschehens.
    }
    \definition{Tulpe}{
        Burse zur Tulpe. Mensa am Hauptcampus.
    }
    \definition{Löwengebäude}{
        Zentrales, representatives Gebäude am Hauptcampus.
    }
    \definition{Imma-Amt}{
        Immatrikulationsamt im Löwengebäude.
    }
    \definition{Prüfungsamt}{
        Anlaufstelle für Probleme bei der Modul-/Prüfungsanmeldung im Löwenportal. \\
        Info\,/\,NatFak~III: \url{https://natfak3.uni-halle.de/pruefungsamt\_natfak\_3/} \\
        Mathe\,/\,NatFak~II: \url{http://natfak2.uni-halle.de/studium/}
    }
    \definition{Löwenportal}{
        Online-Plattform für Uni-Bürokratie.
        Modul-, Prüfungsanmeldung, Bescheinigungen, Rückmeldung.
        \url{https://loewenportal.uni-halle.de/}
    }
    \definition{BAFöG-Amt}{
        Anlaufstelle für Probleme mit BAföG, in der Weini. 
        \sloppy\url{https://studentenwerk-halle.de/bafoeg-studienfinanzierung/bafoeg/allgemeine-infos-zum-bafoeg/}
    }
    \definition{Ersti}{
        Erstsemester, das bist zum Beispiel du.
    }
    \definition{Uni-Sport}{
        siehe USZ.
    }
    \definition{Weini}{
        Weinbergmensa, am Weinbergcampus.
    } % Neue Abkürzung, eingeführt von Peter Böttcher 2019
    \definition{Heidi}{
        Heidemensa, am Heidecampus.
    } % Neue Abkürzung, eingeführt von Peter Böttcher 2019
    \definition{Studienleistung}{
        Teilleistung zum Bestehen des Moduls.
    }
    \definition{Modulleistung}{
        Hauptleistung zum Bestehen des Moduls.
    }
    \definition{Modulvorleistung}{
        Vorleistung zur Anmeldung der Modulprüfung.
    }
    \definition{Modul}{
        Themenbezogene Vorlesungsreihe.
        Unterteilung des Gesamtstudiums.
        Beim erfolgreichen Abschluss bekommt man LP.
    }
    \definition{Anwendungsfach}{
        Für Informatik, fachfremde Module 
        zur Anwendung der informatischen Kenntnisse.
    }
    \definition{Spezialisierung}{
        Wahlpflichtmodule in der Informatik.
    }
    \definition{Selbststudium}{
        Zeit außerhalb des Stundenplans 
        zur Vor- und Nacharbeit der Vorlesung.
    }
    \definition{Heidecampus}{
        Campus der Naturwissenschaften, insb. Mathe/Info.
    }
    \definition{Weinbergcampus}{
        Campus der Chemie/Biologie.
    }
    \definition{Theater}{
        u.a. Opernhaus, Neues Theater, Thalia-Theater, Puppentheater. 
        \url{https://buehnen-halle.de/}
    }
    \definition{Kino}{
        u.a. Unikino, CinemaxX, thelight Cinema, Puschkino, Lux-Kino, Zazie.
    }
    \definition{Stud.IP}{
        Vorlesungs- und Austauschportal.
        Termine, Skripte, Lernübungen, Dateien, Kontakte.
        \url{https://studip.uni-halle.de/}
    }
    \definition{E-Mail}{
        Offizielle Uni-Mails über Studmail.
        \url{https://studmail.uni-halle.de/}.
        Regelmäßig lesen und/oder weiterleiten!
    }
    \definition{Mail}{
        siehe E-Mail.
    }
    \definition{SSC}{
        Studierenden-Service-Center.
        Immatrikulationsamt \& allgemeine Studienberatung.
        \url{https://uni-halle.de/ssc/}.
    }
    \definition{International Office}{
        Organisation für Auslandssemester.
        \url{https://www.international.uni-halle.de/}
    }
    \definition{MLUconf}{
        Videokonferenz-Dienst der MLU, für datensichere Online-Vorlesungen und -Übungen,
        basierend auf Big Blue Button.
        Kann von Studierenden auch privat genutzt werden.
        \url{https://mluconf.uni-halle.de/}
    }
    \definition{MLUchat}{
        Chat-Dienst der MLU, für datensichere Absprachen mit Kommilitonen und Profs.
        \url{https://chat.uni-halle.de/}
    }
    \definition{Rocket.Chat}{
        siehe MLUchat.
    }
    \definition{Webex}{
        Lizensierter Videokonferenz-Dienst von Cisco, für Online-Vorlesungen und -Übungen.
        \url{https://uni-halle.webex.com/}
    }
    \definition{BBB}{
        siehe Big Blue Button.
    }
    \definition{Big Blue Button}{
        Open-Source Videokonferenz-Dienst, v.a. genutzt von MLUconf.
    }
    \definition{Opencast}{
        Open-Source Videostreaming-Dienst, v.a. für Vorlesungsaufzeichnungen im Stud.IP.
    }
    \definition{ILIAS}{
        Lernplattform für Online-Kurse und Selbsttests, verknüpft im Stud.IP.
    }
    \definition{Übungsportal}{
        Plattform zum Hochladen der wöchentlichen Übungsserien, verknüpft im Stud.IP.
    }
    \definition{YAPEX}{
        Programmierumgebung für OOP-Übungen, verknüpft im Übungsportal.
    }
    \definition{LLZ}{
        Zentrum für multi­mediales Lehren und Lernen.
        \url{https://llz.uni-halle.de/}
    }
    \definition{GCH}{
        Georg-Cantor-Haus an der Theodor-Lieser-Straße. Institut für Mathematik.
    }
\end{encyclopedia}
