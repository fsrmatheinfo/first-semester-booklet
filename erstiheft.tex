% zuletzt ueberarbeitet 2014
% von Felix Knispel
% bei Fragen zu dieser LaTeX-„Vorlage“:
%  jan.wagner@fsr-matheinfo.de (ob der noch mitliest, ist fraglich)
%
% Anmerkungen:
%  • mit gedit und LaTeX-Plugin erstellt, verwendet rubber, d.h. das TeX kann
%    Fehler haben und weder mit latex noch pdflatex kann es kompiliert werden,
%    das muss aber nicht sein
%  • Erstellung auch problemlos mit TeXMaker möglich (2014)
%  • Alle Files sind nun in UTF-8. D.h. typographische Anführungszeichen „“ sind
%    explizit erwünscht, also nicht "'"´ oder so etwas nehmen. Auch Umlaute
%    normal schreiben!

\documentclass[a5paper,9pt,twoside]{scrartcl}
% Pakete einbinden
\usepackage[ngerman]{babel}
\usepackage[utf8]{inputenc}
\usepackage{graphicx}                   % schön Grafiken einbinden
\usepackage{amssymb, amstext, amsmath}  % Mathe-Zeug
\usepackage{rotating}                   % für die Titelseite: Schrift rotieren
\usepackage{type1cm}                    % für die Titelseite: Schrift vergrößern
\usepackage{eso-pic,calc}               % fuer BG-Bild
\usepackage[official]{eurosym}          % benötigt für das Euro-Symbol
\usepackage{fancyhdr}                   % für die schönen Kopfzeilen
\usepackage{tabularx}                   % damit man Tabellen exakt so breit wie
                                        % Seiten machen kann; var. Spalten
                                        % werden mit X markiert; Bsp:
                                        % \begin{tabularx}{\textwidth}{lrcX}
\usepackage{textpos}                    % damit kann man das Titelbild in die
                                        % Ecke setzen
\usepackage{color}                      % für Farben
\definecolor{OurLinkColor}{rgb}{0,0,0}  % Schönes Schwarz für die Links im Dokument
\usepackage[pdftex,                     % für die klickbaren Querverweise im PDF
            colorlinks,                 % (Inhaltsverzeichnis, Seitenzahlen,
            linkcolor=OurLinkColor,     % Tabellen,..)
            citecolor=OurLinkColor,
%           pagecolor=OurLinkColor, not available anymore
            filecolor=OurLinkColor,
            urlcolor=OurLinkColor]{hyperref}
\usepackage{helvet}                     % Serifenlose Schrift ist Helvetica
\usepackage{tikz}						% Tikz für Vektorgrafiken

% Hier definieren wir den kompletten Seitenaufbau neu, damit wir so viel Platz
% wie möglich erhalten. Alle Ränder sind 1,5 cm breit.
\setlength{\voffset}{-1in}
\setlength{\hoffset}{-1in}
\setlength{\topmargin}{1.5cm}
\setlength{\oddsidemargin}{1.5cm}
\setlength{\evensidemargin}{1.5cm}
\setlength{\textwidth}{\paperwidth-3cm}
\setlength{\headheight}{0.5cm}
\setlength{\headsep}{0.5cm}
\setlength{\textheight}{\paperheight-4cm}
\setlength{\footskip}{0cm}
\setlength{\marginparwidth}{0cm}
\setlength{\marginparsep}{0cm}
\setlength{\parindent}{0cm}

\parskip = 1em plus 1em minus 1em

% Hier können Wörter eingetragen werden, welche nicht getrennt werden sollen
% oder wie sie getrennt werden sollen. (z.B. Arsch-kriech-er)
\hyphenation{
      uniwlan
      }

% Hier werden die schönen Kopfzeilen angeschaltet und definiert/angpasst.
\pagestyle{fancy}
% renew Sectionmark, damit Nummerierung der Sections nicht in den Kopfzeilen
% erscheint. Vgl.:
% http://www.tex.ac.uk/tex-archive/macros/latex/contrib/fancyhdr/fancyhdr.pdf
\renewcommand{\sectionmark}[1]{\markboth{#1}{}}
\fancyhf{}
\fancyhead[OR]{\sffamily\thepage}
\fancyhead[OL]{\sffamily\nouppercase{\leftmark}}
\fancyhead[EL]{\sffamily\thepage}
\fancyhead[ER]{\sffamily\nouppercase{\leftmark}}

% Bis zu welcher Gliederungsebene soll nummeriert werden
\setcounter{secnumdepth}{2}
% Bis zu welcher Ebene Einträge ins Inhaltsverzeichnis aufgenommen werden.
\setcounter{tocdepth}{2}


% sonsiges
\renewcommand{\baselinestretch}{1.1} % Zeilenabstand 1.1, für bessere Lesbarkeit

% Wo liegen die Bilder
\graphicspath{{./Bilder/}}

% Ein neuer Befehl um Link zu Webseiten und E-Mail-Adressen konsistent
% darzustellen
\newcommand{\verweis}[1]{\mbox{\texttt{#1}}}
\newcommand{\web}[1]{\verweis{\href{#1}{#1}}}
\newcommand{\email}[1]{\verweis{\href{mailto:#1}{#1}}}
% Ein Makro, falls man es besser findet http:// mit anzugeben oder ein anderes
% Protokoll verwendet
\newcommand{\link}[1]{\verweis{\href{#1}{#1}}}

% Hurensoehne etc. verhindern
\widowpenalty=10000

\begin{document}

% Titelseite
\thispagestyle{empty}

\title{Ersti-Hilfe für die Institute für 
    Mathematik und Informatik zum
    Wintersemester 2018/19}
\author{Fachschaftsrat Mathematik/Informatik}
\date{\today}

{
    \sffamily
    \begin{textblock*}{\linewidth}(0cm, -1.5cm)
        \fontsize{1cm}{1cm}\selectfont
        Martin-Luther-Universität \\
        Halle-Wittenberg
    \end{textblock*}
    \begin{textblock*}{\linewidth}(0cm, \textheight - 5.5cm)
        \fontsize{0.925cm}{0.925cm}\selectfont
        \includegraphics[width=0.7\linewidth]{turtle} \\
        Mathematik/Informatik
    \end{textblock*}
    \begin{textblock*}{2cm}(\linewidth - 2cm, 2cm)
        \begin{sideways}
            \fontsize{2.5cm}{2.5cm}\selectfont
            \bfseries
            Ersti-Hilfe ’18
        \end{sideways}
    \end{textblock*}
}

% Impressum
\newpage
\thispagestyle{empty}
\vspace*{\stretch{1}}
\subsubsection{Impressum}
Herausgegeben vom Fachschaftsrat der Institute für Mathematik und Informatik der
Martin-Luther-Universität Halle-Wittenberg.\\[1.0em]
\begin{tabularx}{\textwidth}{@{}lX@{}}
 Autoren: & Franziska Deutschmann,                                              %%%%% AKTUALISIEREN
            Sebastian Fröhlich,
            Stefan Hante,
            Marie-Luise Hein,
            Mara Jakob,
            Doris Kube,
            Marcus Pöckelmann,
            Martin Porsch,
            Steffen Rechner,
            Benjamin Saul,
            Michael Schneider,
            Iris Eckert,
            Jan Wagner,
            Frank Wusterhausen,
            Felix Knispel,
            Florian Johnke,
            Florian Lücke,
            Felix Pahlow,
            Tuğçe Kuru,
            Peter Böttcher,
            Anna Hemmann,
            Jan Heinrich Reimer\\
 Layout:  & Martin Porsch, Benjamin Saul, Jan Wagner\\
 Stand:   & \today\\
% Druck:   & Universitätsdruckerei Halle-Wittenberg
\end{tabularx}\\[1.0em]
Powered by \LaTeX.
\pagebreak



%Inhaltsverzeichnis
\newpage
\setcounter{page}{1}                    % Damit bezieht sich die Nummerierung
                                        % der Seiten auf alle, außer die des
                                        % bunten Einbandes
\thispagestyle{empty}
\tableofcontents

%eingebundene Kapitel
\newpage
\section{Begrüßungen}

\subsection{Vorsitzende des Fachschaftsrats Mathematik/Informatik}

Hallo liebe Erstsemester und Kommilitonen,

im Namen des Fachschaftsrates Mathematik/Informatik begrüße ich euch allerherzlichst an der Martin-Luther-Universität. \\
Nun beginnt für euch ein neuer Lebensabschnitt und sicherlich seid ihr gespannt, was euch in den nächsten Wochen und Monaten an der Universität erwarten wird. Mit dem Beginn der Vorlesungen und den dazugehörigen Übungen umrahmt vom studentischen Leben neben der Uni erwartet euch ein neuer Tagesablauf. 
Seid ab den ersten Wochen fleißig und teilt euch eure Zeit gut ein; auf diesen Weg werdet ihr schnell euren ganz persönlichen Wochenrhythmus finden.
Habt Mut, eure Kommilitonen anzusprechen, gemeinsam lassen sich die Herausforderungen des Studiums leichter bewältigen. Auf diesem Wege werdet ihr viele neue Leute kennenlernen und aus den anfänglichen Lerngruppen werden im Laufe der Zeit Freundschaften. Traut euch auch, Dozenten und Übungsleiter anzusprechen. Neben erhofften Problemlösungen entsteht so schnell ein nettes Gespräch, welches euch über den Rand des Vorlesungsstoffes hinaus blicken lässt.

Darüber hinaus stehen wir euch als Fachschaftsrat jederzeit bei Fragen und Problemen zu Seite. Wir hoffen, euch mit diesem kleinen Heft eine hilfreiche Übersicht zur Bewältigung der anstehenden Herausforderungen bereitzustellen.
Aber auch der Fachschaftsrat braucht eure Hilfe. Wir sind -- wie ihr auch -- Studenten und freuen uns sehr über eure Unterstützung. Falls ihr also dem Fachschaftsrat beitreten wollt, um die Interessen eurer Kommilitonen zu vertreten und den Studenten bei Fragen und Problemen zur Seite zu stehen, dann stellt euch im nächsten Frühjahr doch zur Wahl und werdet selbst ein Mitglied des Fachschaftsrates. Alle Informationen hierzu und weitere Möglichkeiten sich zu engagieren, findet ihr auf unserer Homepage.
Ich wünsche euch viel Spaß und Erfolg im Studium. Auf dass ihr die kleinen und größeren Hürden des Studiums meistert und am Ende einen der begehrten Abschlüsse in den Händen haltet. Habt ihr dieses erreicht, so stehen euch alle Türen für das künftige Berufsleben offen. Bis dahin versuchen wir euch zu unterstützen, wo wir nur können.

Zum Schluss möchte ich euch noch unsere Homepage ans Herz legen, auf der ihr weitere Informationen findet und den Terminkalender mit den nächsten anstehenden Fachschaftsveranstaltungen: \texttt{http://fsr-matheinfo.de/}

Florian Lücke\\
Vorsitzende des Fachschaftsrates Mathematik/Informatik


\subsection{Dekan und Studiendekan der Naturwissenschaftlichen Fakultät II}

Liebe Studienanfängerinnen und Studienanfänger der Naturwissenschaftlichen
Fakultät II,

herzlich willkommen an der Martin-Luther-Universität und an der Naturwissenschaftlichen Fakultät II mit den Instituten für Chemie, für Physik und für Mathematik. Wir freuen uns, dass Sie sich für einen der mehr als 20 Studiengänge unserer Fakultät entschieden haben, und hoffen, dass Sie sich schnell bei uns einleben werden.

Der moderne Weinberg-Campus begrüßt Sie mit vielen in den vergangenen Jahren komplett sanierten oder neu erbauten Gebäuden und kurzen Wegen zwischen den Universitätsinstituten, Forschungseinrichtungen und Bibliotheken. Nach Abschluss der Sanierungsarbeiten wird uns auch wieder direkt am Von-Danckelmann-/Von-Seckendorff-Platz eine Mensa des Studentenwerks zur Verfügung stehen.

Die im Jahr 2006 aus den Fachbereichen Chemie und Physik hervorgegangene Naturwissenschaftlichen Fakultät II umfasst heute neben diesen beiden klassischen naturwissenschaftlichen Fachrichtungen auch das Institut für Mathematik und deckt damit ein breites fachliches Spektrum der experimentellen und theoretischen Forschung und Lehre ab. Alle zwölf Bachelor- und Masterstudiengänge der Fakultät wurden Mitte diesen Jahres erneut akkreditiert und tragen damit auch künftig dieses Gütesiegel der renommierten Akkreditierungsagentur ACQUIN.
 
Wir wünschen Ihnen viel Spaß an unserer Fakultät, viel Erfolg im Studium und eine gute Zeit an unserer altehrwürdigen Martin-Luther-Universität.
 
Prof. Dr. Wolfgang H. Binder (Dekan)\\
Prof. Dr. Martin Arnold (Studiendekan)\\
Naturwissenschaftliche Fakultät II (Chemie, Physik, Mathematik)


\subsection{Dekan und Studiendekan der Naturwissenschaftlichen Fakultät III}

Liebe Studentinnen, liebe Studenten,

ein herzliches Willkommen an der Naturwissenschaftlichen Fakultät III.
Wir freuen uns auf die Zusammenarbeit mit Ihnen.
Sie haben sich mit der Informatik oder Bioinformatik für ein zukunftsträchtiges Fach entschieden, mit dem Sie nach einem erfolgreichen Abschluss exzellente Berufsaussichten haben.
Nahezu alle Branchen sind von der IT durchdrungen und benötigen deshalb Absolventen von Informatikstudiengängen.

Bevor Sie soweit sind liegt jedoch das Studium vor Ihnen.
Mit dem Studium beginnt für Sie ein neuer Lebensabschnitt.
Sie werden viele neue Eindrücke gewinnen, neue Freunde im Kreise Ihrer Kommilitonen (hoffentlich nicht nur in Ihrem eigenen Fach) werden hinzukommen, und nicht zuletzt werden Sie mit einer für Sie neuen Art des Lernens im universitären Umfeld vertraut.
Gerade in den Informatikfächern ist es wichtig, sich selbst den notwendigen Stoff zu erarbeiten.
Vorlesungen, Übungen, Seminare und Praktika bieten den Rahmen dafür.
Scheuen Sie sich nicht, in den Lehrveranstaltungen Fragen an die Dozentinnen und Dozenten zu stellen. 

Bei aller Arbeit für das Studium, sollte auch das soziale Studentenleben nicht zu kurz kommen. Gehen Sie auf Feste, werden Sie in Sport oder Kultur aktiv, arbeiten Sie in der studentischen Interessenvertretung mit -- die Universität und die Vereine in Halle und Umgebung bieten vielfältige Möglichkeiten.
Die mit solchen Erfahrungen erworbenen sogenannten Soft-Skills sind sowohl für Ihr Studium als auch für das spätere Berufsleben mehr als nützlich.

Wir wünschen Ihnen einen guten Start ins Studium sowie viel Spass und viel Erfolg beim Studieren

Prof. Dr. Olaf Christen (Dekan)\\
Prof. Dr. Wolf Zimmermann (Studiendekan)\\
Naturwissenschaftliche Fakultät III (Agrar- und Ernährungswissenschaften, \newline Geowissenschaften, Informatik)


\subsection{Institutsdirektor des Instituts für Mathematik}

Liebe Studienanfängerinnen, liebe Studienanfänger,

herzlich willkommen an der Martin-Luther-Universität und herzlichen Glückwunsch zu Ihrer Entscheidung für ein mathematisches Studienfach, sei es einer unserer Bachelor-Studiengänge „Mathematik“ oder „Wirtschaftsmathematik“ oder einer der Lehramtstudiengänge im Fach Mathematik.

Sie beginnen ein vielseitiges und spannendes, aber auch anspruchsvolles Studium, in dem Sie u.a. ein sehr viel umfassenderes Bild der Mathematik gewinnen werden, als es der Schulunterricht vermitteln kann.Sie werden die Mathematik als eine Wissenschaft kennen lernen, die sich mit präziser Denk- und Arbeitsweise gleichermaßen vielen abstrakten Fragestellungen wie auch angewandten Problemen aus den Naturwissenschaften, der Medizin, den Wirtschaftswissenschaften und anderen Bereichen widmet.Viele von uns Mathematikerinnen und Mathematikern begeistern sich sowohl für die abstrakte als auch für die angewandte Seite der Mathematik.Sie sind herzlich eingeladen, sich hiervon „anstecken“ zu lassen.

Unter Ihnen sind nicht wenige, die mit dem vor Ihnen liegenden Studium in einer wirtschaftlich schwierigen Zeit den Grundstein für ein späteres erfolgreiches Berufsleben legen wollen.Sie haben eine gute Wahl getroffen, denn den Absolventinnen und Absolventen mathematischer Studiengänge bieten sich viele Möglichkeiten in Wissenschaft und Forschung, in der freien Wirtschaft und bei Banken und Versicherungen.Und auch für künftige Lehrerinnen und Lehrer mit Unterrichtsfach Mathematik besteht erheblicher Bedarf.

Aber das Studium besteht nicht aus Mathematik allein und zur Studentenzeit gehört weit mehr als nur das Studium.Als klassisch geprägte Universität lädt unsere 511 Jahre alte Martin-Luther-Universität zu vielen Blicken über den Tellerrand und zum engen Kontakt zu Studierenden anderer Fachrichtungen ein.In Halle (Saale) können Sie sich auf ein für eine Stadt dieser Größe bemerkenswert reichhaltiges kulturelles Angebot freuen, das darauf wartet, von Ihnen entdeckt zu werden.

Begleiten wird Sie über die gesamte Studienzeit unser Institut für Mathematik.Trotz der überschaubaren Größe mit nur elf Professuren sind bei uns die Mathematik und ihre Didaktik universitär vertreten.Anders als an Massenuniversitäten können wir Ihnen ein enges Verhältnis zwischen Studierenden und Hochschullehrern bieten.Nutzen Sie diesen Standortvorteil.Wir freuen uns auf Sie, auf die Studierenden des Jahrgangs 2013, und wünschen Ihnen eine glückliche Studentenzeit sowie ein interessantes und erfolgreiches Studium an unserer Martin-Luther-Universität.

Prof. Dr. Dr. h.c. Wilfried Grecksch (Geschäftsführender Direktor)\\
Institut für Mathematik
    

\subsection{Institutsdirektor des Instituts für Informatik}

Liebe Studienanfängerinnen und –anfänger,

ich begrüße Sie ganz herzlich an unserem Institut. Willkommen in unserer kleinen Familie. Sie haben sich entschieden, an einem der, von der Anzahl der Professoren her  kleinsten Informatik-Instituten in Deutschland zu studieren, wenn nicht an dem kleinsten. Dementsprechend sind auch die Studierendenzahlen mit jeweils um die 40 Studienanfängern in Informatik und Bioinformatik überschaubar. Man kennt sich, die Studierenden untereinander, aber auch die Dozent(inn)en die Studierenden. Man spricht miteinander vor und nach einer Vorlesung, im Treppenhaus, im Flur, über fachliche aber auch persönliche Themen.
 
Sie haben sich entschieden, ein als sehr schwer eingeschätztes Fach zu studieren. 
Das Studium der Informatik und Bioinformatik wird Ihnen viel abverlangen, viel Arbeit, große Ausdauer.
Dabei wird, insbesondere im ersten Jahr Ihres Studiums, Faktenwissen nicht im Mittelpunkt stehen, vielmehr neue, mit Abstraktion verbundene Denkweisen, die notwendig sind, um später erfolgreich als Informatikerin oder Informatiker arbeiten zu können.
Das Erlernen dieser zum Teil ungewohnten Denkweisen wird vielen von Ihnen schwer fallen und wird Ihnen nur durch viel Üben gelingen.
Neue Denkweisen zu erlernen, geht nicht durch Auswändiglernen bzw. „sich kurz vor der Prüfung mal hinsetzen“.
Der Vergleich zu einer Spitzensportlerin bzw. einem Spitzensportler ist in diesem Zusammenhang durchaus legitim.
Auch sie bzw. er muss jahrelang, zum Teil täglich, trainieren, um die Bewegungsabläufe einzutrainieren und die angestrebte Leistung zu erbringen.

Nach Regelstudienplan werden Sie in jedem Semester ungefähr 20 Stunden Präsenz- veranstaltungen haben. Besuchen Sie die Vorlesungen, die Übungen und die Tutorien! Arbeiten Sie aktiv in Übungen und Tutorien mit! Bearbeiten Sie die Übungs-aufgaben! Fragen Sie in den Veranstaltungen nach! Diskutieren Sie die Themen und Aufgaben mit Ihren Kommilitoninnen und Kommilitonen! All dies wird Ihnen helfen, nicht nur das einzelne Modul zu bestehen, sondern Ihr Studium erfolgreich, gewinnbringend für Sie und Ihren späteren Arbeitgeber zu gestalten. 

Vieles wird Ihnen schnell leichter von der Hand gehen, wenn Sie die Angebote des Instituts und der „alten Hasen“ wahrnehmen. Nutzen Sie insbesondere die Möglichkeiten, die ein kleines Institut bietet! Wir, die Professoren, die Mitarbeiterinnen und Mitarbeiter stehen bei Problemen und für Fragen zur Verfügung! Viele Türen stehen tagsüber offen, durch die Sie gehen können. Verstehen Sie uns nicht als Kontrahenten! Lassen Sie uns gemeinsame Sache machen! 

Es lohnt sich, die Kraft zu investieren: de facto sicherer Arbeitsplatz, freie Wahl des Arbeitsortes, ob in Sachsen-Anhalt, in Deutschland oder irgendwo außerhalb von Deutschland, und, das ist das Wichtigste, Spaß an der Arbeit. Ein Problem mit mathematisch-informatischen Methoden zu „knacken“ oder ein Projekt zu einem erfolgreichen Ende zu führen, ob als Einzelkämpfer oder in der Gruppe, es ist ein gutes Gefühl.
 
Last but not least: Wir sind uns bewusst, dass es neben der Universität Verpflichtungen oder Hobbies geben kann, denen Sie nachkommen müssen oder wollen. Ich denke beispielsweise an Studierende mit Kindern, an Studierende, die einem Leistungssport nachgehen, an Studierende, die für  ihren Lebensunterhalt selbst aufkommen müssen. Wenngleich es nicht immer möglich sein wird, Familie, Sport, oder Nebenjob optimal mit Ihrem Studium zu verbinden, sprechen Sie uns bitte rechtzeitig auf Probleme an. Wir werden unser Bestes geben, Ihnen weiterzuhelfen.

Denken Sie bitte auch daran, dass sich das Leben nicht nur hinter Büchern und Rechnern abspielt. Eine Universität zu besuchen, heißt sich weiterentwickeln, nicht nur fachspezifisch, sondern auch als Mensch. 

In diesem Sinne wünsche ich Ihnen eine spannende, erfolgreiche und schöne Zeit am Institut für Informatik unserer Alma Mater.

Univ.-Prof. Dr. Paul Molitor (Geschäftsführender Direktor)\\
Institut für Informatik

\newpage

\section[Was sind Leistungs-Punkte?]{Was sind Leistungs-Punkte~(LP) bzw. Credit Points~(CP)?}
\label{creditpoints}

Das Bachelor-Studium umfasst insgesamt 180~Leistungspunkte~(LP).
Dies entspricht sechs Semestern Regelstudienzeit.
Ein Leistungspunkt entspricht 30~Stunden studentischer Arbeitszeit, die sich aus dem Besuch von Vorlesungen und Übungen, sowie dem Ausarbeiten von Hausaufgaben zusammensetzt.

Das Master-Studium weist insgesamt 120~LP auf, also vier Semester Regelstudienzeit.
Der weiterbildende, berufsbezogene Master kann auch nur 60~LP, also zwei Semester, umfassen.

Ein Bachelor- bzw. ein Master-Studiengang besteht aus einem oder zwei Studienprogrammen.
Ein Studienprogramm regelt das Studium einer wissenschaftlichen Disziplin und entspricht einem Studienfach.
Es kann aber auch interdisziplinär angelegt sein.

Drei Varianten des Bachelor-Studiums sind möglich:
\begin{enumerate}
    \item Studiengänge mit einem Studienprogramm (180~LP)
    \item Studiengänge mit zwei gleichgewichtigen Studienprogrammen\\
          (90~LP pro Fach)
    \item Studiengänge mit einem großen und einem kleinen Studienprogramm\\
          (120~LP und 60~LP)
\end{enumerate}

Die Lehramtsstudiengänge sind 2007 modularisiert worden, das bedeutet, dass in diesem Studiengang ebenfalls Leistungspunkte erworben werden müssen.
Ein Lehramtsstudent für Sekundarschulen muss 80~LP in seinem Erstfach und 75~LP in seinem Zweitfach erwerben, sowie weitere 85~LP in weiteren Bereichen.
Ein Lehramtsstudent für Gymnasien hingegen muss 95~LP in seinem Erstfach und 90~LP in seinem Zweitfach erwerben und ebenfalls die oben genannten 85~LP.
Genaueres kann man auf folgender Webseite einsehen:\\[1.0em]
\web{http://www.zlb.uni-halle.de/studium/}

\newpage

\section{Studiengänge}

In diesem Kapitel wird euch ein kurzer Überblick über euren Studiengang gegeben. Aktuelle Informationen zu den je Studiengang belegbaren Modulen und Regelstudienpläne findet ihr jeweils in den Modulhandbüchern auf den Internetseiten der Fakultäten.
Bei Fragen oder wenn ihr genauere Informationen sucht, solltet ihr euch zunächst an den Fachschaftsrat oder euren Studiengangsberater wenden.
In einigen Fällen wird euch der Fachschaftsrat nicht mehr weiter helfen können.
Dann wird der Studiengangsberater oft der einzige sein, der euch bei Problemen einen Lösungsweg oder eine Lösung anbieten kann.
Der Studiengangsberater ist euer Ansprechpartner, wenn ihr Fragen oder Probleme
mit der Studienordnung, mit der Wahl von Modulen oder anderen \enquote{technischen Problemen} des Studiums habt.
Er kennt die Studienordnung genau und hat daher einen tiefen Einblick in die Details des Studiums.

\textbf{Studienberater Mathematik und Wirtschaftsmathematik}\\
Dr.\ Hans-Georg Rackwitz\\
Theodor-Lieser-Str.\ 5, Raum 127\\
Tel: (0345) 55 24608\\
\email{hans-georg.rackwitz@mathematik.uni-halle.de}\\

\textbf{Studienberater Informatik und Bioinformatik}\\
Apl.\ Prof.\ Dr.\ Klaus Reinhardt\\
Von-Seckendorff-Platz 1, Raum 2.10\\
Tel: (0345) 55 24770\\
\email{klaus.reinhardt@informatik.uni-halle.de}


%%%%%%%%%%%%%%%%%%%%%%%%% INFORMATIK %%%%%%%%%%%%%%%%%%%%%%%%%%%%%%

\subsection{Informatik}
\label{studiengang_informatik}

\begin{table}[tbp]
	\begin{small}
		\begin{tabularx}{\textwidth}{|X||c|c|c|c|c|c||r|}
			\hline
			\textbf{Modul}&\multicolumn{6}{l||}{\textbf{Leistungspunkte im Sem.}}&\textbf{LP}\\\hline
			&1&2&3&4&5&6&\\\hline\hline
			\multicolumn{8}{|c|}{\textbf{Informatik-Grundlagen}}\\\hline
			Objektorientierte Programmierung&5&&&&&&5\\\hline
			Einführung in die Rechnerarchitektur&5&&&&&&5\\\hline
			Mathematische Grundlagen der Informatik und Konzepte der Modellierung &7&8&&&&&15\\\hline
			Einführung in Betriebssysteme&&5&&&&&5\\\hline
			Konzepte der Programmierung&&&5&&&&5\\\hline
			Automaten und Berechenbarkeit&&&&10&&&10\\\hline
			Einführung in die technische Informatik&&5&&&&&5\\\hline
			Datenstrukturen und effiziente Algorithmen~I&&5&&&&&5\\\hline\hline
			%        \multicolumn{1}{c||}{Summe}&17&23&5&10&&&55\\\hline\hline
			\multicolumn{8}{|c|}{\textbf{Mathematik}}\\\hline
			Diskrete Strukturen, lineare Algebra und Analysis (Mathematik B)&8&7&&&&&15\\\hline
			Stochastik für Informatiker&&&&5&&&5\\\hline\hline
			%        \multicolumn{1}{c||}{Summe}&8&7&5&&&&20\\\hline\hline
			\multicolumn{8}{|c|}{\textbf{Informatik-Vertiefung}}\\\hline
			Datenbanken~I&&&10&&&&10\\\hline
			Datenstrukturen und effiziente Algorithmen~II&&&5&&&&5\\\hline
			Einführung in die Rechnernetze und verteilte Systeme&&&&&5&&5\\\hline
			Softwaretechnik&&&5&&&&5\\\hline
			Einführung in die Computergraphik oder Einführung in die Bildverarbeitung&&&&5&&&5\\\hline
			Gestaltung und Durchführung von Fachvorträgen in der Informatik&&&&&5&&5\\\hline
			Projektpraktikum&&&&5&10&&15\\\hline\hline
			%        \multicolumn{1}{c||}{Summe}&&&20&10&20&&50
			\multicolumn{8}{|c|}{\textbf{Nebenfach}}\\\hline
			Anwendungsfach&&&5&5&5&&15\\\hline\hline
			\multicolumn{8}{|c|}{\textbf{Pflichtbereich ASQ}}\\\hline
			Allgemeine Schlüsselqualifikationen&5&&&&&5&10\\\hline\hline
			%        \multicolumn{1}{|c||}{Summe}&5&&&&&5&10\\\hline\hline
			\textbf{Bachelor-Arbeit}&&&&&&15&15\\\hline
		\end{tabularx}
	\end{small}
	\caption{Regelstudienplan für den Bachelor Informatik (180~LP)
		\label{plan-info}. Hinzu kommen noch Veranstaltungen zur „Spezialisierung“, siehe Tabelle~\ref{plan-info2}}
\end{table}

\begin{table}[!th]
	\begin{small}
		\begin{tabularx}{\textwidth}{|X||c|c|c|c|c|c||r|}
			\hline
			\textbf{Modul}&\multicolumn{6}{l||}{\textbf{Leistungspunkte im Sem.}}&\textbf{LP}\\\hline
			&1&2&3&4&5&6&\\\hline\hline
			\multicolumn{8}{|X|}{\textbf{Spezialisierung}}\\\hline
			Informatik&&&&&0/5&0/5/10&0/5/10/15\\
			Wirtschaftsinformatik&&&&&0/5&0/5/10&0/5/10/15\\
			Bioinformatik&&&&&0/5&0/5/10&0/5/10/15\\
			Mathematik&&&&&&0/5&0/5\\
			Anwendungsfach&&&&&&0/5&0/5\\\hline
		\end{tabularx}
	\end{small}
	\caption{Spezialiserung im Studiengang Bachelor Informatik (180~LP) im 5. und 6.~Semester. \label{plan-info2}}
\end{table}

Liebe Studienanfängerinnen, liebe Studienanfänger,\par

es freut uns sehr, dass ihr euch für ein Studium der Informatik an der MLU entschieden habt. Die folgenden Seiten enthalten einige Informationen über den prinzipiellen Ablauf des Studiums, welche euch den Einstieg erleichtern sollen.\par

In den ersten Fachsemestern werden alle Grundkenntnisse der Informatik und Mathematik erworben, auf welchen später das gesamte Studium aufbaut.
In den Vorlesungen wird euch neue Stoff vorgetragen und durch gezielte Fragen eurerseits können hier bereits erste Unklarheiten beseitigt werden.
Zur Festigung des vermittelten Wissens erfolgt die Bearbeitung von Aufgaben in den zur jeweiligen Vorlesung gehörigen Übungen in kleinen Gruppen mit je einem Übungsleiter.
In der Regel müsst ihr 50\%~bis~60\% dieser Aufgaben korrekt gelöst haben, um das Modul erfolgreich abschließen zu können, weswegen die Bildung von Lerngruppen sehr empfehlenswert ist.
Habt ihr die meisten Aufgaben durchgearbeitet und den Lösungsweg verstanden, sollte es kein Problem mehr sein, die Prüfungen zu bestehen.

In der Regel werdet ihr im 3.~Fachsemester eine Vorlesung in dem von euch frei wählbaren Anwendungsfach hören.
Hier werden euch Grundkenntnisse des jeweiligen Gebietes vermittelt.
Geeignet sind beispielsweise Chemie, Physik, Biologie, Mathematik, Geowissenschaften, Psychologie, BWL oder VWL.

Ab dem fünften Semester könnt ihr Spezialisierungsmodule wählen, welche insgesamt 15 LP einbringen müssen.
Welche Module das sind, wird euch im Gegensatz zum übrigen Regelstudienplan allerdings nicht fest vorgeschrieben.
Ihr habt die Freiheit, aus einer ganzen Reihe von Modulen diejenigen zu wählen, die euch am meisten Spaß machen.
Diese Module werden typischerweise im Bereich der Informatik liegen, allerdings könnt ihr auch ausgewählte Vorlesungen aus eurem jeweiligen Anwendungsfach oder aus den Bereichen Mathematik, Wirtschafts- oder Bioinformatik hören und als Spezialisierungsmodul anrechnen lassen.

Neben diesen Pflichtveranstaltungen gibt es insbesondere in den ersten Semestern oftmals die Möglichkeit, Tutorien zusätzlich zu den regulären Übungen zu besuchen.
Im Gegensatz zu den Übungen werden diese in den meisten Fällen von Studenten geleitet, wodurch die Barriere zwischen Studierenden und Dozenten abgebaut wird.
In den Tutorien könnt ihr ganz ungezwungen Fragen stellen, über den Vorlesungstoff oder Übungsaufgaben sprechen oder euch Vorlesungsinhalte anhand von Beispielen näherbringen lassen. Einige Professoren finden es auch gut, wenn viele Fragen während der Vorlesung gestellt werden und es so zu Diskussionen kommt.

Wenn ihr noch Fragen habt, zögert nicht den Fachschaftsrat oder euren Studienberater Dr.~Bauer anzusprechen.

\textbf{Ansprechpartner im Fachschaftsrat:}\\
Peter Böttcher, \email{peter.boettcher@fsr-matheinfo.de}\\
%%%%%%%%%%%%%%%%%%%%%%%%%%%%%%%%%%%% BIOINFORMATIK %%%%%%%%%%%%%%%%%%%%%%%%%%%

\newpage

\subsection{Bioinformatik}
\label{studiengang_bioinformatik}

\begin{table}[tbp]
	\begin{small}
		\begin{tabularx}{\textwidth}{|X||c|c|c|c|c|c||r|}
			\hline
			\textbf{Modul}&\multicolumn{6}{l||}{\textbf{Leistungspunkte im Sem.}}&\textbf{LP}\\\hline
			&1&2&3&4&5&6&\\\hline\hline
			\multicolumn{8}{|c|}{\textbf{Pflichtbereich Informatik}}\\\hline
			Objektorientierte Programmierung&5&&&&&&5\\\hline
			Mathematische Grundlagen der Informatik und Konzepte der Modellierung&7&8&&&&&15\\\hline
			Softwaretechnik&&&5&&&&5\\\hline
			Datenstrukturen und effiziente Algorithmen~I&&5&&&&&5\\\hline
			Datenbanken I&&&&&10&&10\\\hline
			Algorithmen auf Sequenzen~I&&&&5&&&5\\\hline
			Statistische Datenanalyse in der Bioinformatik~I&&&&&5&&5\\\hline
			Spezielle Probleme der Bioinformatik &&&&5&&&5\\\hline
			Gestaltung und Durchführung von Fachvorträgen in der Bioinformatik &&&5&&&&5\\\hline\hline
			%    Summe&12&13&10&10&15&60\\\hline\hline
			\multicolumn{8}{|c|}{\textbf{Pflichtbereich Mathematik}}\\\hline
			Diskrete Strukturen, lineare Algebra und Analysis (Mathematik B)&8&7&&&&&15\\\hline
			Stochastik&&&&5&&&5\\\hline\hline
			%    Summe&8&7&&5&&20\\\hline\hline
			\multicolumn{8}{|c|}{\textbf{Pflichtbereich Biologie}}\\\hline
			Zellbiologie&5&&&&&&5\\\hline
			Botanik für Bioinformatiker&&&5&&&&5\\\hline
			Zoologie für Bioinformatiker&&&5&&&&5\\\hline
			Ökologie für Bioinformatiker&&&&5&&&5\\\hline
			Genetik für Bioinformatiker&&&5&&&&5\\\hline
			Mikrobiologie für Bioinformatiker&&&&5&&&5\\\hline\hline
			%    Summe&5&&15&10&&30\\\hline\hline
			\multicolumn{8}{|c|}{\textbf{Pflichtbereich Biochemie}}\\\hline
			Allgemeine Biochemie&&&&&10&&10\\\hline\hline
			%    Summe&&&10&&&10\\\hline\hline
			\multicolumn{8}{|c|}{\textbf{Pflichtbereich Chemie}}\\\hline
			Allgemeine und Grundlagen der physikalischen Chemie&5&&&&&&5\\\hline
			Organische und Bioorganische Chemie&&5&5&&&&10\\\hline\hline
			%    Summe&5&5&5&&&15\\\hline\hline
			\multicolumn{8}{|c|}{\textbf{Pflichtbereich ASQ}}\\\hline
			Allgemeine Schlüsselqualifikationen&&5&5&&&&10\\\hline\hline
			\textbf{Bachelor-Arbeit}&&&&&&15&15\\\hline
			%    Summe&&5&5&&&10\\\hline
		\end{tabularx}
	\end{small}
	\caption{Regelstudienplan für den Bachelor Bioinformatik (180~LP). Hinzu kommen Veranstaltungen aus dem Wahlbereich, s. Tabelle \ref{plan-bioinfo2} \label{plan-bioinfo}}
\end{table}

\begin{table}[!th]
	\begin{small}
		\begin{tabularx}{\textwidth}{|X||r|}
			\hline
			\textbf{Modul}&\textbf{LP}\\\hline\hline
			\multicolumn{2}{|c|}{\textbf{Wahlbereich Informatik}}\\\hline
			Automaten und Berechenbarkeit&10\\\hline
			Datenstrukturen und effiziente Algorithmen~II&5\\\hline
			Einführung in Rechnernetze und verteilte Systeme&5\\\hline
			Einführung in Rechnerarchitektur&5\\\hline
			Konzepte der Programmierung&5\\\hline
			Einführung in die Computergrafik&5\\\hline
			Theorie der Datensicherheit&5\\\hline
			Einführung in die Bildverarbeitung&5\\\hline
			Komponenten- und Servicebasierte Software&5\\\hline
			Einführung in die künstliche Intelligenz&5\\\hline
			Grundlagen des WWW&5\\\hline
			Introduction to Biodiversity Informatics&5\\\hline\hline
			%        Summe&10&5&5&20&15&55\\\hline\hline
			\multicolumn{2}{|c|}{\textbf{Wahlbereich Biowissenschaften}}\\\hline
			Orientierungsmodul Biologie&5\\\hline
			Pflanzenphysiologie Bioinformatiker&5\\\hline
			Spezielle Mikrobiologie für Bioinformatiker&5\\\hline
			Tierphysiologie für Bioinformatiker&5\\\hline
			Populationsgenetik für Bioinformatiker&5\\\hline
			Biogeographie&5\\\hline
			Molekulare Genetik für Bioinformatiker&5\\\hline
			Biochemie und Biotechnologie für Bioinformatiker (Fortgeschrittene)&10\\\hline
			Ökologiepraktikum&5\\\hline
			Molekularbiologie in der Tierzucht&5\\\hline
			Molekulargenetik der Nutzpflanzen~I&5\\\hline
		\end{tabularx}
	\end{small}
	\caption{Veranstaltungen in den wahlbereichen für das Studienfach Bachelor Bioinformatik (180~LP). Du musst in jedem Wahlbereich auf 10~Leistungspunkte kommen.\label{plan-bioinfo2}}
\end{table}

Liebe Studienanfängerinnen und Studienanfänger,\par
für die meisten von euch beginnt ein neuer Lebensabschnitt:
Ihr habt euch entschieden, Bioinformatik zu studieren und damit für ein nicht ganz einfaches Studium.
Im Bereich der Informatik wird von euch hohes Abstraktionsvermögen und mathematische Logik verlangt.
In den biowissenschaftlichen Fächern müsst ihr beweisen, dass ihr auch mal viel pauken könnt.
Die Anforderungen an diese interdisziplinäre Fachrichtung sind umfangreich – aber keine Sorge:
Mit ein bisschen Fleiß und genügend Kraft werdet ihr das Studium meistern.

Danach stehen euch viele Optionen zur Verfügung.
Bioinformatiker haben sehr gute Vermittlungschancen, was den Beruf angeht.
Sowohl viele Firmen, als auch wissenschaftliche Institute suchen gut ausgebildete Bioinformatiker.
Der Abschluss ist kein klassischer Abschluss, er ist modern und deswegen kann man sagen, dass ihr Sachen lernt, die ihr später auch so anwenden könnt.
Für viele ist auch ein anschließender Master-Studiengang eine Option.
Mit dem Master in der Tasche erhöhen sich auch die Chancen, später den „Traumberuf“ ausführen zu können.

Ihr werdet in der Regel Informatik- und Mathematikveranstaltungen mit Informatikern besuchen und einige Biologievorlesungen gemeinsam mit Biologen und Biochemikern hören.
Die Chemie- und einige Biologieveranstaltungen sind für euch „zugeschnitten“.
Im späteren Studienverlauf kommen auch interdisziplinäre Veranstaltungen (wie z.B. \textit{Algorithmen auf Sequenzen}) dazu.
Grundsätzlich wird jedes Modul mit einer Klausur oder einer mündlichen Prüfung abgeschlossen und geht gewichtet in die Abschlussbenotung ein.

Die Informatik- und Mathematikveranstaltungen sind vor allem von den wöchent\-lichen „Hausaufgaben“, den Übungen, geprägt.
Diese solltest du auf keinen Fall vernachlässigen, da du sonst weder das nötige Wissen erlangst, um die Klausuren zu bestehen, noch die nötigen Punkte, um das Modul abschließen zu können.
Du musst meistens 50\% (bei manchen Professoren auch 60\%) der Übungsaufgaben richtig gelöst haben, um das Modul zu absolvieren.
Sicherlich wirst du am Anfang Probleme haben die Übungen zu lösen, aber lass dich davon nicht entmutigen!
Es wäre geradezu unmenschlich, wenn du auf Anhieb alles richtig lösen könntest.

Bildet Lerngruppen, um euch den Aufgaben gemeinsam zu stellen und um euch gegenseitig die Lösungen zu erklären.
Nur so könnt ihr auf Dauer die erforderlichen Punkte erreichen.
Wenn ihr euch allerdings nicht mit dem Stoff befasst, sondern nur abschreibt, verursacht das am Ende des Semesters deutlich mehr Arbeit, weil ihr alles nacharbeiten müsst.
Unter Umständen merkt ihr dann auch sehr schnell, dass ihr so viel Rückstand habt, dass ihr den Stoff bis zum Klausurtermin nicht mehr aufholen könnt. Habt ihr jedoch fleißig mitgearbeitet, so müsst ihr für die Prüfung nur nochmal alles wiederholen.
Grundsätzlich hat sich gezeigt:
Wer im Semester die Übungsaufgaben selbst bearbeitet und ohne „Mogelei“ die erforderlichen Punkte erreicht, wird die Prüfung bestehen und das auch nicht zu knapp.

Die Bio- und Chemievorlesungen laufen ganz anders ab.
Hier müsst ihr im Laufe des Semester keine einzige Leistung erbringen.
Erst am Ende kommt die Prüfung, für die ihr dann auch richtig viel lernen müsst.
Es lohnt sich natürlich auch hier schon während des Semesters auf zu passen und nachzuarbeiten.

Es ist immer ratsam, sich für die Vorlesungen das Skript (meist wird es offiziell von den Professoren zur Verfügung gestellt) und ältere Klausuren zu Rate zu ziehen.
Wenn ihr hier noch Fragen habt, dann sprecht auch mal ältere Studenten oder den Fachschaftsrat an.
Die Übungen sind auch eine sehr gute Möglichkeit, die Fragen los zu werden.
Oft werden Übungsserien nur wiederholt und vorgerechnet, aber die Übung ist die beste Möglichkeit schnell zu richtigen Antworten zu kommen.

Wenn ihr noch Fragen oder Probleme habt, dann stehen euch der Studienberater Apl.\ Prof.\ Dr.\ Reinhardt und der Fachschaftsrat jederzeit zur Verfügung.
Es gilt wie immer: Fragen kostet nichts!

\textbf{Ansprechpartner im Fachschaftsrat:}\\
Tugce Kuru, \email{tugce.kuru@fsr-matheinfo.de}


%%%%%%%%%%%%%%%%%%%%%%%%%%% LEHRAMT INFO %%%%%%%%%%%%%%%%%%%%%%%%%

\newpage

\subsection{Informatik Lehramt}
\label{studiengang_infolehramt}

\begin{table}[tbp]
	\begin{small}
		\begin{tabularx}{\textwidth}{|b{.4\textwidth}|X|X|X|}
			\hline
			Modultitel                                                            & LP & Note in Abschlusszeugnis & Studien\-semester \\
			\hline
			\multicolumn{4}{|c|}{\textbf{Pflichtmodule Informatik 70}}\\\hline
			Objektorientierte Programmierung                                      &  5 & Ja   & 1. \\
			Einführung in Rechnerarchitektur                                      &  5 & Ja   & 1. \\
			Mathematische Grundlagen der Informatik und Konzepte der Modellierung & 15 & Nein & 1. und 2. \\
			Datenstrukturen und effiziente Algorithmen I                          &  5 & Ja   & 2. oder 4. \\
			Technische Informatik, Betriebssysteme und Rechnernetze (Lehramt)     &  5 & Nein & 3. \\
			Konzepte der Programmierung                                           &  5 & Nein & 3. \\
			Datenbanken I                                                         & 10 & Ja   & 3. , 5. oder 7. \\
			Automaten und Berechenbarkeit*                                        & 10 & Ja   & 4. oder 6. \\
			Softwaretechnik (Lehramt)                                             &  5 & Ja   & 5. \\
			Informatik und Gesellschaft                                           &  5 & Nein & 5. oder später \\
			\hline
			\multicolumn{4}{|c|}{\textbf{Wahlmodule Informatik 10(5)}}\\\hline
			Algorithmen auf Sequenzen I                      & 5 & Nein & 5. oder später \\
			Datenstrukturen und effiziente Algorithmen II    & 5 & Nein & 5. oder später \\
			Einführung in die Bildverarbeitung               & 5 & Nein & 5. oder später \\
			Einführung in die Computergraphik                & 5 & Nein & 5. oder später \\
			Einführung in die KI                             & 5 & Nein & 5. oder später \\
			Einführung in Rechnernetze und verteilte Systeme & 5 & Nein & 5. oder später \\
			Grundlagen des WWW                               & 5 & Nein & 5. oder später \\
			Komponenten- und serviceorientierte Software     & 5 & Nein & 5. oder später \\
			Theorie der Datensicherheit                      & 5 & Nein & 5. oder später \\
			\hline
			\multicolumn{4}{|c|}{\textbf{Fachdidaktik Informatik 15}}\\\hline
			Didaktik der Informatik AB  & 5 & Ja   & 3. oder 4. \\
			Didaktik der Informatik CDE & 5 & nein & ab 4. \\
			Didaktik der Informatik FG  & 5 & Ja   & ab 5. \\
			\hline
		\end{tabularx}
	\end{small}
	\caption{\label{plan-laI}Regelstudienplan für den Studiengang Informatik LAG bzw. LAS. Das mit * gekennzeichnete Modul sowie die Wahlmodule sind nur für die LAGs zu belegen. Von den Wahlmodulen sind von den LAGs ein bzw. zwei Fächer, falls Informatik das erste Fach ist, zu belegen.}
\end{table}

Liebe Studienanfängerinnen und Studienanfänger,\\
willkommen an der Martin-Luther-Universität Halle-Wittenberg.
Schön, dass ihr euch gerade für diesen Studiengang entschieden habt, denn von euch gibt es hier noch nicht so viele.

Am Besten lest ihr euch die Texte für Informatik und für die Lehrämter der Mathematik durch.

Ansonsten kann euch jetzt noch Eines mit auf den Weg gegeben werden:
Dieses Studium ist kein Leichtes aber ihr werdet euch sicher nach ein paar Wochen hinein gefunden haben.
Sprecht mit euren Kommilitoninnen und Kommilitonen, bildet Gruppen und dann werdet ihr sicher auch Erfolg haben.

Die Pflichtmodule sind bis auf eines dieselben: LAG-Studenten haben zusätzlich das Modul \textit{Automaten und Berechenbarkeit}.

Falls ihr ein Problem haben solltet, wendet euch einfach an den Fachschaftsrat, wir sind für euch da.

\textbf{Ansprechpartner im Fachschaftsrat:}\\
Peter Böttcher, \email{peter.boettcher@fsr-matheinfo.de}\\

\newpage

%%%%%%%%%%%%%%%%%%%%%%%%% MATHEMATIK %%%%%%%%%%%%%%%%%%%%%%%%%%%%%%%%
\subsection{Mathematik}
\label{studiengang_mathematik}

\begin{table}[tbp]
	\begin{small}
		\begin{tabularx}{\textwidth}{|X||c|c|c|c|c|c||r|}
			\hline
			\textbf{Modul}&\multicolumn{6}{l||}{\textbf{Leistungspunkte im Sem.}}&\textbf{LP}\\\hline
			&1&2&3&4&5&6&\\\hline\hline
			Analysis~I+II&9&9&&&&&18\\\hline
			Lineare Algebra~I+II&9&9&&&&&18\\\hline
			Informatik&5&5&&&&&10\\\hline
			Numerik~I+II&&9&9&&&&18\\\hline
			Analysis~III&&&9&&&&9\\\hline
			Algebra&&&9&&&&9\\\hline
			Maßtheorie&&&&8&&&8\\\hline
			Wahrscheinlichkeitstheorie und Statistik&&&&8&&&8\\\hline
			Funktionalanalysis&&&&&8&&8\\\hline
			Aufbaumodul~(Analysis)&&&&8&&&8\\\hline
			Proseminarund Seminar&&&&5&5&&10\\\hline
			Praktikum&&&&6&&&6\\\hline
			Vertiefungsmodul&&&&&15&&15\\\hline
			Anwendungsfach&&&5&5&5&5&20\\\hline
			ASQ&5&&&&5&&10\\\hline\hline
			\textbf{Bachelor-Arbeit}&&&&&&15&15\\\hline
		\end{tabularx}
	\end{small}
	\caption{\label{plan-mathe}Regelstudienplan für den Bachelor Mathematik (180~LP). Die Pflichmodule „Aufbaumodul (Analysis)“ und „Vertiefungsmodul“ sind Platzhalter für jeweils ein passendes Wahlpflichmodul. Brückenmodule sind auch Wahlplichtmodule, die jedoch sowohl im Bachelor- als auch im Masterstudiengang belegt werden können. Dieser Regelstudienplan ist nur eine Empfehlung.}
\end{table}

Liebe Studienanfängerinnen und Studienanfänger,\par

herzlich willkommen an der Martin-Luther-Universität und herzlichen Glückwunsch zu eurer Entscheidung für ein Bachelor-Studium der Mathematik.
Euch erwartet ein vielseitiges und spannendes, aber auch sehr anspruchsvolles Studium, in dem ihr unter anderem ein sehr viel umfassenderes Bild der Mathematik gewinnen werdet, als es der Schulunterricht vermitteln kann und will.
Ihr werdet Mathematik als eine Wissenschaft kennenlernen, die sich mit präziser Denk- und Arbeitsweise gleichermaßen vielen abstrakten Fragestellungen wie auch angewandten Problemen aus den Naturwissenschaften, der Medizin, den Wirtschaftswissenschaften und anderen Bereichen widmet.

Da sich die Mathematik praktisch in sehr vielen Bereichen wiederfindet, ist die berufliche Situation für Absolventen sehr gut.
Die wahrscheinlich bekanntesten Berufsfelder liegen im Bank- und Versicherungswesen.
Eine ebenfalls sehr attraktive Beschäftigungsmöglichkeit ist sicherlich die des wissenschaftlichen Mitarbeiters an der Universität.
Aber auch in anderen Bereichen gibt es viele Möglichkeiten der Beschäftigung mit einem abgeschlossenen Mathematikstudium.

Dieses Studium besteht aus einer soliden mathematischen Ausbildung, die von Studienbeginn an zu selbstständigem Arbeiten anhält. Das in den Vorlesungen vermittelte Wissen wird in den zugehörigen Übungen durch die Bearbeitung von Übungsauf\-gaben gefestigt.
An dieser Stelle legen wir euch ans Herz, die Übungsaufgaben sorgfältig zu bearbeiten. Diese sind nämlich meist Voraussetzung für die Teilnahme an den Prüfungen und dienen allem voran der Festigung des Stoffes, was das Lernen für Klausuren oder für mündliche Prüfungen erleichtert und vor allem verkürzt. 
In den ersten beiden Fachsemestern werden in den Grundlagenvorlesungen \textit{Analysis}, \textit{Lineare Algebra} und \textit{Numerik} unverzichtbare Grundkenntnisse und Methoden der Mathematik erworben, welche eine solide Grundlage für das gesamte Mathematikstudium bilden. Zusätzlich werden Grundkenntnisse in Informatik im Rahmen der Module \textit{Objektorientierte Programmierung} und \textit{Datenstrukturen und Effiziente Algorithmen I}, welche ihr durch die Wahl des Anwendungsfaches Informatik noch erweitern könnt. Weitere wählbare
Anwendungsfächer sind Biologie, Chemie, Physik, und Wirtschaftswissenschaft.

Zu Beginn des Studiums gibt es neben Vorlesungen und Übungen auch noch Tutorien und Workshops.
Studenten aus höheren Semestern wiederholen hier den wichtigsten Stoff der Vorlesung und beantworten eure Fragen dazu. Diese Veranstaltungen sind zwar freiwillig, aber dennoch ein wichtiger Bestandteil der Wissensvermittlung. Sie ersetzen jedoch nicht das Selbststudium daheim. 

Aber auch die Verbindung zur Praxis soll nicht zu kurz kommen. Dazu dient das Praktikum im vierten Semester.
Es sollte in einem wirtschaftlichen Betrieb absolviert werden, kann aber auch in einem anderen Institut der Martin-Luther-Universität stattfinden.
In dem Praktikum sollen typische Studieninhalte des Studienganges zur Anwendung kommen.
Dabei muss es von einer Hochschullehrerin bzw. einem Hochschullehrer des Instituts für Mathematik betreut werden.

\textbf{Ansprechpartner im Fachschaftsrat:}\\
Alexander Klemps, \email{alexander.klemps@fsr-matheinfo.de}\\
Lukas Rettler, \email{lukas.rettler@fsr-matheinfo.de}\\

\newpage

%%%%%%%%%%%%%%%%%%%%%%%%%% WIRTSCHAFTSMATHE %%%%%%%%%%%%%%%%%%%%%%%%%%%%
\subsection{Wirtschaftsmathematik}
\label{studiengang_wima}

\begin{table}[tbp]
	\begin{small}
		\begin{tabularx}{\textwidth}{|X||c|c|c|c|c|c||r|}\hline
			\textbf{Modul}&\multicolumn{6}{l||}{\textbf{Leistungspunkte im Sem.}}&\textbf{LP}\\\hline
			&1&2&3&4&5&6&\\\hline\hline
			\multicolumn{8}{|c|}{\textbf{Pflichtmodule}}\\\hline
			Analysis~I+II&9&9&&&&&18\\\hline
			Lineare Algebra~I+II&9&9&&&&&18\\\hline
			Informatik&10&5&&&&&15\\\hline
			ASQ&&&&&5&5&10\\\hline
			Optimierung (2-semestrig)&&20&&&&&20\\\hline
			Analysis III&&&9&&&&9\\\hline
			Maßtheorie&&&&8&&&8\\\hline
			Numerik für Wirtschaftsmathematiker&&&&8&&&8\\\hline
			Wahrscheinlichkeitstheorie und Statistik&&&&8&&&8\\\hline
			Proseminar und Seminar&&&&5&5&&10\\\hline
			Versicherungsmathematik und Risikotheorie&&&&&8&&8\\\hline
			Vertiefungsmodul Mathematik&&&&&5&&5\\\hline
			Wirtschaftswissenschaftsmodul&&&10&5&5&5&25\\\hline
			Praktikum&&&&8&&&8\\\hline\hline
			\textbf{Bachelor-Arbeit}&&&&&&15&15\\\hline\hline
			\multicolumn{8}{|c|}{\textbf{Wirtschaftswissenschaftsmodule}}\\\hline
			\multicolumn{7}{|X||}{Grundlagen der BWL}&5\\\hline
			\multicolumn{7}{|X||}{Grundlagen der VWL}&5\\\hline
			\multicolumn{7}{|X||}{Mikroökonomik I+II}&je 5\\\hline
			\multicolumn{7}{|X||}{Makroökonomik I+II}&je 5\\\hline
			\multicolumn{7}{|X||}{Wertschöpfungsmanagement}&5\\\hline
			\multicolumn{7}{|X||}{Internes Rechnungswesen}&5\\\hline
			\multicolumn{7}{|X||}{Produktion und Logistik}&5\\\hline
			\multicolumn{7}{|X||}{Investition und Finanzierung}&5\\\hline
			\multicolumn{7}{|X||}{Entscheidungs- und Spieltheorie}&5\\\hline
		\end{tabularx}
	\end{small}
	\caption{\label{plan-wima}Regelstudienplan für den Bachelor Wirtschaftsmathematik (180~LP). Das Pflichtmodul „Vertiefungsmodul“ ist ein Platzhalter für ein Vertiefungsmodul, die auf Seite \pageref{studiengang_mathematik} aufgelistet sind. Das Pflichtmodul „Wirtschaftswissenschaftsmodul“ ist ein Platzhalter für eins der im unteren Teil der Tabelle aufgeführten Wirtschaftswissenschaftsmodule. Dieser Regelstudienplan ist nur eine Empfehlung.}
\end{table}

Liebe Studienanfängerinnen, liebe Studienanfänger,\par

herzlich willkommen an der Martin-Luther-Universität und herzlichen Glückwunsch zu eurer Entscheidung für ein Bachelor-Studium der Wirtschaftsmathematik.

Mit diesem Studium soll der Spagat zwischen Mathematik und Wirtschaft gelingen.
Das heißt neben der Beschäftigung mit der Mathematik (vgl. Informationen für Bachelor Mathematik auf Seite~\pageref{studiengang_mathematik}) stehen noch Betriebs- und Volkswirtschaftlehre auf dem Stundenplan.
Hier sind während des gesamten Studiums fünf Fächer zu belegen.
In den ersten beiden Semestern bieten sich z.B. \textit{Grundlagen der BWL} oder \textit{Grundlagen der VWL} und \textit{Mikroökonomik~I} an.
Weitere Möglichkeiten findet ihr in der Prüfungsordnung.

Wirtschaft wirkt am Anfang meist sehr einfach, dadurch nimmt man es gern auf die leichte Schulter.
Die Klausuren, für die man sich rechtzeitig im Löwenportal anmelden muss, sind aber nicht immer ein Kinderspiel, deswegen versucht kontinuierlich den Stoff zu lernen und auch mit anderen (auch VWL- und BWL-Studenten) zusammen zu arbeiten.

Diese Gruppenarbeit ist in der Mathematik das wichtigste, weil der Stoff am Anfang gar nicht einfach, sondern oft unverständlich wirkt.
Hier sind die Übungen das Wichtigste zum Verstehen der mathematischen Inhalte.
In den ersten beiden Semestern hört ihr Analysis I und II, Lineare Algebra I und II, sowie Optimierung.
Es ist in den Mathematik-Modulen immer sehr wichtig, die Übungsaufgaben zu bearbeiten, die euch beim Verstehen des Stoffes helfen und außerdem oft Voraussetzung sind, um an den Prüfungen teilzunehmen.
Neben den beiden Bestandteilen Wirtschaft und Mathematik müsst ihr euch auch noch mit Informatik „herumärgern“.

Wir hoffen, ihr beißt euch durch die ersten Semester, denn das sind die Schwierigsten.
Wir mussten uns auch erst an die neue Denk- und Arbeitsweise gewöhnen.
Also gebt nicht auf!
Wenn ihr Hilfe braucht, könnt ihr euch an uns oder euren Studienberater, Dr.~Rackwitz, wenden.

\textbf{Ansprechpartner im Fachschaftsrat:}\\
Alexander Klemps, \email{alexander.klemps@fsr-matheinfo.de}\\

\newpage

%%%%%%%%%%%%%%%%%%%%%%%%%%%%% MATHE LAG %%%%%%%%%%%%%%%%%%%%%%%%%%%%%

\subsection{Mathematik Lehramt}
\label{studiengang_lehramt}

\subsubsection{LAG Mathematik}
\label{studiengang_lag}

\begin{table}[tbp]
	\begin{small}
		\begin{tabularx}{\textwidth}{|@{~}X@{~}|@{~}c@{~}|@{~}X@{~}|@{~}X@{~}|@{~}c@{~}|@{~}c@{~}|}
			\hline
			Modul & Semes\-ter & Voraus\-setzungen & Modul\-leistung & \begin{sideways}Anteil an Abschlussnote\end{sideways} & \begin{sideways}Leistungspunkte\end{sideways}\\\hline\hline
			Analysis~I und II &   1.-2.  &                &   mdl. Prf.   &   ja    &   15\\\hline
			Lineare Algebra &  1.-2.   &         &   mdl. Prf.   &   nein    &   15\\\hline
			Wahr\-schein\-lich\-keits\-theorie und Sta\-tis\-tik&4.&Analysis~I+II&mdl. Prf.&ja&6\\\hline
			Pro\-seminar&4.&(Analy\-sis~I+II), (Lineare Algebra)&Vortrags\-aus\-arbeitung&nein&5\\\hline
			Grundl. der numerischen Mathe\-matik&ab 3.&(Analy\-sis~I), (Lineare Algebra)&Klausur&ja&5\\\hline
			Algebra&ab 3.&(Analy\-sis~I+II), (Lineare Algebra)&Klausur&ja&7\\\hline
			Fach\-seminar&5.&&Vortrags\-aus\-arbeitung&nein&5\\\hline
			Ver\-tiefungs\-modul (nur wenn Erst\-fach)&ab 3.&(Analysis~I+II), (Lineare Algebra)&Klausur od. mdl. Prf.&nein&5\\\hline
			Mathe\-matik\-didaktik AB&3.-4.&&Beleg\-arbeit oder Klausur&ja&5\\\hline
			Mathe\-matik\-didaktik CDE&4.-5.&(Mathe\-matikdid.~AB)&Beleg\-arbeit&nein&5\\\hline
			Mathe\-matik\-didaktik FG&6.-8.&Mathe\-matik\-did.~AB+CDE&mdl. Prf.&ja&5\\\hline\hline
			\multicolumn{6}{|c|}{Geometrie*}\\\hline
			%            Geometrie/ Differential\-geometrie&5.od. 7.&Analysis~I+II, Lineare Algebra&Klausur oder mdl. Prf.&6&ja&ja&7\\\hline\hline
			Geometrie&5.od. 7.&(Analysis~I+II), Lineare Algebra&Klausur oder mdl. Prf.&ja&7\\\hline
			Differential\-geometrie&5.od. 7.&Analysis~I+II, Lineare Algebra&Klausur oder mdl. Prf.&ja&7\\\hline\hline
			\multicolumn{6}{|c|}{Grundlagen*}\\\hline
			%            Geschichte der Mathe\-matik/Grund\-lagen der Mathe\-matik&ab 4.&Analysis~I+II, Lineare Algebra&Beleg\-arbeit&ja&5\\\hline\hline
			Geschichte der Mathe\-matik&ab 4.&(Analysis~I), (Lineare Algebra)&Beleg\-arbeit&ja&5\\\hline
			Grund\-lagen der Mathe\-matik&ab 4.&(Analysis~I+II), (Lineare Algebra)&Beleg\-arbeit&ja&5\\\hline\hline
			\multicolumn{6}{|c|}{Analysis/Numerik*}\\\hline
			%            Funktionen\-theorie/gewöhnl. Differentialgl./Theorie u. Num. gewöhnl. Dgl.&ab 5.&&Klausur&ja&5\\\hline\hline
			Funktionen\-theorie&ab 5.&(Analysis~I+II), (Lineare Algebra)&Klausur od. mdl. Prf.&ja&5\\\hline
			Gewöhnl. Differentialgl.&ab 5.&(Analysis~I+II), (Lineare Algebra)&Klausur od. mdl. Prf.&ja&5\\\hline
			Theorie u. Num. gewöhnl. Dgl.&ab 5.&&Klausur&ja&5\\\hline
		\end{tabularx}
	\end{small}
	\caption{\label{plan-lag}Regelstudienplan für den Studiengang
		Mathematik -- Lehramt an Gymnasien. Bei den mit * gekennzeichneten
		Teilgebieten muss jeweils nur eine Veranstaltung besucht werden. Dies
		ist nur eine Empfehlung.}
\end{table}

\subsubsection{LAS Mathematik}
\label{studiengang_las}

\begin{table}[tbp]
	\begin{small}
		\begin{tabularx}{\textwidth}{|@{~}X@{~}|@{~}c@{~}|@{~}X@{~}|@{~}X@{~}|@{~}c@{~}|@{~}c@{~}|}
			\hline
			Modul & Semes\-ter & Voraus\-setzungen & Modul\-leistung & \begin{sideways}Anteil an Abschluss\-note \end{sideways}& \begin{sideways}Leistungspunkte\end{sideways}\\\hline\hline
			
			Lineare Algebra &1.-2.&&mdl. Prf.&nein&15\\\hline
			Elemente der Mathematik &1.-2.&&Klausur&nein&5\\\hline
			Analysis~I &3.&&mdl. Prf.&ja&10\\\hline
			Elemente der Kombinatorik und Stochastik &5.&Elemente der Mathematik&Klausur&ja&5\\\hline
			Elemente der Geometrie &3. od. 5.&Elemente der Mathematik&mdl. Prf.&ja&5\\\hline
			Proseminar & ab 4. & & Vortrags\-ausarbeitung &nein&5\\\hline
			Algebra & 5. &Analysis~I, Lineare Algebra & Klausur & ja & 5\\\hline
			
			Fachseminar & 5./6. & & Vortrags\-ausarbeitung & nein & 5\\\hline
			Ver\-tiefungs\-modul (nur wenn Erst\-fach)&ab 4.&&&nein&5\\\hline
			Mathe\-matik\-didaktik AB&3.-4.&&Beleg\-arbeit oder Klausur&ja&5\\\hline
			Mathe\-matik\-didaktik CDE&4.-5.&Mathe\-matikdid. AB&Beleg\-arbeit&nein&5\\\hline
			Mathe\-matik\-didaktik FG&6.-8.&Mathe\-matikdid. CDE&mdl. Prf.&ja&5\\\hline\hline
			
			\multicolumn{6}{|c|}{Mathematik*}\\\hline
			Analysis~II & ab 4. & Analysis~I & mdl. Prf. & ja & 5\\\hline
			Geschichte der Mathematik & ab 4. & Analysis~I, Lineare Algebra & Belegarbeit & ja & 5\\\hline
			Grundlagen der numerischen Mathematik & ab 5. & Analysis~I, Lineare Algebra & Klausur & ja & 5\\\hline
			Mathematische Biologie & ab 4. & & Klausur & ja & 5\\\hline
			Funktionentheorie & ab 5. & & mdl. Prf. & ja & 5\\\hline
			Geometrie & ab 5. & Analysis~I, Lineare Algebra & Klausur od. mdl. Prf.& ja & 5\\\hline
			Diskrete Mathematik & ab 5. & Analysis~I, Lineare Algebra & Klausur od. mdl. Prf.& ja & 5\\\hline
		\end{tabularx}
	\end{small}
	\caption{\label{plan-las}Regelstudienplan für den Studiengang Mathematik -- Lehramt an Sekundarschulen. Bei den mit * gekennzeichneten Teilgebiet muss jeweils nur eine Veranstaltung besucht werden. Dies ist nur eine Empfehlung.}
\end{table}

Ihr wollt also Lehrer werden?!\\
Gute Wahl! Denn Mathe-Lehrer sind mehr gesucht als eh und je! 
Hier vorab ein paar Infos, worauf ihr euch einlasst: Auch wenn ihr in der Schule nicht zu den Einserkandidaten in Mathematik gehört habt, habt ihr durchaus reale Chancen dieses komplexe Studium erfolgreich zu bestreiten.
Es kommt nur auf Eines an: Durchbeißen.
Und das heißt in den ersten zwei Semestern am Ball bleiben, wenn ihr mit den Bachelor-Mathematikern zusammen \textit{Analysis} und \textit{Lineare Algebra} hört.
Und auch wenn es schwer fällt:
Aufstehen, zur Vorlesung gehen, fleißig Übungsaufgaben lösen und auf jeden Fall mit anderen Leuten zusammentun, damit die Verzweiflung mit anderen geteilt werden kann.

Und nein: Ihr seid nicht blöd.
Man kann als normalsterbliches Wesen (fast) nie alle Aufgaben fehlerfrei lösen, aber dafür gibt es auch die 50\% (bei manchen Profs auch 60\%) Hürde ($\rightarrow$  genaueres steht in den Modulbeschreibungen bzw. wird zu Beginn des Semesters bekannt gegeben).
Wenn ihr euch regelmäßig mit den Aufgaben beschäftigt und sie auch wirklich verstanden habt, sollte es auch gut machbar sein, die abschließenden Klausuren zu bestehen.

Lasst euch aber nicht allzu schnell entmutigen.
Rauft euch zusammen.
Löst die Aufgaben zusammen.
Erklärt’s euch gegenseitig.
Schließlich werdet ihr ja Lehrer.

Leider ist das noch nicht alles, denn neben der Mathematik studiert ihr ja noch ein anderes Fach (manche sogar noch zwei) und das erziehungswissenschaftliche Begleitstudium mit den Fächern Pädagogik und Psychologie muss auch noch belegt werden.
Da ist euer Organisationstalent gefragt.
Informiert euch so früh wie möglich über die Veranstaltungszeiten (\link{http://studip.uni-halle.de} oder Aushänge) und sprecht bei Überschneidungen die entsprechenden Dozenten an.
Manchmal kann da noch geschoben werden, machmal aber auch nicht.
Dann versucht vielleicht schon einmal eine andere Veranstaltung zu besuchen, wenn ihr die Voraussetzungen dafür schon habt.

Für Psychologie und Pädagogik ist zusätzlich zu beachten, dass die Einschreibungen momentan online im Stud.IP erfolgen und nur eine begrenzte Anzahl von Seminarplätzen verfügbar ist (Einschreibungen dafür enden schon sehr zeitig.).
Versucht dann lieber Vorlesungen, für die es oft keine Begrenzungen gibt, im 1.~Semester zu besuchen, wenn ihr dafür noch Zeit habt.

Für jedes Modul müsst ihr Modulvorleistungen und Studienvorleistungen erbringen. Modulleistungen müsst ihr einfach nur erfüllen, um eure LP zu erhalten. Eure Prüfungen könnt ihr auch ablegen, wenn ihr die Modulleistung nicht erfüllt habt. Ihr bekommt die LP allerdings erst, wenn auch die Modulleistung erfüllt ist. Modulleistungen können z.B. das wöchentliche Abgeben von Übungsserien sein oder das Bestehen von Klausuren.
Studienvorleistungen sind dafür wichtig, um zur Prüfung zugelassen zu werden. An sich können die Studienvorleistungen ähnlich zu den Modulvorleistungen sein. Was genau ihr für das Jeweilige Modul zu erfüllen habt, erklären euch die Professor*innen am Anfang des Semesters.
Für all diejenigen von euch, die BAföG erhalten ist es wichtig euer Studium so zu gestalten, dass ihr am Ende des 4.~Semesters auf 105~Leistungspunkte kommt.
Davon dürfen maximal 30 in Form von Prüfungen angemeldet sein, der Rest muss schon bestanden sein.
Dabei ist es egal, woher diese stammen, sei es euer zweites Fach oder aus Pädagogik und Psychologie.

Bei Problemen und Fragen stehen euch neben den Vertretern des Fachschaftsrates auch sehr gerne Frau Prof. Richter zur Verfügung.
Sie ist für die Mathematik-Didaktik zuständig.


\textbf{Ansprechpartner im Fachschaftsrat:}\\
Sarah Henkel, \email{sarah.henkel@fsr-matheinfo.de}\\
Anna Hemmann, \email{anna.hemmann@fsr-matheinfo.de}




\label{studiengang_lehramt}

\label{studiengang_lag}

\label{studiengang_las}




\newpage

\section{Wer vertritt wen?}
\label{vertretung}

Es gibt zwei Arten von Gremien, die im Folgenden vorgestellt werden: Einerseits gibt es studentische Gremien (Fachschaftsräte, Studierendenrat), in denen nur Studenten vertreten sind. Andererseits gibt es universitäre Gremien (Fakultätsräte, Senat), in denen Professoren, wissenschaftliche Mitarbeiter, nichtwissenschaftliche Mitarbeiter und Studierende gemeinsam vertreten sind.
%Eine Übersicht über die Gremien ist in Abbildung \ref{fig:gremien} auf Seite \pageref{fig:gremien} gegeben.

\begin{figure}[htb]
    \centering
    %Hier kommt eine Tikz Picture hin und kein odg Ding zwecks kompatibilität und Sexyness
    %\includegraphics[width=\textwidth]{Bilder/studentische_gremien.odg}\\
    \begin{tikzpicture}
    \node[draw] (Senatoren) at (0,0) {4 stud. Senatoren};
    \node[draw] (Fakultaetsmitglieder) at (3.5,0) {stud. Fakultätsmitglieder};
    \node[draw] (StuRa) at (7,0) {StuRamitglied(er)};
    \node[draw] (FSR) at (9.7,0) {Fachschaftsrat};
    \node[draw] (Studis) at (1,-2) {Studierendenschaft};
    \node[draw] (FS) at (6.5,-2) {17 Fachschaften};
	%
  	\path (Studis) -- (FS) node[midway, above] {gliedert sich in};
  	\draw[->,draw] (Studis) to (FS);
  	%
  	\path (Studis) -- (Senatoren) node[midway, right] {wählt};
  	\draw[->,draw] (Studis) to (Senatoren);
  	%
  	\path (FS) -- (Fakultaetsmitglieder) node[midway, right] {wählt};
  	\draw[->,draw] (FS) to (Fakultaetsmitglieder);
  	%
  	\path (FS) -- (StuRa) node[midway, right] {wählt};
  	\draw[->,draw] (FS) to (StuRa);
  	%
  	\path (FS) -- (FSR) node[midway, right] {wählt};
  	\draw[->,draw] (FS) to (FSR);
    \end{tikzpicture}
    \caption{Übersicht über die studentischen Gremien}
    \label{fig:gremien}
\end{figure}

\subsection{Fachschaftsräte}

\begin{figure}[h]
    \centering
    \includegraphics[scale=0.36]{fsrsw}
    \caption{hinten: Maximilian Büttner, Norman Holtz, René-Pierre Geiß, Florian Lücke vorne: Steffen Manigk, Florian Johnke, Felix Schmidt}
             \label{fig:fsr}
\end{figure}

\begin{table}[!h]
    \centering
    \begin{tabular}{llll}
        \toprule
        \textbf{Sprecher} & Florian Lücke & Master Informatik & 3.~Sem.\\
                 & \multicolumn{3}{l}{\email{florian.luecke@fsr-matheinfo.de}}\\
        \midrule
        \textbf{Stellvertreter} & Florian Johnke & Lehramt Mathematik, & 3.~Sem.\\
                                && Lehramt Informatik & 3.~Sem.\\
                       & \multicolumn{3}{l}{\email{florian.johnke@fsr-matheinfo.de}}\\
        \midrule
        \textbf{Finanzer~I} & Steffen Manigk & Lehramt Mathematik  & 7.~Sem.\\
                   & \multicolumn{3}{l}{\email{steffen.manigk@fsr-matheinfo.de}}\\
        \midrule
        \textbf{Finanzer~II}  & Norman Holtz & Bachelor Wirtschafts- & 5.~Sem.\\
                              && mathematik\\
                     & \multicolumn{3}{l}{\email{norman.holtz@fsr-matheinfo.de}}\\
        \midrule
                     & Felix schmidt & Master Informatik & 3.~Sem.\\
                     & \multicolumn{3}{l}{\email{felix.schmidt@fsr-matheinfo.de}}\\
        \midrule
                     & Maximilian Büttner & Bachelor Mathematik & 5.~Sem.\\
                     & \multicolumn{3}{l}{\email{maximilian.buettner@fsr-matheinfo.de}}\\
        \midrule
                     & René-Pierre Geiß & Bachelor Mathematik, & 3.~Sem.\\
                     && Bachelor Bioinformatik & 5.~Sem.\\
                     & \multicolumn{3}{l}{\email{rene.geiss@fsr-matheinfo.de}}\\
        \bottomrule
    \end{tabular}
    \caption{Mitglieder des Fachschaftsrats 2014/15}
\end{table}
\newpage
Der Fachschaftsrat ist die Interessenvertretung der Studentinnen und Studenten eines Fachbereichs und besteht in unserem Fall aus sieben Mitgliedern, welche jedes Jahr im Mai neu gewählt werden.
Er ist unter anderem dazu da, den Kontakt zu den Professoren und Mitarbeitern zu halten.
Wenn also Fragen und Probleme jeglicher Art auftreten, habt keine Scheu, sondern wendet euch an uns.
Wir versuchen dann gemeinsam eine Lösung zu finden.

Außerdem organisiert der Fachschaftsrat Feiern, damit bei dem vielen Stress des Studiums der Spaß nicht verloren geht.
Da wären unter anderem das alljährliche Sommerfest, die Weihnachtsfeier und die Erstsemesterparty.
Diese Veranstaltungen sind für die Mitglieder der Fachschaft -- also für euch.
Eintrittspreise habt ihr also nicht zu erwarten und Getränke und Grillgut sind sehr günstig.
Ihr findet detaillierte Infos weiter hinten im Heft.

Der Fachschaftsrat finanziert sich aus einem Teil der \EUR{6,10}, die ihr jedes Semester als Studierendenbeitrag bezahlt.
Schließlich ist der Fachschaftsrat ein Teil des Studierendenrates.
Wenn ihr eine Idee haben solltet, was man für die Fachschaft tun oder kaufen könnte, so versuchen wir es gerne zu finanzieren.
Gleiches gilt für besondere Aufgaben, die eigentlich nicht zum Studium gehören (\;z.B. Programmierwettbewerbe, Seminare, \ldots).

Wenn ihr später einmal selbst den Studierenden des Fachbereichs helfen wollt, lasst euch einfach bei der nächsten Wahl im Frühling aufstellen.

\web{http://fachschaft.mathinf.uni-halle.de}\\
\email{fachschaft@mathinf.uni-halle.de}

\subsection{Studierendenrat}

Ähnlich wie der Fachschaftsrat (FSR) die Interessen einer Fachschaft vertritt, setzt sich der Studierendenrat (StuRa) für die Belange der gesamten Studierendenschaft ein.
Die Aufgaben der beiden Gremien sind dabei verschieden.
Im Gegensatz zum FSR ist der StuRa für universitätsweite und allgemeine Angelegenheiten zuständig.
Zum Einen diskutiert der StuRa über hochschulpolitische Themen,  vertritt die Studierenden gegenüber der Universität und Land und verwaltet die Studierendenschaftsgelder, zum Anderen bietet er verschiedene Service-Leistungen an, unterstützt die verschiedensten Projekte finanziell (jeder kann einen solchen Projektantrag stellen) und bietet eine kostenlose Rechtsberatung, die bei Rechtsschwierigkeiten von jedem Studierenden in Anspruch genommen werden kann, Sozialberatung und Bafög-Beratung an.
Sogar zinslose Sozialdarlehen kann der Studierendenrat bewilligen.

Der Studierendenrat trifft sich in der Vorlesungszeit in der Regel zweiwöchentlich.
Bei diesen Sitzungen wird viel diskutiert und über verschiedenste Anträge abgestimmt, jeder kann dabei sein und Anträge stellen.
Zur Organisation der anfallenden Arbeit wählt der StuRa Sprecher:

\begin{itemize}
    \item Vorsitzende des Sprecherkollegiums, die die Arbeit des StuRa koordinieren und den StuRa nach Außen vertreten.
    \item Sitzungsleitende Sprecher, die die Sitzungen des StuRa organisieren, an denen Anträge an den StuRa eingereicht werden und die dazu Sprechstunden anbieten.
    \item Sprecher für Finanzen, die sich um korrekte Abrechnungen und den Haushaltsplan kümmern.
    \item Sprecher für Soziales, die Sozialanträge behandeln, und bei Fragen rund um BAföG, Wohngeld, GEZ, Stipendien, Krankenkasse, Versicherung, Studieren mit Kind usw. euch zur Seite stehen,
    \item einen Senatssprecher (siehe \ref{senat} Senat)
    \item einen Fachschaftskoordinator, der die Koordination und den Austausch unter den Fachschaftsräten leitet
\end{itemize}

Alle Sprecher bilden gemeinsam das Sprecherkollegium.
Daneben gibt es diverse Ausschüsse und Arbeitskreise (z.B. den AK Studierende mit Kind oder die Interessenvertretung Lehramt).
In Arbeitskreisen kann jeder Studierende mitwirken bzw. solche gründen.
Eine aktuelle Auflistung aller Arbeitskreise befindet sich auf der Website des StuRa.
Die Mitglieder des Studierendenrates werden - wie der FSR - von den Studierenden, also euch, direkt gewählt.
Dabei kandidiert ein Studierender immer in seiner jeweiligen Fachschaft, also nicht auf universitätsweiten Listen.
Die Fachschaft Mathematik/Informatik ist eine sehr kleine.
Daher haben wir auch nur eines der 35 Mandate im StuRa.


\web{www.stura.uni-halle.de}\\
\email{stura@uni-halle.de}\\

\subsection{Fakultätsräte}

Die Fakultätsräte setzen sich aus Mitgliedern der Gruppe der Hochschullehrer, Vertretern der Studierenden, des akademischen Mittelbaus sowie der technischen Angestellten zusammen.
Die Institute Mathematik und Informatik können jeweils einen studentischen Vertreter in ihren Fakultätsrat schicken.
Dabei gehört die Mathematik in die Naturwissenschaftliche Fakultät II und die Informatik in die Naturwissenschaftliche Fakultät III.
Aufgabe ist die Entscheidung über die Mittelverwendung und über Fragen der Forschung und Lehre an der jeweiligen Fakultät.
Dazu gehört u.a. die Beschlussfassung der Studien- und Prüfungsordnung.
Außerdem ist der Fakultätsrat für die Erteilung akademischer Grade und Titel zuständig und auch für die Wahl des Vertreters der Fakultät, den Dekan.
Somit sind die Entscheidungen, die dort getroffen werden, für die Studenten eher von mittel- und langfristiger Bedeutung.

Jedoch sind Vorschläge, wie das Studium aus eurer Sicht hier zu verbessern ist, sehr willkommen.
Wir als studentische Vertreter werden auf jeden Fall von unserem Stimm- und Rederecht Gebrauch machen, falls ihr Vorschläge habt.

Ihr könnt euch mit jedem Problem an den Fachschaftsrat wenden, der eure Anliegen gegebenenfalls natürlich auch in den Fakultätsrat weiterleitet.

\web{http://www.natfak3.uni-halle.de/fakultaetsrat/}\\
\email{fachschaft@mathinf.uni-halle.de}\\


\subsection{Senat}
\label{senat}

Der Senat ist das zentrale Beschlussfassungsorgan der Hochschule.
Er bestimmt den Unihaushalt, beschließt Richtungsentscheidungen der Uni, beschließt die Einrichtung und Schließung von Studiengängen, beschließt die Berufungen von Professoren und muss die von den Fakultätsräten beschlossenen Prüfungsordnungen u.Ä. genehmigen.
Stimmberechtigte Mitglieder im Senat sind 12 Professorinnen bzw. Professoren, vier Studierende, vier wissenschaftliche MitarbeiterInnen und zwei sonstige MitarbeiterInnen.
Die vier studentischen Vertreter könnt ihr bei den Hochschulwahlen auf uniweiten Listen wählen.
Die Dekane sind beratende Mitglieder, ebenso wie der durch den Studierendenrat gewählte Senatssprecher und der Kanzler.

Der Senat bildet Kommissionen zur Vorbereitung und Beratung von Entscheidungen.
Zurzeit gibt es vier Kommissionen, die sich verschiedenen Themenbereichen widmen:

\begin{itemize}
    \item Studium und Lehre
    \item Forschung
    \item Haushalt und Struktur
    \item Berufungsprüfungskommission
\end{itemize}

In jeder Kommission wirken bis zu 3 Studierende mit.
Sie werden von den studentischen Senatoren im Senat vorgeschlagen und vom Senat bestätigt.
In der Regel schreibt der Studierendenrat zu Beginn des Studienjahres die Mitarbeit in den Kommissionen aus.
Dann kann sich jeder Studierende melden und wird zu einem Gespräch in den Studierendenrat eingeladen.

\link{http://verwaltung.uni-halle.de/senat\_der\_verwaltung/}


\newpage

\section{Studieren in Halle}

\subsection{Was kann mein Studentenausweis?}
Die Uni Service Card (USC) besitzt einen kontaktlosen, äußerlich nicht sichtbaren Chip und einen wiederbeschreibbaren Streifen für visuell lesbare Aufdrucke.
Damit kann sie mehrere Funktionen erfüllen. Sie dient...

\begin{itemize}
 \item als Studierendenausweis.
       Dazu muss sie den Gültigkeitsaufdruck für das jeweilige Semester tragen.
 \item als Mensakarte.
       Alle Hinweise dazu findet ihr auf dem Faltblatt des Studentenwerks, das ihr zusammen mit der USC erhalten haben solltet
       oder unter \link{http://www.studentenwerk-halle.de/hochschulgastronomie/}.
 \item als Kopierkarte.
 \item als Bibliothekskarte.
       Dazu muss man sich jedoch einmalig in der Hauptbibliothek anmelden.
       Dann kann sie anstelle der bisherigen Bibliothekskarte benutzt werden.
       Wenn ihr eine alte Bibliothekskarte habt, werdet ihr sie bei eurem nächsten Bibliotheksbesuch abgeben müssen.
 \item als MDV Vollticket.
       Im Studierendenausweis ist seit dem Wintersemester 14/15 eine Fahrkarte für Bus und Bahn im Raum Halle enthalten. Nähere Informationen findet ihr unter: \link{http://semesterticket-halle.de/}.
 \item als Mitgliedskarte für die Studierendenschaft.
       Das ist für Wahlen von Bedeutung.
       Auch dafür erhält die USC einen Aufdruck.
 \item als Zugangsmedium für einige Universitätsgebäude.
\end{itemize}

Für alle Bezahlfunktionen ist in der USC ein Chip integriert, für das ein Guthaben geführt wird.
Dieses könnt ihr an den Aufwertern des Studentenwerkes aufladen.

\clearpage %schick machen

\paragraph{Was es bei der USC zu beachten gibt}

Zu Beginn jedes Semesters muss die USC validiert werden. Das könnt ihr nach der Rückmeldung fürs neue Semester jederzeit tun.
Dabei wird der Chip für das neue Semester freigeschaltet und die USC erhält die bereits erwähnten Aufdrucke.
Diese Validierung müsst ihr an einer der Validierungsstationen selbst vornehmen.
In unserer Universität gibt es vier Validierungsstationen, die an folgenden Orten stehen:

\begin{itemize}
    \item im Löwengebäude, Raum~1 (Inforaum)
    \item im Juridicum, Eingangshalle
    \item im IT-Servicezentrum, Kurt-Mothes-Str.~1, 1.~Etage
    \item in der Heidemensa und in der Weinbergmensa, im Erdgeschoss
\end{itemize}

Wichtig ist es, dass ihr eure USC pfleglich behandelt, um sie funktionsfähig zu erhalten.
Ihr dürft sie nicht knicken, nicht extremen Temperaturen oder starkem Druck aussetzen und nicht in stärkere Magnetfelder bringen.

Wenn ihr Probleme mit eurer USC habt, dann wendet euch an das Immatrikulationsamt.
Die erste USC erhaltet ihr gratis.
Wenn ihr sie funktionsunfähig macht oder sie euch abhanden kommt, erhaltet ihr gegen eine Gebühr eine neue USC.

\paragraph{Studentenkarte mit erweiterten Funktionen}

Seit Oktober 2010 führt das Institut für Informatik einen Testbetrieb mit neuen Studentenkarten durch.
Diese neuen Karten verfügen über einen weiteren Chip mit dem man unter anderem Authentifizierungsdienste nutzen kann.
Das heißt für den Login, in den Computer-Pools der Informatik benötigt man nur noch die Karte mit PIN und kein schwierig zu merkendes Passwort mehr.
Auch für das Stud.IP existiert diese Art der Anmeldung schon.
Im Löwenportal soll man künftig mit der Karte auch Prüfungen anmelden können ohne weitere Eingabe von TANs.
Weiterhin kann man mit den Karten kryptographische Funktionen nutzen.
So kann man z.B. verschlüsselte Emails empfangen oder Emails signieren -- also unterschreiben.

Wer eine solche Karte haben möchte (dies gilt für alle Studenten), kann diese bei Dr. Sandro Wefel im Raum 2.16 beantragen.
Weitere Informationen zum neuen Studentenausweis stellt das IT-Servicezentrum (ITZ) bereit:\\[1.0em]
    \link{http://itz.uni-halle.de/dienstleistungen/zertifizierungsbetrieb/}


\subsection{Was macht das Studentenwerk?}

Als sozialer Service für Studierende ist das Studentenwerk Halle für folgende Hochschulen verantwortlich:

\begin{itemize}
    \item Martin-Luther-Universität Halle-Wittenberg
    \item Burg Giebichenstein Hochschule für Kunst und Design
    \item Hochschule Anhalt~(FH) mit Standorten in Köthen, Bernburg und Dessau
    \item Hochschule Merseburg~(FH)
\end{itemize}

Der Verwaltungsrat des Studentenwerkes beeinflusst und kontrolliert die wirtschaftliche Entwicklung und das Handeln der Geschäftsführung.
Das Gremium setzt sich aus studentischen und nichtstudentischen Vertretern zusammen, die durch die Senate der Hochschulen alle zwei Jahre legitimiert werden.

\paragraph{BAföG}

Das Studentenwerk Halle ist als Amt für Ausbildungsförderung mit der Durchführung des Bundesausbildungsförderungsgesetzes (BAföG) beauftragt.
Um innerhalb der gesetzlichen Vorgaben umfassender und persönlicher informieren zu können, stehen dir die Mitarbeiter des Amtes für Ausbildungsförderung zu individuellen Beratungsgesprächen zur Verfügung.
Ein kurzer Gang zum BAFöG-Amt spart euch \;u.U. sehr viel Zeit beim Ausfüllen der Formulare.

\paragraph{Verpflegung}
Ein vollwertiges Essen: schmackhaft, appetitlich, preiswert -- das Studentenwerk bietet in den Verpflegungseinrichtungen eine Auswahl unterschiedlicher Speisekomponenten.
Du kannst dir dein Komplettmenü nach deinen Wünschen selbst gestalten, das du mit deiner USC oder (nicht mehr in allen Mensen) mit Bargeld bezahlen kannst.

\paragraph{Wohnen}
Dem bundesweiten Trend studentischer Wohnformen folgend, kann dir das Studentenwerk vom Einzelzimmer bis zum Appartement Wohnraum anbieten und
versucht, deine veränderten Wohnbedürfnisse zu realisieren und je nach Sanierungsgrad die Mieten sozialverträglich zu gestalten.

\paragraph{Beratung}
Das Studentenwerk berät dich.
Es versucht, mit dir eine Lösung aus einer vermeintlichen Auswegslosigkeit zu finden und hilft dir bei der Vermittlung zu den speziellen Fachkräften.

\paragraph{Kultur}
Durch die Semesterbeiträge der Studierenden ist das Studentenwerk in der Lage, für kulturelle Projekte oder Initiativen Fördermittel auszugeben.
Damit wird die Hochschullandschaft um ein vielfältiges buntes Angebot an Kleinkunst reicher und kann sich als Ergänzung zu den Angeboten der Städte durchaus messen.

Das Studentenwerk organisiert im Sommersemester eine kulturelle Veranstaltung -- das Unisportfest, welches neben sportlichem Können auch Gaudi -- es gibt ein Trabi-Wettschieben -- zum Gegenstand des Spektakels hat und in Zusammenarbeit mit dem Universitätssportzentrum den Abschluss des Universitätssportfestes bildet.
Der anschließende Sportlerball erfreut sich großer Beliebtheit.

Der studentische Foto-Club „Conspectus“ steht für Interessierte jederzeit zur Ver\-fügung, wenn es um fachliche Anleitung beim Fotografieren geht.
Die Mensa Harz bietet ideale Möglichkeiten, Ausstellungen zu präsentieren und vielfältige Veranstaltungen durchzuführen.
Verschiedene interessante Projekte konnten bisher mit Fördergeldern bedacht werden:

\begin{itemize}
    \item Kino-Clubs an den verschiedenen Standorten
    \item Faschingsveranstaltungen
    \item Filmprojekte
    \item Musikveranstaltungen in den Studentenclubs
    \item Vorträge
    \item Ausstellungen
    \item Foto-Clubs
    \item Projekte ausländischer Studierender
\end{itemize}

\paragraph{Kinderbetreuung}
Zur Betreuung der Kinder studentischer Eltern unterhält das Studentenwerk an den Standorten Halle und Köthen jeweils eine Kindertageseinrichtung, in denen Studentenkinder bevorzugt aufgenommen werden und die speziell auf die Bedürfnisse studentischer Eltern eingestellt sind.
Am Standort Halle berät das Studentenwerk insbesondere auch die Beschäftigten der Martin-Luther-Universität Halle-Wittenberg bei der Antragstellung auf einen Platz in der Kindertageseinrichtung „Weinberg“ und informiert über die jeweilige Unterbringungssituation.
Der Elternbeitrag in beiden Kindertageseinrichtungen wird nach Festbeträgen ermittelt.
Es werden Kinder im Alter von 0 bis zu 6~Jahren betreut und diese können auf der Basis des pädagogischen Konzeptes der Einrichtung verschiedene Angebote wahrnehmen.

Weitere Informationen kann man unter
\web{http://www.studentenwerk-halle.de/} nachlesen.

\subsection{Stipendien}

Für viele Studenten sind Stipendien attraktiv.
Sie dienen als Fi\-nan\-zier\-ungs\-mö\-glich\-keit des Studiums und helfen vielen Studenten über die Runden, bieten aber mehr als „nur“ Geld.
Die Studienstiftungen, die Stipendien anbieten, veranstalten Seminare und Workshops zur besseren Qualifikation und begleiten Stipendiaten im Wust des bürokratischen Alltags.

Das Bundesministerium für Bildung und Forschung (BMBF) fördert eine Vielzahl solcher Stiftungen, die i.A. unabhängig von anderen Organisationen handeln, oft aber Kirchen, Gewerkschaften oder Parteien nahe stehen.
Somit zahlt sich dann auch ehrenamtliches Engagement aus.
Oft werden engagierte Leute bevorzugt gefördert.
Natürlich ist es auch hilfreich erfolgreiche Leistungen während des Studiums vorweisen zu können.

Falls dich Stipendien interessieren, kannst du gern beim Fachschaftsrat anfragen.
Im Fachschaftssraum liegen in der Regel viele Flyer von Stiftungen aus, in denen genau beschrieben ist, was man für ein Stipendium machen muss, also welche Pflichten und Rechte mit einem Stipendium einhergehen.
Es lohnt sich ein Blick auf die Website des BMBF.

\link{http://www.bmbf.de/de/11869.php}\\
\link{http://www.bmbf.de/de/294.php}\\
\link{http://www.deutschland-stipendium.de/}\\
\link{http://www.uni-halle.de/deutschland-stipendium/}

\subsection{Studentenalltag}

Zuerst soll hier gesagt sein, dass Uni nichts mit der Schule gemeinsam hat.
Dies wird nur oberflächlich durch die Tatsache negiert, dass es an unserem Fachbereich „Hausaufgaben“ gibt.
In der Schule konnte man diese ignorieren, ohne dass es zu großen Problemen kam.
Im Studium ist das anders, hier sind Hausaufgaben meist Grundlage für die folgenden Prüfungen und gleichzeitig werden sie (meist jede Woche) korrigiert und bewertet.
Solltest du meinen, dass du ohne Hausaufgaben zu machen eine Prüfung bestehen wirst, so wirst du bald merken, dass du nicht einmal zur Prüfung zugelassen wirst.
Dafür musst du in der Regel mindestens 50\% der Punkte in den Übungen haben.
Die genauen Bedingungen legt aber jeder Professor für seine Veranstaltung fest, welche in der ersten Vorlesung bekanntgegeben werden müssen -- also nicht verschlafen!

Weiter kann man nur sagen: Sei fleißig, bilde Übungsgruppen (nicht zum Abschreiben!) und wende dich bei Fragen an uns, denn nichts ist schlimmer als mitzuspielen, ohne die Spielregeln zu kennen!

\subsection{Was mache ich in meiner Freizeit?}

Wenn jemand den ganzen Tag fleißig studiert hat, darf er sich hin und wieder auch anderen Dingen widmen.
Viele Infos zu Halle und zu kulturellen Einrichtungen findet ihr meist gut sortiert unter:

\web{http://www.halle.de/}\\
\web{http://www.kulturfalter.de/}

\paragraph{Fachschaftsratsveranstaltungen}
Wir als Fachschaftsrat organisieren einige Feiern bzw. Feste für unsere Studenten und wir freuen uns immer sehr auf euch.
Einige Termine, die ihr nicht verpassen solltet:

\begin{itemize}
    \item Das Erstsemestergrillen ist speziell für unsere „Erstis“; zum Kennenlernen und um uns auszufragen.
          So werdet ihr gut informiert und mit viel Unterstützung von euren Kommilitoninnen und Kommilitonen euer Studium möglichst gut meistern!
    \item Unsere Weihnachtsfeier findet immer kurz vor den Weihnachtsferien -- meist in unserer Caféteria statt.
          Ein Kommen lohnt sich nicht nur, um einen schönen Abend mit euren Kommilitoninnen und Kommilitonen zu verbringen und neue kennenzulernen,
          sondern auch, weil wir für euch eine tolle Karaoke-Anlage aufbauen.
          Außerdem sind Glühwein und Getränke günstig, Plätzchen und Naschkram sogar umsonst.
    \item Die NatFusion ist eine gemeinsam mit den Physikern im Frühling ausgerichtete Party,
          bei der der Zusammenhalt zwischen unseren beiden Fakultäten verstärkt werden soll.
          Es gibt Musik, Ratespiele und andere Überraschungen.
    \item Unser Sommerfest, welches meist Mitte Juni stattfindet, bezeichnen wir auch oft als „Sportfest“,
          denn wir richten kleine Turniere in Volley- und Fußball aus.
          Bei diesen dürfen nur Studierende aus den Instituten für Informatik und Mathematik teilnehmen, wir sind also „unter uns“
          und ohne blöde Übersportler, die einem den Spielspass versauen könnten.\\
          Wo wir grade bei Spielspass sind: Traditionell gibt es noch den Festplattenweitwurf,
          bei dem ihr -- wie ihr schon ahnt -- eine Festplatte so weit wie möglich werfen sollt.
          Das macht -- wir sprechen aus eigener Erfahrung -- unglaublich viel Spass!
\end{itemize}

Des Weiteren organisieren wir dann und wann ein paar Spielabende und Grillfeiern, bei denen es uns einfach um ein gemütliches Zusammensein geht oder wir auf etwas hinweisen möchten, wie zum Beispiel mit unserem traditionellen Wahlgrillen.
Wir bemühen uns immer auch fleischfreies Grillgut anzubieten, damit auch die Vegetarier und Feinschmecker unter euch auf ihre Kosten kommen.

Wir hängen in jedem Fall immer Plakate aus; leider haben wir oft das Problem, dass diese trotz Allem oft übersehen werden und deswegen bitten wir euch:
Lauft mit offenen Augen durch die Gebäude, oft gibt es interessante kleine Aushänge oder eben Plakate von uns.

\paragraph{Theater}
Als größte Stadt Sachsen-Anhalts hat Halle natürlich einiges zu bieten.
Da wäre z.B. das Neue Theater (Große Ulrichstraße), das Thalia-Theater (Puschkinstraße bzw. Thaliapassage), das Opernhaus (Opernplatz), das Puppentheater und jede Menge kleinerer und größerer Theatergruppen (\;u.a. die Kiebitzensteiner als bekannte Kabarettgruppe).
Es lohnt sich auf jeden Fall mal bei ihnen reinzuschauen.
Manchmal organisieren wir auch gemeinsame Theater- bzw. Opernbesuche und verlosen Freikarten!

\paragraph{Kino}
Wer sich lieber den bewegten Bildern widmen möchte, kann dies entweder in großen Multiplex-Kinos (CinemaxX im Charlottencenter) und dem neuen 3D-Kino (thelightcinema im Neustadt Centrum) oder in vielen kleinen Programm-Kinos tun.

Multiplex-Kino: \web{http://www.cinemaxx.de}\\
3D-Kino: \web{http://www.lightcinemas.de/}\\
Programm-Kino Zazie: \web{http://www.kino-zazie.de}\\
Programm-Kino Lux: \web{http://www.luxkino.de}\\
Unikino: \web{http://www.unikino.uni-halle.de}


\paragraph{Bars und Diskos}
Wer Zerstreuung in geistigen Getränken sucht, kann dies ausgiebig auf der halleschen Barmeile tun.
Sie erstreckt sich von der Sternstraße über den Marktplatz hinweg bis zur Kleinen Ulrichstraße.
Aber auch abseits der Barenmeile ist noch viel los.
Vor allem am Universitätsring entwickelte sich in den letzten Jahren eine rege Bar-Szene.
Wer gerne Cocktails trinkt, sollte auf jeden Fall mal im Enchilada vorbei schauen.
Wer allerdings gern tanzen geht, findet im Turm, im Flower-Power und in der Tanzbar „Palette“ gute Anlaufstellen.

Wer zwar gern Musik hört, aber nicht so gern in Diskos, sondern lieber auf Konzerte geht, dem seien außerdem noch die Rockstation, das (neue) Hühnermanhattan, das Objekt 5, das Reil 78 und das VL ans Herz gelegt.
In letzteren beiden genannten findet man auch andere Beschäftigungs- und Engagementmöglichkeiten, es lohnt sich also die Webseiten mal abzuchecken.

Partytermine: \web{http://partytermine-halle.de/} \\
Enchilada: \web{http://www.enchilada.de/} \\
Turm Halle: \web{http://www.turm-halle.de/} \\
Flower-Power: \web{http://www.urania70.com/} \\
Tanzbar „Palette“: \web{http://www.tanzbar-palette.de/} \\
Rockstation: \web{http://www.rockstation-halle.com/} \\
Halle-Nightlife: \web{http://halle-nightlife.de}\\
Reil 78: \web{http://www.reil78.de/} \\
Charles Bronson: \web{http://www.wearecharlesbronson.de/}\\
Chaise: \web{http://www.chaise.de/}

\paragraph{Bauernclub}
In Halle gibt einen Studentenclub namens Bauernclub.
Dieser wird von Studenten für Studenten geführt und zeichnet sich deshalb durch eine entsprechende lockere Atmosphäre und sehr niedrige Preise aus.
Auch bietet der Bauernclub die Möglichkeit, seine Räumlichkeiten für eigene Feiern zu mieten.
Am besten man besucht ihn und macht sich selbst ein Bild.

\web{http://www.bauernclub.de}
        
\subsection{hastuzeit}
Die hastuzeit ist die hallesche Studierendenzeitschrift und wird von der Studierendenschaft der MLU herausgegeben.

In ihr könnt ihr von aktuellen Ereignissen der Hochschulpolitik und amüsanten Geschichten aus dem Studentenalltag lesen.
Auch Erfahrungsberichte von Auslandssemestern oder die Kulturszene Halles sind beliebte Themen.

Den Druck der Hefte finanziert der StuRa, sodass sie für euch kostenlos sind.
In der Regel erscheint die hastuzeit viermal im Semester und wird uniweit, zum Teil auch in den Studentenwohnheimen verteilt.
Im Institut für Informatik findet ihr sie meist in der 3. Etage oder im Fachschaftsraum (0.31).
Ihr könnt euch aber auch über ihre Homepage und per Twitter auf dem Laufenden halten.

Die fleißige Redaktion freut sich nicht nur über Leserbriefe und Anregungen, sondern ihr könnt auch eigene Beiträge verfassen oder das Team als neues Mitglied verstärken.

\web{http://www.hastuzeit.de/}\\
\web{http://twitter.com/hastuzeit}\\
\email{redaktion@hastuzeit.de}




\newpage

\section{Hilfen}
\subsection[Kommunikationswege des FSR]{Die Kommunikationswege des Fachschaftsrates}
In den letzten Jahren haben wir per facebook und Website neue Mög\-lich\-kei\-ten geschaffen, regelmäßig wichtige Informationen an euch
weiterzuleiten.
Aber auch mit den klassischen Wegen, wie der Homepage oder den Aushängen, halten wir euch über aktuelle Ereignisse und kommende Veranstaltungen
auf dem Laufenden.
Hier findet ihr eine Übersicht über unsere Kommunikationswege:

\subsubsection{Homepage}
\web{http://fachschaft.mathinf.uni-halle.de/} \\
oder kurz \web{http://www.fsr-matheinfo.de/}
\subsubsection{facebook}
%\begin{wrapfigure}{r}{0.25\textwidth}\centering
%	\includepdf[pages=-,pagecommand={},width=100pt,offset=180 -75]{Bilder/qrcode_facebook.pdf}
%\end{wrapfigure}
Seite: \verweis{Fachschaftsrat Mathe/Info Halle} \\
\web{https://www.facebook.com/fsr.matheinfo/} \\
\subsubsection{Aushänge}
Pinnwände der 3. Etage im Institut für Informatik \\
Korkwand im Fachschafts-Raum (Raum~0.31) \\
und manchmal auch reichlich in den Instituten verteilt

Wenn ihr etwas auf dem Herzen habt, könnt ihr uns auch jederzeit mit einer E-mail an \email{fachschaft@mathinf.uni-halle.de} erreichen.
Zudem könnt ihr bei schwerwiegenden Problemen oder auch einfach aus Interesse bei den öffentlichen Sitzungen vorbeischauen.
In der Vorlesungszeit finden diese meist wöchentlich, in der vorlesungsfreien Zeit unregelmäßig, alle zwei bis drei Wochen, im Raum 0.31 statt.
Die genauen Termine findet ihr auf der Homepage.

\subsection[Mathe-Treffpunkt]{Mathe-Treffpunkt für Studierende der Mathematik und Informatik}

Liebe Erstis,

wir, das Team vom Mathe-Treffpunkt, heißen Euch herzlich willkommen und 
wollen Euch kurz den Mathe-Treffpunkt vorstellen. Seit einem Jahr ist 
dieser nun erfolgreich im Betrieb. Aber was ist ein Mathe-Treffpunkt? Es 
handelt sich dabei um einen Treffpunkt (wer hätte das gedacht?) für 
Mathematik- und Informatik- Studierende im ersten und zweiten Semester, 
die Fragen zu den Mathematik-Vorlesungen haben, gemeinsam Aufgaben lösen 
oder sonstige Probleme rund ums Studium besprechen wollen. Der 
Mathe-Treffpunkt im Raum 1.30 des Cantor-Hauses ist jeden Tag bis zu 5~Stunden geöffnet und während 
dieser Öffnungszeiten könnt ihr immer einen Tutor oder eine Tutorin aus 
einem höheren Semester dort antreffen. Auch außerhalb der Öffnungszeiten 
könnt ihr den Raum zum Arbeiten nutzen. Die Öffnungszeiten und weitere 
Informationen findet ihr im StudIP. Tragt Euch dort einfach in die 
Studiengruppe "`Mathe-Treffpunkt"' ein.
Wir freuen uns, Euch kennenzulernen!

Euer Mathe-Treffpunkt-Team

\begin{center}
    \includegraphics[width=0.4\linewidth]{logo_mathe_treff}%
\end{center}

\subsection[Mentoringprogramm]{Mentoringprogramm für Studierende der Bioinformatik}

Zu Beginn des Bachelorstudiengangs 
wird das Mentoringprogramm angeboten. 
Ziel ist es zum einen, den Studierenden 
Informationen über Richtlinien und Aufbau des Studiums zu geben. 
Dazu erfolgt während des ersten Semesters 
das Treffen zwischen dem Mentor und einer Studentengruppe,
bestehend aus drei bis vier Studenten, 
im zwei Wochen Rhythmus.
Im darauffolgenden zweiten Semester 
finden die Treffen aller vier Wochen statt.

Während des Mentorings erhalten die Studenten 
Erläuterungen zu existierenden Abhängigkeiten zwischen Modulen, 
sowie Informationen zu den Wahlmöglichkeiten 
für die Allgemeinen Schlüsselqualifikationen 
und den Wahlpflichtmodulen. 
Im Weiteren wird ebenfalls auf das Verfahren der
Modul- und Prüfungsanmeldung eingegangen. 
Ein weiteres Ziel ist es, bereits frühzeitig einen Einblick 
in verschiedene Themengebiete der Bioinformatik 
zu gewähren und die Dozenten der Bioinformatik kennen
zu lernen. 
Dafür stellen sich die Dozenten der Bioinformatik 
in einer Diskussionsrunde vor und geben
Einblick in ihre Forschungsthemen.

Ein weiteres Kernelement des Mentorings 
ist die Vermittlung von Techniken für ein erfolgreiches Studium. 
Dazu zählen Erläuterungen zu Strategien von Gruppenarbeit, 
Lerntechniken für Prüfungen, Zeitmanagement sowie der Umgang 
mit negativen Situationen. 
Während des gesamten Mentorings sollten die Mentoren 
als Vertrauenspersonen gelten, bei denen die Studierenden 
ihre Probleme im Studium (beispielsweise bei Vorlesungen) 
oder offene Fragen ansprechen können. 
Dieses Vertrauensverhältnis soll auch nach dem ersten Studienjahr 
beibehalten werden.

\subsection{Wie erstelle ich ein PDF?}

Die Lösungen zu euren wöchentlichen Hausaufgaben müsst ihr meist gedruckt vorlegen oder als PDF einsenden.
Im Folgenden werde ich euch erklären wie ihr am besten ein PDF erstellt und dabei einwandfreie mathematische Formeln darstellt.
Es gibt mehrere Möglichkeiten ein PDF zu erstellen:

\begin{enumerate}
    \item mit OpenOffice bzw. LibreOffice.\\
          Dokumente, die mit Programmen dieser Bürosoftware erstellt werden, können mit einem Knopfdruck bequem exportiert werden.
          OpenOffice und Libreoffice sind einfach zu bedienen und mit allen wichtigen Funktionen ausgestattet.\\
          \web{http://de.libreoffice.org/}\\
          \web{http://de.openoffice.org/}
    \item mit \LaTeX.\\
          Um \LaTeX~(sprich: 'latech) kommen die Studierenden in unserer Fachschaft nicht herum. 
          Es ist ein System das aus der Textsatztechnik entstanden ist.
          Es bietet alle Funktionen, die man brauchen könnte und sieht unglaublich professionell im Ergebnis aus.
          Allerdings braucht es eine gewisse Arbeit sich mit dem System vertraut zu machen.
          Dieses Heft wurde auch mit \LaTeX~gestaltet und ihr seht die Möglichkeiten:
          \begin{eqnarray*}
               A & = & \left(\begin{array}{ccc} a_{11} & \cdots & a_{1n}\\
                                                    \vdots & \ddots & \\
                                                   a_{m1} &        & a_{mn} \end{array}\right)\\
               \sum_{i=1}^n i & = & \frac{n(n+1)}{2}\\
               \int_{0}^{3}x^2dx & = & 9\\
               \forall n\in\mathbb{N}:\lim
                                  \limits_{m\to \infty}\frac{n}{m} & = & 0
          \end{eqnarray*}
    \item mit MS Word kann man direkt im Format PDF exportieren. Es ist jedoch sehr mühsam, Formeln mit Word zu erstellen.
%    \item mit einem „virtuellen Drucker“\\
%          Was hierbei hochtrabend klingt, ist eigentlich sehr einfach.
%          Man installiert ein Programm und kann dann aus einem beliebigen Programm, z.B. Microsoft Word, Dokumente in ein PDF drucken,
%          so dass dieser Druck nicht auf Papier, sondern als Datei vorliegt.
%          Einige freie virtuelle Drucker findet ihr unter\\
%          \web{http://www.softonic.de/s/virtueller-drucker}
%    \item mit einem Konvertierer.\\
%          Es gibt auch Programme, die PDFs aus anderen Formate, z.B. MS Word, konvertieren, z.B. WordToPDF Pro.\\
%          \web{http://www.topdf.de/DownloadD.htm}
\end{enumerate}

\paragraph{\LaTeX}
Im Folgenden möchte ich nur auf \LaTeX~genauer eingehen, da es das professionellste Tool zum Erzeugen von Texten und Formeln ist, viele aber von der ungewohnten Bedienung abgeschreckt werden.
Hier wird nur auf die Installation unter Windows eingegangen, da Windows-User meist noch keinen Kontakt mit \LaTeX~hatten und oft technisch unerfahrener sind.
GNU/Linux-User finden unter den Stichworten texlive, gedit (Gnome) und Kile (KDE) Hilfe, Mac~OS~X-User sollten sich an TeX-Shop halten.

\paragraph{Installation}
Als erstes muss man unter Windows zwei Programme installieren, bevor man starten kann.
Das erste ist die eigentliche \LaTeX -Software.
Also das Programm, welches am Ende die pdf-Dateien erstellt.
Das Zweite ist dann ein Editor, der euch mit Autovervollständigung und Knöpfen zum Erstellen der Dokumente unterstützt.

\textbf{\LaTeX -Software:}
\begin{itemize}
    \item MiK\TeX~enthält das eigentliche \LaTeX-System und alle wichtigen Pakete, ist jedoch auf das Betriebssystem Windows beschränkt \\
        \link{http://miktex.org/}\\
        Wenn man die Basic-Version installiert, dann muss man immer, wenn benötigte Pakete nachgeladen werden, eine Internetverbindung haben.
        Wählt man das komplette MiK\TeX, so muss man keine Pakete mehr nachladen.
        Dafür benötigt das ganze mehr Festplatten-Speicher.
    \item \TeX~Live ist ein alternatives System, ähnlich zu MiK\TeX, dafür aber auch unter anderen Betriebssystemen verwendbar \\
        \link{www.tug.org/texlive/}
\end{itemize}

\textbf{Editoren:}
\begin{itemize}
    \item \TeX nicCenter benötigt als Grundlage MiK\TeX und kann daher auch nur unter Windows verwendet werden \\
        \link{www.texniccenter.org/}
    \item \TeX maker wird von vielen Linux-Anwendern verwendet, da es standardmäßig Unicode als Kodierung verwendet \\
        \link{www.xm1math.net/texmaker/}
    \item \TeX works ist ähnlich zu \TeX maker und unterscheidet sich nur leicht davon \\
        \link{www.tug.org/texworks}
\end{itemize}

Man installiert \textbf{zuerst} das \LaTeX-Programm \textbf{und dann} den Editor.
Normalerweise muss man einfach den Anweisungen in den Installern folgen.
Bei Problemen mit \LaTeX~hilft es oft, google zu fragen.

\paragraph{Den ersten Text setzen}
Der große Unterschied zu anderen Texverarbeitungssystemen ist, dass nicht wie in Word der Text direkt formatiert angezeigt wird („What you see is what you get“), sondern dass man die Formatierung nur „beschreibt“.
Ähnlich wie in HTML schreibt man alle Formatierungen mit in den Text und erst beim Setzen entsteht das fertige Dokument („What you see is what you mean“).
Davon darf man sich aber nicht abschrecken lassen.
Wenn ihr die folgenden Schritte befolgt, habt ihr in wenigen Minuten euer erstes PDF-Dokument selbst erzeugt.
Die Anweisungen beziehen sich auf das \TeX nicCenter, in den anderen Editoren können die Schaltflächen anders angeordnet sein und verschieden heißen.

\begin{enumerate}
 \item Öffnet euer \TeX nicCenter.
 \item Stellt in der Werkzeugleiste in dem Feld, in dem \LaTeX $=>$ DVI steht, \LaTeX $=>$ PDF ein.
 \item Öffnet ein neues Dokument und gebt folgende Zeilen ein:
    \begin{verbatim}
            \documentclass[a4paper]{article}
            \begin{document}
                    Hallo Welt!
            \end{document}
    \end{verbatim}
%Verbatim ignoriert Tabs, deshalb zum Einrücken 8 Leerzeichen benutzen
 \item Anschließend müsst ihr euer Dokument übersetzen (in Fachkreisen auch „setzen“ oder „\TeX en“ genannt),
       dazu klickt ihr im Menü „Ausgabe - Aktives Dokument -
       Erstellen und betrachten“. Euer gesetztes Dokument sollte
       gleich angezeigt werden.
\end{enumerate}

\paragraph{Die Lösungen zu den Übungen schreiben}
Dazu verwendet ihr am besten ein Vorlage, in der schon alle wichtigen Pakete (z.B. für Matheformeln) eingebunden sind.
Ihr findet eine auf der Seite des Fachschaftsrates:\\
\web{http://fachschaft.mathinf.uni-halle.de/informationen/latex}\\
In diese Vorlage tragt ihr euren Namen, eure Matrikelnummer und die Nummer der Übungsserie ein, dann könnt ihr loslegen.
Außerdem sind in der Vorlage noch ein paar Beispiele aufgeführt.

Probiert einfach ein bisschen rum.
Am besten ihr beschäftigt euch so zeitig wie möglich in eurem Studium damit, dann beherrscht ihr \LaTeX~, wenn es darauf ankommt.
Spätestens zur Bachelorarbeit kommt ihr nicht mehr um \LaTeX~herum.

\subsection[Computerpools]{Die Computerpools}
Im Informatikgebäude gibt es 
in der dritten Etage zwei Computerpools, 
die für Lehrveranstaltungen 
und die Studenten der Institute für Mathematik 
und Informatik vorgesehen sind:

\begin{itemize}
    \item PC-Pool (Raum 3.34) mit Linux-PCs (Xubuntu)
    \item Uni-Pool (Raum 3.32) mit Windows-PCs (Windows 7)
\end{itemize}

In den Pools ist jede Software, die ihr gebrauchen könntet 
und vielleicht sogar unter einer teuren Lizenz steht, installiert.
Außerdem könnt ihr aus den Pools heraus die Drucker nutzen:

Es stehen jedem Studenten der mathematischen Studiengänge 
einmalig zum ersten Semester 100~Druckseiten kostenlos zur Verfügung.
Für die Studenten der Informatik und Bioinformatik sind es 
sogar 200~Seiten pro Semester.

Benutzername und Passwort stehen im Bestätigungsbrief 
zu eurer Immatrikulation.
Bei Problemen mit dem Login könnt ihr euch 
an die Poolaufsicht (Raum~3.18) wenden.
Weitere Informationen zu den Pools, verfügbarer Software, 
Diensten u.a. kann man nachlesen unter:\\
\web{http://www.informatik.uni-halle.de/studium/pools/}

\subsection{WLAN}

In den meisten Unigebäuden und auf dem Uniplatz gibt es WLAN für alle Studenten.
So auch in den Gebäuden der Mathematik und Informatik.
Benutzt einfach die Zugangstechnik 802.1X , wenn ihr euch mit dem WLAN namens „uni-halle“ verbindet.
Für technische Details verweise ich euch einfach an die entsprechenden Wikipedia-Artikel.

Die Uni stellt gut verständliche, bebilderte Anleitungen für die meisten Betriebssysteme zur Verfügung.
Diese erreicht ihr von zu Hause unter\\
\web{http://wlan.urz.uni-halle.de/}.



\newpage

\section{Anhang}

\subsection{Uni-ABC}

\begin{description}
\item[AAA] Akademisches Auslandsamt (\web{http://aaa.verwaltung.uni-halle.de/})
\item[akademisches Viertel] siehe~c.t.
\item[BAFöG] Kürzel für „Bundesausbildungsförderungsgesetz“.
             Regelt die finanzielle Unterstützung von Studierenden durch den Staat in Form von Stipendium und Darlehen.
             Für immer weniger Studierende Hauptfinanzierungsquelle des Studiums.
\item[c.t.] Abkürzung für „cum tempore“.
            Der zugehörigen Uhrzeit wird das sogenannte „akademische Viertel“ zugerechnet.
            Eine mit 14.00~c.t. ausgezeichnete Veranstaltung beginnt erst 14.15~Uhr.
\item[CP] Abkürzung für „Credit Points“ (Leistungs-Punkte). Wichtige Einheit in den Studienordnungen der Bachelor/Master-Stdiengänge.
          s.~Kapitel~\ref{creditpoints}: Was sind Leistungs-Punkte (LP) bzw. Credit Points (CP)?
\item[Dekan] lat. „Führer von zehn“. Das Oberhaupt einer Fakultät, Sprecher des Fakultätsrates.
\item[dies academicus] lat. „akademischer Tag“. Stellt einen Feiertag an der Uni dar.
                       Während des „dies“, wie er kurz heißt, finden keine Vorlesungen oder Seminare statt.
                       Der dies findet ein- oder zweimal pro Semester zu solchen Veranstaltungen wie Sportfest,
                       Universitätsfest bzw. StudentInnenfestival oder feierlicher Immatrikulation statt.
\item[Fakultät] Gruppierung von verschiedene Instituten. Die Uni besteht aus 10~Fakultäten.
                Das Institut Mathematik befindet sich in der Naturwissenschaftlichen Fakultät~II (NatFak~II)
                und das Institut der Informatik in der Naturwissenschaftlichen Fakultät~III (NatFak~III).
\item[FSR] Abkürzung für „Fachschaftsrat“. Studentische Interessenvertretung am Institut.
           (s.~Kapitel~\ref{vertretung}: Wer vertritt wen?)
\item[HiWi] Kurz für „Hilfswissenschaftler“, also eine wissenschaftliche Hilfskraft.
            HiWis sind in der Regel selbst Studenten, meist höheren Semesters.
\item[(IT)$^2$] Der „Industrietag Informationstechnik“ ist eine Veranstaltung des Universitätszentrum für Informatik (UZI)
                mit der Industrie- und Handelskammer (IHK) und findet einmal pro Semester statt.
\item[Kanzler] Verwaltungschef der Uni.
\item[Kommilitone] lat. „Waffenbruder“, Mitstudent, Studiengenosse, existiert auch in der weiblichen Form Kommilitonin.
\item[LHG] Abkürzung für „Landeshochschulgesetz“.
           Regelt die Strukturen an den Hochschulen in Sachsen-Anhalt, definiert Inhalt des Studiums, legt fest, wer studieren darf etc.
\item[Mensa] Kurz für „mensa academica“, einem Speisehaus für StudentInnen mit verbilligtem Essen.
\item[MLU] Kürzel für die „Martin-Luther-Universität Halle-Wittenberg“
\item[Rektor] Akademisches Oberhaupt und erster Repräsentant der Universität
\item[Senat] Beschlussfassendes Organ der akademischen Selbstverwaltung.
             Zuständig für die meisten Entscheidungen in Hinblick auf Forschung und Studium.
\item[SoSe] Häufig genutzte alternative Abkürzung für das Sommersemester.
\item[SS] Ist das vom 1.4.--30.09. dauernde Sommersemester.
\item[s.t.] Abkürzung für „sine tempore“, d.h. die zugehörige Uhrzeit ist ohne akademisches Viertel zu verstehen.
\item[StuRa] Abkürzung für „Studierendenrat“. Oberstes Organ der studentischen Selbstverwaltung an ostdeutschen Hochschulen.
             In anderen Universitäten heißt er meist AStA (Allgemeiner Studentenausschuss).\\
             \web{http://www.stura.uni-halle.de/}
\item[SWS] Abkürzung für „Semesterwochenstunden“. Hier werden die Stunden je Woche in einem Semester zusammengezählt.
           Habt ihr eine Vorlesung mit 4+2~SWS, dann heißt das, dass ihr in einer Woche 4~Stunden Vorlesung und 2~Stunden Übung habt.
\item[ULB] Universitäts- und Landesbibliothek\\
           \web{http://bibliothek.uni-halle.de/}
\item[URZ] Universitätsrechenzentrum\\
           \web{http://www.urz.uni-halle.de/}
\item[USZ] Universitätssportzentrum\\
           \web{http://www.usz.uni-halle.de/}
\item[UZI] Universitätszentrum für Informatik\\
           \web{http://www.uzi.uni-halle.de/}
\item[WiSe] Häufig genutzte alternative Abkürzung für das Wintersemester.
\item[WS] Ist das vom 1.10.--31.03. dauernde Wintersemester.
\end{description}

\subsection{Nützliche Links}
\begin{description}
\item[\web{http://www.fsr-matheinfo.de/}]
    Das ist die Seite des Fachschaftsrats für Mathematik und Informatik.
    Hier findet ihr viele nützliche Informationen zu eurem Studium und auch immer Hilfe bei Problemen.
\item[\web{http://www.uni-halle.de/}]
    Die Seite der Uni. Von hier aus findet man alle Informationen, die irgendwie mit der Universität zu tun haben.
\item[\web{http://www.informatik.uni-halle.de/}]
    Das ist die Seite des Instituts für Informatik.
\item[\web{http://www.mathematik.uni-halle.de/}]
    Das ist die Seite des Instituts für Mathematik.
\item[\web{http://www.stura.uni-halle.de/}]
    Die Seite des Studierendenrats. Hier findet ihr alle Informationen zu dem wichtigsten Studentengremium und mehr.
\item[\link{https://www.studip.uni-halle.de}]
    Das Vorlesungsverzeichnis der Uni.
    Hier bekommt ihr Informationen zu Veranstaltungen, schreibt euch in diese ein und erhaltet Termine, Dateien sowie Hilfe dazu.
\item[\link{https://studmail.uni-halle.de}]
    Das ist die Seite eurer Uni-Mailadressen.
    Über diesen Account (deren Zugangsdaten euch per Brief zugesandt wurden) erhaltet ihr alle wichtigen Mails.
    Ihr solltet daher hier öfter mal reinschauen oder ihr leitet euch diese Mails auf euer normales E-Mail-Konto weiter.
\item[\link{https://loewenportal.uni-halle.de}]
    Die Selbstbedienungsfunktion der Uni, neuerdings auch „Löwenportal“ genannt.
    Hier könnt ihr euch Immatrikulationsbeschneigungen und andere Belege ausdrucken und für Module anmelden.
%\item[\web{http://www.informatik.uni-halle.de/studenten-infos/pruefungsamt/}]
%    Das Prü\-fungs\-amt unserer Institute. Hier findet ihr Prüfungstermine und Ergebnisse.
\item[\web{http://www.bibliothek.uni-halle.de/}]
    Das ist der Internetauftritt unserer Bibliothek. Hier kann man Literatur suchen, vorbestellen und Ausleihfristen verlängern.
\item[\web{http://www.usz.uni-halle.de/}]
    Hier könnt ihr euch über das Sportangebot an der Uni informieren und euch für kostenpflichtige Sportkurse anmelden.
\item[\web{http://www.studentenwerk-halle.de/}]
    Das Studentenwerk Halle setzt sich für die soziale, wirtschaftliche, kulturelle und gesundheitliche Förderung der Studierenden ein.
    Hier findet ihr Speisepläne, Hilfe zum BAföG und vieles mehr.
\end{description}

\subsection{Wichtige Kontakte}

\subsubsection{Fachschaftsrat Mathematik/Informatik}
Von-Seckendorff-Platz 1, Raum 0.31\\
Sprechzeiten: Dienstag 09:00--10:00~Uhr\\
Tel: (0345) 55 24605\\
\email{fachschaft@mathinf.uni-halle.de}

\subsubsection{Prüfungsamt der NatFak III (incl. Informatik)}
Erika Brandt\\
Von-Seckendorff-Platz 4\\
Raum: 4.25\\
Sprechzeiten: Dienstag und Donnerstag, 9.00~Uhr bis 11.00~Uhr und
13.00~Uhr bis 15.00~Uhr\\
Tel: (0345) 55 24711\\
\email{pruefungsamt@informatik.uni-halle.de}\\
\web{http://www.natfak3.uni-halle.de/103\_2272405/}

\subsubsection{Prüfungsamt der NatFak II (incl. Mathematik)}
Kerstin Baranowski\\
Von-Danckelmann-Platz 3
Sprechzeiten: Dienstag und Donnerstag 10.30~Uhr bis 11.30~Uhr, sowie
Dienstag, Mittwoch und Donnerstag 13.30~Uhr bis 16.00 Uhr\\
Tel: (0345) 55 25600\\
\email{kerstin.baranowski@natfak2.uni-halle.de}\\
\web{http://www.natfak2.uni-halle.de/studium/}

\subsubsection{Studienberater Mathematik}
Dr. Hans-Georg Rackwitz\\
Herrenstr. 20, Raum 170\\
Tel: (0345) 55 24608\\
\email{hans-georg.rackwitz@mathematik.uni-halle.de}\\


\subsubsection{Studienberater Informatik}
Dr. Christoph Bauer\\
Von-Seckendorff-Platz 1, Raum 2.21\\
Tel: (0345) 55 24718\\
\email{christoph.bauer@informatik.uni-halle.de}

\subsubsection{Studierendenrat}
Universitätsplatz 7\\
Tel: (0345) 55 21411\\
\email{stura@uni-halle.de}
\web{http://www.stura.uni-halle.de/studierendenrat/kontakt/}

\subsubsection{BAföG-Amt}
Wolfgang-Langenbeck-Str. 3\\
Sprechzeiten: Dienstag 09:00~Uhr bis 18.00~Uhr,Montag/Mittwoch/Donnerstag 09.00~Uhr bis
15.00~Uhr und Freitag 09.00~Uhr bis 13.00~Uhr\\
Tel: (0345) 68 47113\\
\email{bafoeg@studentenwerk-halle.de}\\
\web{http://www.studentenwerk-halle.de/bafoeg/bafoeg/}

\subsubsection{Studierenden-Service-Center}
Universitätsplatz 11 (im Löwengebäude)\\
Hier findet ihr das Immatrikulationsamt und die allgemeine
Studienberatung.\\
Öffnungszeiten: Montag bis Donnerstag 10.00~Uhr bis 16.00~Uhr, sowie
\web{http://www.uni-halle.de/ssc/}

\subsubsection{Akademisches Auslandsamt}
Barfüßerstraße 17\\
Sprechzeiten: Dienstag 10.00~Uhr bis 12.00~Uhr und 13.00~Uhr bis
17.00~Uhr, sowie Donnerstag 10.00~Uhr bis 12.00~Uhr und 13.00~Uhr bis
15.00~Uhr\\
Tel: (0345) 55 21313\\
\email{auslandsamt@uni-halle.de}\\
\web{http://aaa.verwaltung.uni-halle.de/}




% weiße Seiten, damit der farbige Einband nicht bedruckt wird
\newpage
\thispagestyle{empty}
~
\newpage
\thispagestyle{empty}
~
\newpage
\thispagestyle{empty}
~

\end{document}
